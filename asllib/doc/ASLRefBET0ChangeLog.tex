\section{BET0}

The following changes have been made.

\subsection{ASL-796: introduce \texttt{readonly} and \texttt{pure} keywords}

These keywords (known as ``qualifiers'') optionally annotate subprograms:
\begin{lstlisting}
readonly func Foo{params}(args) => return_type
pure func Bar{params}(args) => return_type

accessor Baz() <=> value : return_type
begin
    readonly getter
        ...
    end;

    ...
end;
\end{lstlisting}

\noindent
A subprogram marked \texttt{readonly} or \texttt{pure} identifies a \readonly{} or \pure{} subprogram respectively.
These have the following behaviour:
\begin{itemize}
  \item A \readonly{} subprogram does not modify global storage elements or throw exceptions, but can read global storage elements and use non-determinism ($\ARBITRARY$).
    It can only call read-only or pure subprograms.
  \item A \pure{} subprogram does not read or modify global storage elements, throw exceptions, or use non-determinism. It can only call pure subprograms. It can therefore be considered read-only too.
  \item Both pure and read-only subprograms may use assertions (including asserting type conversions), or cause dynamic errors (e.g.\ divide by zero, calling the \unreachablestatementterm{}, etc.).
\end{itemize}

\noindent
These keywords are used as the basis for a more coarse-grained side-effect analysis.
In particular, they are used to define whether expressions are \pure{} or \readonly{}, and then:
\begin{itemize}
  \item Assertions, \texttt{for}-loop bounds, and slices must involve only read-only subprograms.
  \item \hyperlink{def-staticeval}{Static evaluation} requires pure expressions.
\end{itemize}

\noindent
The previous side-effect analysis is weakened in two ways:
\begin{itemize}
  \item \texttt{for}-loop bounds may be non-deterministic; and
  \item assertions may be used in symbolically evaluable expressions.
\end{itemize}

\noindent
For overloading and overriding, qualifiers must match.
For accessors, the \texttt{getter} part can be annotated \texttt{readonly}, but otherwise no qualifiers are permitted.

\subsection{ASL-806: refining \texttt{collection}s}

Collection types cannot be named; they must be anonymous.
They can only be declared at the top-level (as global declarations), with the following syntax:
\begin{lstlisting}
  var PSTATE : collection {
    N : bits(1),
    ...
  }
\end{lstlisting}

\subsection{ASL-812: token refinements}

The following invalid forms of syntax are now more clearly forbidden \emph{during lexical analysis}, rather than in parsing/type-checking as before.
\begin{itemize}
  \item \texttt{0xY} where \texttt{Y} is not a hexadecimal character
  \item \texttt{1.Z} where \texttt{1} represents any decimal digit and \texttt{Z} is not a decimal digit
\end{itemize}
See \chapref{LexicalStructure} for precise details.\\

\noindent
The following operators have been changed:
\begin{itemize}
  \item Implication: \texttt{{-}{-}{>}} has become \texttt{{=}{=}{>}}
  \item Boolean equality: \texttt{{<}{-}{>}} has become \texttt{{<}{=}{>}}
\end{itemize}

\subsection{ASL-823: control-flow analysis across subprograms}

A new \texttt{noreturn} keyword can be used as a qualifier to annotate subprograms which always terminate abnormally (i.e.\ by throwing an exception or invoking the \unreachablestatementterm).
Control-flow analysis is aware of these subprograms, and so permits previously-forbidden examples like the following:
\begin{lstlisting}
noreturn func Failure()
begin
  Unreachable();
end;

func Foo(b: boolean) => integer
begin
  if b then
    return 0;
  else
    Failure();
  end;
end;
\end{lstlisting}

\subsection{ASL-829: pending constrained integer syntax}

The syntax for the \pendingconstrainedintegertype{} has been changed from \texttt{integer\{-\}}  to \texttt{integer\{\}}.

\subsection{ASL-830: local constants}

Local \texttt{constant} declarations are no longer permitted.

\subsection{ASL-831: side-effects in slices}

Slice expressions are no longer permitted to be side-effecting.
In the parlance of ASL-796 above, they must be \readonly{}.

\subsection{ASL-836: constrained array sizes}

Sizes of arrays must be \constrainedintegers{}.
This brings them in line with widths of \bitvectortypesterm{}.

\subsection{ASL-840: parameter eliding for global declarations}

Eliding of parameters is now permitted for global declarations as well as local declarations.
For example:
\begin{lstlisting}
let w: bits(64) = Zeros{};
var x: bits(64) = Zeros{};
constant y: bits(64) = Zeros{};
config z: bits(64) = Zeros{};
\end{lstlisting}


\subsection{ASL-846: declaring multiple global variables}

Multiple global variables can be declared simultaneously without initialisation, mirroring local variables.
For example:
\begin{lstlisting}
var x, y, z : integer;
\end{lstlisting}


\subsection{ASL-847: declaring empty records/exceptions}

Records or exceptions without fields must be explicitly declared as empty using the syntax \texttt{Name\{-\}}.
This brings them in line with record/exception construction.
For example:

\begin{lstlisting}
type Foo of record;       // ERROR
type Bar of record{-};    // OK
type Baz of exception;    // ERROR
type Qux of record{-};    // OK
\end{lstlisting}


\subsection{ASL-849: statements that resemble functions}

The constructs \texttt{Unreachable()}, \texttt{print(...)}, and \texttt{println(...)} resemble functions but are actually statements.
They have been promoted to keywords, used as follows:
\begin{lstlisting}
  unreachable;
  print a, b, c;
  println x, y, z;
\end{lstlisting}
