\chapter{Global Storage Declarations\label{chap:GlobalStorageDeclarations}}

Global storage declarations are grammatically derived from $\Ndecl$ via the subset of productions shown in
\secref{GlobalStorageDeclarationsSyntax} and represented as ASTs via the production of $\decl$
shown in \secref{GlobalStorageDeclarationsAbstractSyntax}.
%
Global storage declarations are typed by $\declareglobalstorage$, which is defined in \TypingRuleRef{DeclareGlobalStorage}.
%
The semantics of a list of global storage declarations is defined in \SemanticsRuleRef{EvalGlobals},
where the list is ordered via \SemanticsRuleRef{BuildGlobalEnv}.
The semantics of a single global storage declarations is defined in \SemanticsRuleRef{DeclareGlobal}.

\RequirementDef{GlobalStorageCycles} \identr{TTMQ}
The declarations, and any associated initialization expressions, for global storage elements must not include cycles.

The specifications in \listingref{GlobalStorageCycles-bad1} and \listingref{GlobalStorageCycles-bad2}
are illegal, since they contain cycles.
\ASLListing{Cyclic global storage declarations}{GlobalStorageCycles-bad1}{\definitiontests/GuideRule.GlobalStorageCycles.bad1.asl}
\ASLListing{Cyclic global storage declarations}{GlobalStorageCycles-bad2}{\definitiontests/GuideRule.GlobalStorageCycles.bad2.asl}

\section{Configurable Global Storage Declarations\label{sec:ConfigurableGlobalStorageDeclarations}}
Global storage declarations with keyword \texttt{config} aim to assist implementation-specific support for ``configuration'' of an ASL specification.
In particular, implementations may provide mechanisms to override \texttt{config} values, such as:
\begin{itemize}
  \item Preprocessing source code before typechecking of a specification to rewrite initialisation expressions.
  \item Internally modifying \texttt{config} values after typechecking but before evaluation of a specification.
\end{itemize}
Note that if modifying values after typechecking, care must be taken to ensure that the modified values remain compatible with the expected types of the original values.
For example, a \texttt{config} value declared with type \verb|integer{0..10}| can be modified to \texttt{5} but should not be modified to \texttt{11}.
This is because as \texttt{5} is compatible with \verb|integer{0..10}|, but \texttt{11} is not.

The language supports such overriding mechanisms (and in particular, simplifies tracking of types for \texttt{config} storage elements) as follows:
\begin{itemize}
  \item The \texttt{config} keyword syntactically identifies configurable storage elements.
  \item Types of \texttt{config} storage elements must be \emph{both} explicitly declared \emph{and} singular (\TypingRuleRef{SingularType}).
  \item The \purity{} of \texttt{config} storage elements is $\SEReadonly$, so their values are not relied upon by typechecking.
  \item Values of \texttt{config} storage elements must have \purity{} of $\SEPure$, so they can depend only on constant values (and not other \texttt{config} storage elements for example).
\end{itemize}

%%%%%%%%%%%%%%%%%%%%%%%%%%%%%%%%%%%%%%%%%%%%%%%%%%%%%%%%%%%%%%%%%%%%%%%%%%%%%%%%
\section{Syntax\label{sec:GlobalStorageDeclarationsSyntax}}

\RequirementDef{DiscardingGlobalStorageDeclarations}
Global storage declarations must bind a name.
This is ensured by the ASL grammar.
All the global storage declarations in \listingref{local-storage-discards} are illegal, as they discard their declared storage elements.
\ASLListing{Illegal global storage declarations that discard storage elements}{global-storage-discards}{\syntaxtests/GuideRule.DiscardingGlobalStorageDeclarations.asl}

%%%%%%%%%%%%%%%%%%%%%%%%%%%%%%%%%%%%%%%%%%%%%%%%%%%%%%%%%%%%%%%%%%%%%%%%%%%%%%%%
\begin{flalign*}
\Ndecl  \derives \ & \Nglobaldeclkeywordnonconfig \parsesep \Tidentifier \parsesep \option{\Tcolon \parsesep \Nty} \parsesep &\\
        & \wrappedline\ \Teq \parsesep \Nexpr \parsesep \Tsemicolon &\\
  |\ & \Tconfig \parsesep \Tidentifier \parsesep \Tcolon \parsesep \Nty \parsesep \Teq \parsesep \Nexpr \parsesep \Tsemicolon &\\
  |\ & \Tvar \parsesep \Tidentifier \parsesep \Tcolon \parsesep \Ntyorcollection \parsesep \Tsemicolon&\\
  |\ & \Nglobaldeclkeyword \parsesep \Tidentifier \parsesep \Tcolon \parsesep \Nty &\\
    & \wrappedline\ \Teq \parsesep \Nelidedparamcall \parsesep \Tsemicolon &\\
  |\ & \Tpragma \parsesep \Tidentifier \parsesep \ClistZero{\Nexpr} \parsesep \Tsemicolon&
\end{flalign*}

\begin{flalign*}
\Nglobaldeclkeywordnonconfig \derives \ & \Tlet \;|\; \Tconstant \;|\; \Tvar &
\end{flalign*}

%%%%%%%%%%%%%%%%%%%%%%%%%%%%%%%%%%%%%%%%%%%%%%%%%%%%%%%%%%%%%%%%%%%%%%%%%%%%%%%%
\section{Abstract Syntax\label{sec:GlobalStorageDeclarationsAbstractSyntax}}
%%%%%%%%%%%%%%%%%%%%%%%%%%%%%%%%%%%%%%%%%%%%%%%%%%%%%%%%%%%%%%%%%%%%%%%%%%%%%%%%
\begin{flalign*}
\decl \derives\ & \DGlobalStorage(\globaldecl) &\\
\globaldecl \derives\ &
{\left\{
  \begin{array}{rcl}
  \GDkeyword &:& \globaldeclkeyword, \\
  \GDname &:& \identifier,\\
  \GDty &:& \ty?,\\
  \GDinitialvalue &:& \expr?
  \end{array}
  \right\}
 } &\\
 \globaldeclkeyword \derives\ & \GDKConstant \;|\; \GDKConfig \;|\; \GDKLet \;|\; \GDKVar &
\end{flalign*}

\ASTRuleDef{GlobalStorageDecl}
\begin{mathpar}
\inferrule[global\_storage]{
  \buildglobaldeclkeywordnonconfig(\keyword) \astarrow \astof{\keyword}\\
  \buildoption[\buildasty](\tty) \astarrow \ttyp\\
  \buildexpr(\vinitialvalue) \typearrow \astof{\vinitialvalue}
}{
  {
      \builddecl\left(\overname{\Ndecl\left(
      \begin{array}{r}
      \namednode{\vkeyword}{\Nglobaldeclkeywordnonconfig}, \\
      \wrappedline\ \Tidentifier(\name), \\
      \wrappedline\ \namednode{\tty}{\option{\Nasty}}, \Teq, \namednode{\vinitialvalue}{\Nexpr}, \Tsemicolon
      \end{array}
  \right)}{\vparsednode}\right)
  } \astarrow \\
  {
    \overname{\left[
  \DGlobalStorage\left(\left\{
    \begin{array}{rcl}
    \GDkeyword &:& \astof{\vkeyword},\\
    \GDname &:& \astof{\name},\\
    \GDty &:& \ttyp,\\
    \GDinitialvalue &:& \astof{\vinitialvalue}
  \end{array}
  \right\}\right)
  \right]}{\vastnode}
  }
}
\end{mathpar}

\begin{mathpar}
\inferrule[global\_uninit\_var]{}{
  {
    \begin{array}{r}
      \builddecl(\overname{\Ndecl(\Tvar, \Tidentifier(\name), \Nasty, \Tsemicolon)}{\vparsednode}) \astarrow
    \end{array}
  } \\
  \overname{\left[\DGlobalStorage\left(
    \left\{
    \begin{array}{rcl}
      \GDkeyword&:& \GDKVar,\\
      \GDname&:& \name,\\
      \GDty&:& \langle\astof{\Nasty}\rangle,\\
      \GDinitialvalue&:& \None
    \end{array}
    \right\}\right)\right]}{\vastnode}
}
\end{mathpar}

\begin{mathpar}
\inferrule[global\_storage\_config]{}{
  {
    \begin{array}{r}
      \builddecl(\overname{\Ndecl(\Tconfig, \Tidentifier(\name), \Tcolon, \punnode{\Nty}, \Teq, \punnode{\Nexpr}, \Tsemicolon)}{\vparsednode}) \astarrow
    \end{array}
  } \\
  \overname{\DGlobalStorage(\{ \GDkeyword: \GDKConfig, \GDname: \name, \GDty: \Some{\astof{\vt}}, \GDinitialvalue: \Some{\astof{\vexpr}}\})}{\vastnode}
}
\end{mathpar}

\begin{mathpar}
\inferrule[global\_storage\_elided\_parameter]{
  \buildglobaldeclkeyword(\keyword) \astarrow \astof{\keyword}\\
  \buildelidedparamcall (\vcall) \astarrow \astversion{\vcall} \\
  \desugarelidedparameter(\astof{\tty}, \astversion{\vcall}) \astarrow \vexpr
}{
  {
      \builddecl\left(\overname{\Ndecl\left(
      \begin{array}{r}
      \namednode{\vkeyword}{\Nglobaldeclkeywordnonconfig}, \Tidentifier(\name), \\
      \wrappedline\ \Tcolon, \punnode{\Nty}, \Teq, \namednode{\vcall}{\Nelidedparamcall}, \Tsemicolon
      \end{array}
  \right)}{\vparsednode}\right)
  } \astarrow \\
  {
    \overname{\left[
  \DGlobalStorage\left(\left\{
    \begin{array}{rcl}
    \GDkeyword &:& \astof{\vkeyword},\\
    \GDname &:& \astof{\name},\\
    \GDty &:& \astof{\tty},\\
    \GDinitialvalue &:& \vexpr
  \end{array}
  \right\}\right)
  \right]}{\vastnode}
  }
}
\end{mathpar}

\begin{mathpar}
\inferrule[global\_storage\_config\_elided\_parameter]{
  \buildelidedparamcall (\vcall) \astarrow \astversion{\vcall} \\
  \desugarelidedparameter(\astof{\tty}, \astversion{\vcall}) \astarrow \vexpr
}{
  {
    \begin{array}{r}
      \builddecl(\overname{\Ndecl(\Tconfig, \Tidentifier(\name), \Tcolon, \punnode{\Nty}, \Teq, \namednode{\vcall}{\Nelidedparamcall}, \Tsemicolon)}{\vparsednode}) \astarrow
    \end{array}
  } \\
  \overname{\DGlobalStorage(\{ \GDkeyword: \GDKConfig, \GDname: \name, \GDty: \Some{\astof{\tty}}, \GDinitialvalue: \vexpr\})}{\vastnode}
}
\end{mathpar}

\ASTRuleDef{GlobalDeclKeywordNonConfig}
\hypertarget{build-globaldeclkeywordnonconfig}{}
The function
\[
\begin{array}{r}
\buildglobaldeclkeywordnonconfig(\overname{\parsenode{\Nglobaldeclkeywordnonconfig}}{\vparsednode}) \aslto \\
  \overname{\globaldeclkeyword}{\vastnode}
\end{array}
\]
transforms a parse node $\vparsednode$ into an AST node $\vastnode$.

\begin{mathpar}
\inferrule[let]{}{
  \buildglobaldeclkeywordnonconfig(\overname{\Nglobaldeclkeywordnonconfig(\Tlet)}{\vparsednode}) \astarrow \overname{\GDKLet}{\vastnode}
}
\end{mathpar}

\begin{mathpar}
\inferrule[constant]{}{
  {
\begin{array}{r}
  \buildglobaldeclkeywordnonconfig(\overname{\Nglobaldeclkeywordnonconfig(\Tconstant)}{\vparsednode}) \astarrow \\
  \overname{\GDKConstant}{\vastnode}
\end{array}
  }
}
\end{mathpar}

\begin{mathpar}
\inferrule[var]{}{
  {
\begin{array}{r}
  \buildglobaldeclkeywordnonconfig(\overname{\Nglobaldeclkeywordnonconfig(\Tvar)}{\vparsednode}) \astarrow \\
  \overname{\GDKVar}{\vastnode}
\end{array}
  }
}
\end{mathpar}

\ASTRuleDef{GlobalDeclKeyword}
\hypertarget{build-globaldeclkeyword}{}
The function
\[
\begin{array}{r}
\buildglobaldeclkeyword(\overname{\parsenode{\Nglobaldeclkeyword}}{\vparsednode}) \aslto \\
  \overname{\globaldeclkeyword}{\vastnode}
\end{array}
\]
transforms a parse node $\vparsednode$ into an AST node $\vastnode$.

\begin{mathpar}
\inferrule[non\_config]{}{
  \buildglobaldeclkeyword(\overname{\Nglobaldeclkeyword(\punnode{\Nglobaldeclkeywordnonconfig})}{\vparsednode}) \astarrow \overname{\astof{\vglobalkeywordnonconfig}}{\vastnode}
}
\end{mathpar}

\begin{mathpar}
\inferrule[config]{}{
  {
\begin{array}{r}
  \buildglobaldeclkeyword(\overname{\Nglobaldeclkeyword(\Tconfig)}{\vparsednode}) \astarrow \\
  \overname{\GDKConfig}{\vastnode}
\end{array}
  }
}
\end{mathpar}

%%%%%%%%%%%%%%%%%%%%%%%%%%%%%%%%%%%%%%%%%%%%%%%%%%%%%%%%%%%%%%%%%%%%%%%%%%%%%%%%
\section{Typing}
%%%%%%%%%%%%%%%%%%%%%%%%%%%%%%%%%%%%%%%%%%%%%%%%%%%%%%%%%%%%%%%%%%%%%%%%%%%%%%%%
We also define the following helper rules:
\begin{itemize}
  \item \TypingRuleRef{DeclareGlobalStorage}
  \item \TypingRuleRef{AnnotateTyOptInitialValue}
  \item \TypingRuleRef{AddGlobalStorage}
\end{itemize}

\TypingRuleDef{DeclareGlobalStorage}
\hypertarget{def-declareglobalstorage}{}
The function
\[
  \declareglobalstorage(\overname{\globalstaticenvs}{\tenv} \aslsep \overname{\globaldecl}{\gsd})
  \aslto
  (\overname{\globalstaticenvs}{\newgenv} \aslsep \overname{\globaldecl}{\newgsd})
  \cup
  \overname{\TTypeError}{\TypeErrorConfig}
\]
annotates the global storage declaration $\gsd$ in the global static environment $\genv$,
yielding a modified global static environment $\newgenv$ and annotated global storage declaration $\newgsd$.
\ProseOtherwiseTypeError

\ExampleDef{Declaring Global Storage}
\listingref{DeclareGlobalStorage-config} shows examples of well-typed global storage declarations
and ill-typed global storage declarations in comments,
focused on \verb|config| storage elements.
\ASLListing{Configuration storage elements}{DeclareGlobalStorage-config}{\typingtests/TypingRule.DeclareGlobalStorage.config.asl}

\listingref{DeclareGlobalStorage-non-config} shows examples of well-typed global storage declarations,
and ill-typed global storage declarations in comments,
focused on storage elements that are not \verb|config|.
\ASLListing{Well-typed non-config storage elements}{DeclareGlobalStorage-non-config}{\typingtests/TypingRule.DeclareGlobalStorage.non_config.asl}

The specification in \listingref{DeclareGlobalStorage-bad3} is illegal since
\texttt{config} storage elements must have an initializing expression.
\ASLListing{Uninitialized configuration storage declaration}{DeclareGlobalStorage-bad3}{\typingtests/TypingRule.DeclareGlobalStorage.bad3.asl}

The specifications in \listingref{DeclareGlobalStorage-bad1} and \listingref{DeclareGlobalStorage-bad2}
are ill-typed since
\texttt{config} storage elements have the \purity{} $\SEPure$,
which requires that all expressions used in its type annotation and initializing expression
also have \purity{} $\SEPure$,
whereas \verb|x| has the \purity{} $\SEReadonly$.
\ASLListing{Ill-typed configuration storage declaration}{DeclareGlobalStorage-bad1}{\typingtests/TypingRule.DeclareGlobalStorage.bad1.asl}
\ASLListing{Ill-typed configuration storage declaration}{DeclareGlobalStorage-bad2}{\typingtests/TypingRule.DeclareGlobalStorage.bad2.asl}

See \ExampleRef{Rejected Declaration Because of Precision Loss} for an example
of an ill-typed global storage declaration due to precision loss.

\ProseParagraph
\AllApply
\begin{itemize}
  \item $\gsd$ is a global storage declaration with keyword $\keyword$, initial value \\ $\initialvalue$,
        \optional\ type $\tyopt$, and name $\name$;
  \item checking that $\name$ is not already declared in $\genv$ yields $\True$\ProseOrTypeError;
  \item applying $\withemptylocal$ to $\genv$ yields $\tenv$;
  \item define $\vmustbepure$ as $\True$ if $\keyword$ is $\GDKConstant$ or $\GDKConfig$, and $\False$ otherwise;
  \item applying $\annotatetyoptinitialvalue$ to $\keyword$, $\vmustbepure$, $\tyoptp$, \\
        and $\initialvalue$ in $\tenv$ yields\\
        $(\typedinitialvalue, \tyoptp, \declaredt)$\ProseOrTypeError;
  \item adding a global storage element with name $\name$, global declaration keyword \\ $\keyword$ and type $\declaredt$
        to $\tenv$ via $\addglobalstorage$ yields $\tenvone$\ProseOrTypeError;
  \item applying $\withemptylocal$ to $\genvone$ yields $\tenvone$;
  \item view $\typedinitialvalue$ as $(\Ignore, \initialvaluep, \vsesinitialvalue)$;
  \item applying $\updateglobalstorage$ to $\name$, $\keyword$, $\initialvaluep$, and \\
        $\vsesinitialvalue$ in $\tenvone$ yields $\tenvtwo$\ProseOrTypeError;
  \item define $\newgsd$ as $\gsd$ with its type component ($\GDty$) set to $\tyoptp$ and its initial value component
        ($\GDinitialvalue$) set to $\initialvaluep$;
  \item define $\newgenv$ as the global component of $\tenvtwo$.
\end{itemize}

\FormallyParagraph
\begin{mathpar}
\inferrule{
  \gsd \eqname \{
    \GDkeyword : \keyword,
    \GDinitialvalue : \initialvalue,
    \GDty : \tyopt,
    \GDname: \name
  \}\\
\checkvarnotingenv{\genv, \name} \typearrow \True \OrTypeError\\\\
\withemptylocal(\genv) \typearrow \tenv\\
{
  \begin{array}{l}
    \vmustbepure \eqdef \hfill \\
    \choice{\keyword \in \{\GDKConstant, \GDKConfig\}}{\True}{\False}
  \end{array}
}\\
{
  \begin{array}{r}
    \annotatetyoptinitialvalue(\tenv, \keyword, \vmustbepure, \tyopt, \initialvalue) \\ \typearrow
    (\typedinitialvalue, \tyoptp, \declaredt) \OrTypeError
  \end{array}
}\\
\addglobalstorage(\genv, \name, \keyword, \declaredt) \typearrow \genvone \OrTypeError\\\\
\withemptylocal(\genvone) \typearrow \tenvone\\\\
\typedinitialvalue \eqname (\Ignore, \initialvaluep, \vsesinitialvalue)\\
{
    \updateglobalstorage\left(
      \begin{array}{l}
      \tenvone, \\
      \name, \\
      \keyword, \\
      \typedinitialvalue, \\
      \vsesinitialvalue
      \end{array}
      \right) \typearrow
    \tenvtwo \OrTypeError
}\\
{
\newgsd \eqdef \left\{
  \begin{array}{rcl}
  \GDkeyword &:& \keyword, \\
  \GDinitialvalue &:& \langle\typedinitialvalue\rangle, \\
  \GDty &:& \tyoptp, \\
  \GDname &:& \name
  \end{array}
\right\}
}
}{
  \declareglobalstorage(\genv, \gsd) \typearrow (\overname{G^\tenvtwo}{\newgenv}, \newgsd)
}
\end{mathpar}
\CodeSubsection{\DeclareGlobalStorageBegin}{\DeclareGlobalStorageEnd}{../Typing.ml}
\identr{YSPM} \identr{FWQM}

\TypingRuleDef{AnnotateTyOptInitialValue}
\hypertarget{def-annotatetyoptinitialvalue}{}
The helper function
\[
\begin{array}{r}
\annotatetyoptinitialvalue(
  \overname{\staticenvs}{\tenv} \aslsep
  \overname{\globaldeclkeyword}{\gdk} \aslsep
  \overname{\Bool}{\vmustbepure} \aslsep
  \overname{\langle\ty\rangle}{\tyoptp} \aslsep
  \overname{\langle\expr\rangle}{\initialvalue}
  ) \\ \aslto
  (\overname{(\expr\times \ty \times \TSideEffectSet)}{\typedinitialvalue}
  \times \overname{\langle\ty\rangle}{\tyoptp} \times \overname{\ty}{\declaredt})
  \cup \overname{\TTypeError}{\TypeErrorConfig}
\end{array}
\]
is used in the context of a declaration of a global storage element with optional type annotation $\tyoptp$
and optional initializing expression $\initialvalue$, in the static environment $\tenv$.
It determines $\typedinitialvalue$, which consists
of an expression, a type, and a \sideeffectsetterm,
the annotation of the type in $\tyoptp$ (in case there is a type), and the type
that should be associated with the storage element $\declaredt$.

See \ExampleRef{Declaring Global Storage}.

\ProseParagraph
\OneApplies
\begin{itemize}
  \item \AllApplyCase{some\_some\_config}
  \begin{itemize}
    \item $\tyoptp$ is the singleton set for the type $\vt$;
    \item $\initialvalue$ is the singleton set for the expression $\ve$;
    \item $\gdk$ is $\GDKConfig$;
    \item annotating the expression $\ve$ in $\tenv$ yields $(\vte, \vep, \vsese)$\ProseOrTypeError;
    \item annotating the type $\vt$ in $\tenv$ yields $(\vtp, \vsest)$\ProseOrTypeError;
    \item define $\typede$ as $(\vte, \vep, \vsese)$;
    \item checking that $\vte$ \typesatisfies\ $\vtp$ in $\tenv$ yields $\True$\ProseOrTypeError;
    \item checking that if $\vmustbepure$ is $\True$ then $\vsest$ and $\vsese$ are \pure{} yields $\True$\ProseOrTypeError;
    \item define $\typedinitialvalue$ as $\typede$;
    \item define $\tyoptp$ as the singleton set for $\vtp$;
    \item define $\declaredt$ as $\vtp$;
  \end{itemize}

  \item \AllApplyCase{some\_some}
  \begin{itemize}
    \item $\tyoptp$ is the singleton set for the type $\vt$;
    \item $\initialvalue$ is the singleton set for the expression $\ve$;
    \item $\gdk$ is not $\GDKConfig$;
    \item annotating the expression $\ve$ in $\tenv$ yields $(\vte, \vep, \vsese)$\ProseOrTypeError;
    \item determining the \structure{} of $\vte$ in $\tenv$ yields $\vtep$\ProseOrTypeError;
    \item propagating integer constraints from $\vtep$ to $\vt$ using $\inheritintegerconstraints$ yields $\vtpp$\ProseOrTypeError;
    \item annotating the type $\vtpp$ in $\tenv$ yields $(\vtp, \vsest)$\ProseOrTypeError;
    \item define $\typede$ as $(\vte, \vep, \vsese)$;
    \item checking that $\vte$ \typesatisfies\ $\vtp$ in $\tenv$ yields $\True$\ProseOrTypeError;
    \item checking that if $\vmustbepure$ is $\True$ then $\vsest$ and $\vsese$ are \pure{} yields $\True$\ProseOrTypeError;
    \item define $\typedinitialvalue$ as $\typede$;
    \item define $\tyoptp$ as the singleton set for $\vtp$;
    \item define $\declaredt$ as $\vtp$;
  \end{itemize}

  \item \AllApplyCase{some\_none}
  \begin{itemize}
    \item $\tyoptp$ is the singleton set for the type $\vt$;
    \item $\initialvalue$ is $\None$;
    \item annotating the type $\vt$ in $\tenv$ yields $(\vtp, \vsest)$\ProseOrTypeError;
    \item checking that if $\vmustbepure$ is $\True$ then $\vsest$ is \pure{} yields $\True$\ProseOrTypeError;
    \item obtaining the \basevalueterm\ of $\vtp$ in $\tenv$ yields $\vep$\ProseOrTypeError;
    \item define $\typedinitialvalue$ as $(\vtp, \vep, \emptyset)$;
    \item define $\tyoptp$ as the singleton set for $\vtp$;
    \item define $\declaredt$ as $\vtp$;
  \end{itemize}

  \item \AllApplyCase{none\_some}
  \begin{itemize}
    \item $\tyoptp$ is $\None$;
    \item $\initialvalue$ is the singleton set for the expression $\ve$;
    \item annotating the expression $\ve$ in $\tenv$ yields $(\vte, \vep, \vsese)$\ProseOrTypeError;
    \item \Prosenoprecisionloss{$\vte$};
    \item checking that if $\vmustbepure$ is $\True$ then $\vsese$ is \pure{} yields $\True$\ProseOrTypeError;
    \item define $\typede$ as $(\vte, \vep, \vsese)$;
    \item define $\typedinitialvalue$ as $\typede$;
    \item define $\tyoptp$ as $\None$;
    \item define $\declaredt$ as $\vte$;
  \end{itemize}
\end{itemize}
The case where both $\tyopt$ and $\initialvalue$ are $\None$ is considered a syntax error.

\FormallyParagraph
\begin{mathpar}
\inferrule[some\_some\_config]{
  \annotateexpr{\tenv, \ve} \typearrow (\vte, \vep, \vsese) \OrTypeError\\\\
  \annotatetype{\tenv, \vt} \typearrow (\vtp, \vsest) \OrTypeError\\\\
  \typede \eqdef (\vte, \vep, \vsese)\\
  \checktypesat(\tenv, \vte, \vtp) \typearrow \True \OrTypeError\\\\
  \checktrans{\lnot\vmustbepure \lor \sesispure(\vsest \cup \vsese)}{\SideEffectViolation} \typearrow \True \OrTypeError
}{
  {
    \begin{array}{r}
  \annotatetyoptinitialvalue(\tenv, \overname{\GDKConfig}{\gdk}, \vmustbepure, \overname{\langle\vt\rangle}{\tyoptp}, \overname{\langle\ve\rangle}{\initialvalue})
  \\ \typearrow
  (\overname{\typede}{\typedinitialvalue}, \overname{\langle\vtp\rangle}{\tyoptp}, \overname{\vtp}{\declaredt})
    \end{array}
  }
}
\end{mathpar}

\begin{mathpar}
\inferrule[some\_some]{
  \gdk \neq \GDKConfig \\
  \annotateexpr{\tenv, \ve} \typearrow (\vte, \vep, \vsese) \OrTypeError\\\\
  \tstruct(\tenv, \vte) \typearrow \vtep \OrTypeError \\\\
  \inheritintegerconstraints(\vt, \vtep) \typearrow \vtpp \OrTypeError \\\\
  \annotatetype{\tenv, \vtpp} \typearrow (\vtp, \vsest) \OrTypeError\\\\
  \typede \eqdef (\vte, \vep, \vsese)\\
  \checktypesat(\tenv, \vte, \vtp) \typearrow \True \OrTypeError\\\\
  \checktrans{\lnot\vmustbepure \lor \sesispure(\vsest \cup \vsese)}{\SideEffectViolation} \typearrow \True \OrTypeError
}{
  {
    \begin{array}{r}
  \annotatetyoptinitialvalue(\tenv, \gdk, \vmustbepure, \overname{\langle\vt\rangle}{\tyoptp}, \overname{\langle\ve\rangle}{\initialvalue})
  \typearrow \\
  (\overname{\typede}{\typedinitialvalue}, \overname{\langle\vtp\rangle}{\tyoptp}, \overname{\vtp}{\declaredt})
    \end{array}
  }
}
\end{mathpar}

\begin{mathpar}
\inferrule[some\_none]{
  \annotatetype{\tenv, \vt} \typearrow (\vtp, \vsest) \OrTypeError\\\\
  \checktrans{\lnot\vmustbepure \lor \sesispure(\vsest)}{\SideEffectViolation} \typearrow \True \OrTypeError\\\\
  \basevalue(\tenv, \vtp) \typearrow \vep \OrTypeError\\\\
  \typedinitialvalue \eqdef (\vtp, \vep, \emptyset)
}{
  {
  \begin{array}{r}
    \annotatetyoptinitialvalue(\tenv, \gdk, \vmustbepure, \overname{\langle\vt\rangle}{\tyoptp}, \overname{\None}{\initialvalue})
    \typearrow \\
    (\typedinitialvalue, \overname{\langle\vtp\rangle}{\tyoptp}, \overname{\vtp}{\declaredt})
  \end{array}
  }
}
\end{mathpar}

\begin{mathpar}
\inferrule[none\_some]{
  \annotateexpr{\tenv, \ve} \typearrow (\vte, \vep, \vsese) \OrTypeError\\\\
  \checknoprecisionloss{\vte} \typearrow \True \OrTypeError\\\\
  \typede \eqdef (\vte, \vep, \vsese) \\
  \checktrans{\lnot\vmustbepure \lor \sesispure(\vsese)}{\SideEffectViolation} \typearrow \True \OrTypeError
}{
  {
    \begin{array}{r}
  \annotatetyoptinitialvalue(\tenv, \gdk, \vmustbepure, \overname{\None}{\tyoptp}, \overname{\langle\ve\rangle}{\initialvalue})
  \typearrow \\
  (\overname{\typede}{\typedinitialvalue}, \overname{\None}{\tyoptp}, \overname{\vte}{\declaredt})
    \end{array}
  }
}
\end{mathpar}

\TypingRuleDef{UpdateGlobalStorage}
\hypertarget{def-updateglobalstorage}{}
The helper function
\[
\updateglobalstorage
\left(
\begin{array}{c}
  \overname{\staticenvs}{\tenv} \aslsep\\
    \overname{\identifier}{\name} \aslsep\\
    \overname{\globaldeclkeyword}{\gdk} \aslsep\\
    \overname{(\ty\times\expr\times\TSideEffectSet)}{\typedinitialvalue}
\end{array}
\right) \aslto \overname{\staticenvs}{\newtenv}
\]
updates the static environment $\tenv$ for the global storage element
named $\name$ with global declaration keyword $\gdk$,
and a tuple (obtained via \\
\TypingRuleRef{AnnotateTyOptInitialValue})
$\typedinitialvalue$, which consists a type for the initializing value,
the annotated initializing expression, and the inferred \sideeffectsetterm\ for the initializing value.
The result is the updated static environment $\newtenv$.
\ProseOtherwiseTypeError
This helper function is applied following $\addglobalstorage(\tenv, \name, \gdk, \vt)$ where $\vt$
is the type associated with $\name$.

\ExampleDef{Updating the Global Storage for a Constant}
The specification in \listingref{UpdateGlobalStorage-constant}
binds $\vy$ to $\lint(5040)$ in the $\constantvalues$ map of the global static environment.
\ASLListing{Updating the global storage for a constant}{UpdateGlobalStorage-constant}{\typingtests/TypingRule.UpdateGlobalStorage.constant.asl}

\ExampleDef{Updating the Global Storage for Configuration Storage Elements}
The specification in \listingref{UpdateGlobalStorage-config}
shows well-typed global storage \texttt{config} elements.
\ASLListing{Updating the global storage for configuration storage elements}{UpdateGlobalStorage-config}{\typingtests/TypingRule.UpdateGlobalStorage.config.asl}

The specification in \listingref{UpdateGlobalStorage-config-bad}
is ill-typed, since only a \Prosesingulartype{} can be used for \texttt{config} storage elements.
\ASLListing{An ill-typed configuration storage element}{UpdateGlobalStorage-config-bad}{\typingtests/TypingRule.UpdateGlobalStorage.config.bad.asl}

\ProseParagraph
\OneApplies
\begin{itemize}
  \item \AllApplyCase{constant}
  \begin{itemize}
    \item $\gdk$ is $\GDKConstant$;
    \item view $\typedinitialvalue$ as $(\Ignore, \initialvaluep, \vsesinitialvalue)$;
    \item \Prosestaticeval{$\tenv$}{$\initialvaluep$}{$\vv$};
    \item applying $\addglobalconstant$ to $G^\tenv$, $\name$ and $\vv$ yields $G'$;
    \item \Proseeqdef{$\newtenv$}{the static environment whose global component is $G'$ and local component is
    the local component of $\tenv$}.
  \end{itemize}

  \item \AllApplyCase{let\_normalizable}
  \begin{itemize}
    \item $\gdk$ is $\GDKLet$;
    \item applying $\normalizeopt$ to $\ve$ in $\tenv$ yields $\langle\vep\rangle$\ProseOrTypeError;
    \item applying $\addglobalimmutableexpr$ to $\name$ and $\vep$ in $\tenv$ yields $\newtenv$.
  \end{itemize}

  \item \AllApplyCase{let\_non\_normalizable}
  \begin{itemize}
    \item $\gdk$ is $\GDKLet$;
    \item applying $\normalizeopt$ to $\ve$ in $\tenv$ yields $\None$\ProseOrTypeError;
    \item \Proseeqdef{$\newtenv$}{$\tenv$}.
  \end{itemize}

  \item \AllApplyCase{config}
  \begin{itemize}
    \item $\gdk$ is $\GDKConfig$;
    \item view $\typedinitialvalue$ as \\ $(\initialvaluetype, \initialvaluep, \vsesinitialvalue)$;
    \item checking that $\initialvaluetype$ is a singular type yields $\True$\ProseOrTypeError;
    \item \Proseeqdef{$\newtenv$}{$\tenv$}.
  \end{itemize}

  \item \AllApplyCase{var}
  \begin{itemize}
    \item $\gdk$ is $\GDKVar$;
    \item \Proseeqdef{$\newtenv$}{$\tenv$}.
  \end{itemize}
\end{itemize}

\FormallyParagraph
\begin{mathpar}
\inferrule[constant]{
  \typedinitialvalue \eqname (\Ignore, \initialvaluep, \vsesinitialvalue)\\
  \staticeval(\tenv, \initialvaluep) \typearrow \vv\\
  \addglobalconstant(G^\tenv, \name, \vv) \typearrow G'\\
  \newtenv \eqdef (G', L^\tenv)
}{
  {
    \begin{array}{r}
  \updateglobalstorage(
    \tenv,
    \name,
    \overname{\GDKConstant}{\gdk},
    \typedinitialvalue) \typearrow \\
    \newtenv
    \end{array}
  }
}
\end{mathpar}

\begin{mathpar}
\inferrule[let\_normalizable]{
  \normalizeopt(\tenv, \ve) \typearrow \langle\vep\rangle \OrTypeError\\\\
  \addglobalimmutableexpr(\tenv, \name, \vep) \typearrow \newtenv
}{
  {
    \begin{array}{r}
  \updateglobalstorage(
    \tenv,
    \name,
    \overname{\GDKLet}{\gdk},
    \typedinitialvalue) \typearrow \\
    \newtenv
    \end{array}
  }
}
\end{mathpar}

\begin{mathpar}
\inferrule[let\_non\_normalizable]{
  \normalizeopt(\tenv, \ve) \typearrow \None
}{
  {
    \begin{array}{r}
  \updateglobalstorage(
    \tenv,
    \name,
    \overname{\GDKLet}{\gdk},
    \typedinitialvalue) \typearrow \\
    \overname{\tenv}{\newtenv}
    \end{array}
  }
}
\end{mathpar}

\begin{mathpar}
\inferrule[config]{
  \typedinitialvalue \eqname (\initialvaluetype, \initialvaluep, \vsesinitialvalue)\\
  \issingular(\tenv, \initialvaluetype) \typearrow \True \OrTypeError
}{
  {
    \begin{array}{r}
  \updateglobalstorage(
    \tenv,
    \name,
    \overname{\GDKConfig}{\gdk},
    \typedinitialvalue) \typearrow \\
    \overname{\tenv}{\newtenv}
    \end{array}
  }
}
\end{mathpar}

\begin{mathpar}
\inferrule[var]{}{
  {
    \begin{array}{r}
  \updateglobalstorage(
    \tenv,
    \name,
    \overname{\GDKVar}{\gdk},
    \typedinitialvalue) \typearrow \\
    \overname{\tenv}{\newtenv}
    \end{array}
  }
}
\end{mathpar}

\TypingRuleDef{AddGlobalStorage}
\hypertarget{def-addglobalstorage}{}
The function
\[
  \addglobalstorage(
    \overname{\globalstaticenvs}{\genv} \aslsep
    \overname{\identifier}{\name} \aslsep
    \overname{\globaldeclkeyword}{\keyword} \aslsep
    \overname{\ty}{\declaredt}
  )
  \aslto
    \overname{\globalstaticenvs}{\newgenv} \cup \overname{\TTypeError}{\TypeErrorConfig}
\]
returns a global static environment $\newgenv$ which is identical to the global static environment $\genv$,
except that the identifier $\name$, which is assumed to name a global storage element,
is bound to the global storage keyword $\keyword$ and type $\declaredt$.
\ProseOtherwiseTypeError

\ExampleDef{Adding Global Storage Elements to the Static Environment}
Adding the global storage elements in \listingref{EvalGlobals},
adds the following bindings to the $\globalstoragetypes$ map of a given
static environments:
\[
\begin{array}{rcl}
  \PI &\mapsto& (\TReal, \GDKConstant)\\
  \vx &\mapsto& (\unconstrainedinteger, \GDKVar)\\
  \vc &\mapsto& (\TInt(\wellconstrained(\AbbrevConstraintExact{\lint(42)})), \GDKConfig)
\end{array}
\]

\ProseParagraph
\AllApply
\begin{itemize}
  \item checking that $\name$ is not declared in the global environment of $\tenv$ yields $\True$\ProseOrTypeError;
  \item $\newgenv$ is the global static environment of $\tenv$ with its $\globalstoragetypes$ component updated by binding $\name$ to
        $(\declaredt, \keyword)$.
\end{itemize}
\FormallyParagraph
\begin{mathpar}
\inferrule{
  \checkvarnotingenv{\genv, \name} \typearrow \True \OrTypeError\\\\
  \newgenv \eqdef \genv.\globalstoragetypes[\name \mapsto (\declaredt, \keyword)]
}{
  \addglobalstorage(\genv, \name, \keyword, \declaredt) \typearrow \newgenv
}
\end{mathpar}
\CodeSubsection{\AddGlobalStorageBegin}{\AddGlobalStorageEnd}{../Interpreter.ml}

%%%%%%%%%%%%%%%%%%%%%%%%%%%%%%%%%%%%%%%%%%%%%%%%%%%%%%%%%%%%%%%%%%%%%%%%%%%%%%%%
\section{Semantics\label{sec:GlobalStorageDeclarationsSemantics}}
%%%%%%%%%%%%%%%%%%%%%%%%%%%%%%%%%%%%%%%%%%%%%%%%%%%%%%%%%%%%%%%%%%%%%%%%%%%%%%%%

This section defines the following relations:
\begin{itemize}
  \item \SemanticsRuleRef{EvalGlobals}
  \item \SemanticsRuleRef{DeclareGlobal}
\end{itemize}

\SemanticsRuleDef{EvalGlobals}
The relation
\hypertarget{def-evalglobals}{}
\[
  \evalglobals(\overname{\decl^*}{\vdecls}, (\overname{\overname{\envs}{\env} \times \overname{\XGraphs}{\vgone}}{\envm}))
  \;\aslrel\; \overname{(\envs \times \XGraphs)}{C} \cup
  \overname{\TThrowing}{\ThrowingConfig} \cup \overname{\TDynError}{\DynErrorConfig}
\]
updates the input environment and execution graph by initializing the global storage declarations.
Otherwise, the evaluation terminates abnormally.

\ExampleDef{Evaluating a List of Global Declarations}
Evaluating the global storage declarations in \listingref{EvalGlobals}
results in updating the $\storage$ map of global dynamic environment by binding
\verb|PI| to $\nvreal(157/50)$,
\verb|PC| to \\
$\nvbitvector(\overbrace{0 \ldots 0}^{32})$,
\verb|MaxIrq| to $\nvint(480)$, and
\verb|Regs| to
\[
\nvvector{ [\overbrace{\nvbitvector(\overbrace{0 \ldots 0}^{32}) \ldots \nvbitvector(\overbrace{0 \ldots 0}^{32})}^{16}] } \enspace.
\]

\ASLListing{Evaluating a list of global declarations}{EvalGlobals}{\semanticstests/SemanticsRule.EvalGlobals.asl}

\ExampleDef{A Storage Declaration Resulting in a Thrown Exception}
The specification in \listingref{EvalGlobals-bad1} evaluates the
global storage declaration for \verb|x|, which results in a thrown exception,
before having the chance to evaluate \verb|main|.
\ASLListing{Evaluating a global storage declaration resulting in a thrown exception}{EvalGlobals-bad1}{\semanticstests/SemanticsRule.EvalGlobals.bad1.asl}

\ProseParagraph
\OneApplies
\begin{itemize}
  \item \AllApplyCase{empty}
  \begin{itemize}
    \item there are no declarations of global variables;
    \item the result is $\envm$.
  \end{itemize}

  \item \AllApplyCase{non\_empty}
  \begin{itemize}
    \item $\vdecls$ has $\vd$ as its head and $\vdecls'$ as its tail;
    \item $d$ is the AST node for declaring a global storage element with initial value $\ve$,
          name $\name$, and type $\vt$;
    \item $\envm$ is the environment-execution graph pair $(\env, \vgone)$;
    \item the evaluation of the expression $\ve$ in $\env$ yields $\Normal((\vv, \vgtwo), \envtwo)$\ProseOrAbnormal;
    \item declaring the global $\name$ with value $\vv$ in $\envtwo$ gives $\envthree$;
    \item evaluating the remaining global declarations $\vdecls'$ with the environment $\envthree$ and the execution graph
          that is the ordered composition of $\vgone$ and $\vgtwo$ with the $\aslpo$ label gives $C$;
    \item the result of the entire evaluation is $C$.
  \end{itemize}
\end{itemize}

\FormallyParagraph
\begin{mathpar}
\inferrule[empty]{}{
  \evalglobals(\overname{\emptylist}{\vdecls}, \envm) \evalarrow \overname{\envm}{C}
}
\end{mathpar}

\begin{mathpar}
\inferrule[non\_empty]{
  \vd \eqname \DGlobalStorage(\{ \GDinitialvalue:\langle\ve\rangle, \GDname:\name, \ldots \})\\
  \envm \eqname (\env, \vgone)\\
  \evalexpr{ \env, \ve} \evalarrow \Normal((\vv,\vgtwo), \envtwo) \OrAbnormal\\\\
  \declareglobal(\name, \vv, \envtwo) \evalarrow \envthree\\
  \evalglobals(\vdecls', (\envthree, \ordered{\vgone}{\aslpo}{ \vgtwo })) \evalarrow C
}{
  \evalglobals(\overname{[\vd] \concat \vdecls'}{\vdecls}, \envm) \evalarrow C
}
\end{mathpar}
\CodeSubsection{\EvalGlobalsBegin}{\EvalGlobalsEnd}{../Interpreter.ml}

\SemanticsRuleDef{DeclareGlobal}
\ProseParagraph
The relation
\hypertarget{def-declareglobal}{}
\[
  \declareglobal(\overname{\Identifiers}{\name} \aslsep \overname{\vals}{\vv} \aslsep \overname{\envs}{\env}) \;\aslrel\; \overname{\envs}{\newenv}
\]
updates the environment $\env$ by mapping $\name$ to $\vv$ in the $\storage$ map of the global dynamic environment $G^\denv$.

\ExampleDef{Declaring a Global Storage Element}
Evaluating the specification in \listingref{DeclareGlobal}, results in updating the $\storage$ map
of the global dynamic environment by binding $\vx$ to $\nvint(5)$.
\ASLListing{Declaring a global storage element}{DeclareGlobal}{\semanticstests/SemanticsRule.DeclareGlobal.asl}

\FormallyParagraph
\begin{mathpar}
\inferrule{
  \env \eqname (\tenv, (G^\denv, L^\denv))\\
  \newenv \eqdef (\tenv, (G^\denv.\storage[\name\mapsto \vv], L^\denv))
}{
  \declareglobal(\name, \vv, \env) \evalarrow \newenv
}
\end{mathpar}
