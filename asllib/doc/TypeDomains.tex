%%%%%%%%%%%%%%%%%%%%%%%%%%%%%%%%%%%%%%%%%%%%%%%%%%%%%%%%%%%%%%%%%%%%%%%%%%%%%%%%%%%%
\section{Domain of Values for Types\label{sec:DomainOfValuesForTypes}}
%%%%%%%%%%%%%%%%%%%%%%%%%%%%%%%%%%%%%%%%%%%%%%%%%%%%%%%%%%%%%%%%%%%%%%%%%%%%%%%%%%%%
This section formalizes the concept of the set of values for a given type.
The formalism is given in the form of inference rules, although those are not meant
to be implemented.

We define the concept of a \emph{dynamic domain} of a type
and the \emph{static domain} of a type.
Intuitively, domains assign potentially infinite sets of \nativevalues\ to types.
Dynamic domains are used by the semantics to evaluate expressions of the form \texttt{ARBITRARY: t}
by choosing a single value from the dynamic domain of $\vt$ (\SemanticsRuleRef{EArbitrary}).
Static domains are used to define subtype satisfaction (\TypingRuleRef{SubtypeSatisfaction})
via a conservative subsumption test as defined in \chapref{SymbolicDomainSubsetTesting}.

\subsection{Dynamic Domain of a Type\label{sec:DynDomain}}
\hypertarget{def-dyndomain}{}
The dynamic domain is defined via the partial function
\[
  \dynamicdomain : \overname{\envs}{\env} \times \overname{\ty}{\vt}
  \partialto \overname{\pow{\vals}}{\vd}
\]
which assigns the set of values that a type $\vt$ can hold in a given environment $\env$.
%
We say that $\dynamicdomain(\env, \vt)$ is the \emph{dynamic domain} of $\vt$
in the environment $\env$.
%
The \emph{static domain} of a type is the set of values which storage elements of that type may hold
\underline{across all possible dynamic environments}.
%
The reason for this distinction is that the sets of values
of integer types, bitvector types, and array types can depend on the dynamic values of variables.

Types that do not refer to variables whose values are only known dynamically have
a static domain that is equal to any of their dynamic domains.
In those cases, we simply refer to their \emph{domain}.

Associating a set of values to a type is done by evaluating any expression appearing
in the type definitions.
%
Expressions appearing in types are guaranteed to be side-effect-free by the
function $\annotatetype$.
%
Evaluation is defined by the relation $\evalexprsef\empty$.
which evaluates side-effect-free expressions and either returns
a configuration of the form $\ResultExprSEF(\vv,\vg)$ or a dynamic error configuration $\DynErrorConfig$.
In the first case, $\vv$ is a \nativevalue\ and $\vg$
is an \emph{execution graph}. Execution graphs are related to the concurrent semantics
and can be ignored in the context of defining dynamic domains.
In the latter case (which can occur if, for example, an expression attempts to divide
\texttt{8} by \texttt{0}), a \DynamicErrorConfigurationTerm{}, for which we use the notation
$\DynErrorConfig$, is returned.
%
The dynamic domain is empty in cases where evaluating side-effect-free expressions
results in a \DynamicErrorConfigurationTerm{}.
%
The dynamic domain is undefined if the type $\vt$ is not well-typed in $\tenv$.
That is, if $\annotatetype(\tenv, \vt) \typearrow \TypeErrorConfig$.

As part of the definition, we also associate dynamic domains to integer constraints
by overloading $\dynamicdomain$:
\[
  \dynamicdomain : \overname{\envs}{\env} \times \overname{\intconstraint}{\vc}
  \partialto \overname{\pow{\vals}}{\vd}
\]

\ProseParagraph
\OneApplies
\begin{itemize}
  \item \AllApplyCase{t\_bool}
  \begin{itemize}
    \item $\vt$ is the Boolean type, $\TBool$;
    \item $\vd$ is the set of native Boolean values, $\tbool$.
  \end{itemize}

  \item \AllApplyCase{t\_string}
  \begin{itemize}
    \item $\vt$ is the \stringtypeterm{}, $\TString$;
    \item $\vd$ is the set of all native string values, $\tstring$.
  \end{itemize}

  \item \AllApplyCase{t\_real}
  \begin{itemize}
    \item $\vt$ is the \realtypeterm{}, $\TReal$;
    \item $\vd$ is the set of all native real values, $\treal$.
  \end{itemize}

  \item \AllApplyCase{t\_enumeration}
  \begin{itemize}
    \item $\vt$ is the \enumerationtypeterm{} with labels $\vl_{1..k}$, that is $\TEnum(\vl_{1..k})$;
    \item $\vd$ is the set of all native labels $\nvlabel(\vl_\vi)$, for $\vi = 1..k$.
  \end{itemize}

  \item \AllApplyCase{t\_int\_unconstrained}
  \begin{itemize}
    \item $\vt$ is the unconstrained integer type, $\unconstrainedinteger$;
    \item $\vd$ is the set of all native integer values, $\tint$.
  \end{itemize}

  \item \AllApplyCase{t\_int\_well\_constrained}
  \begin{itemize}
    \item $\vt$ is the well-constrained integer type $\TInt(\wellconstrained(\vc_{1..k}))$;
    \item $\vd$ is the union of the dynamic domains of each of the constraints $\vc_{1..k}$ in $\env$.
  \end{itemize}

  \item \AllApplyCase{constraint\_exact\_okay}
  \begin{itemize}
    \item $\vc$ is a constraint consisting of a single side-effect-free expression $\ve$, that is, $\ConstraintExact(\ve)$;
    \item evaluating $\ve$ in $\env$ results in a configuration with the native integer for $n$;
    \item $\vd$ is the set containing the single native integer value for $n$.
  \end{itemize}

  \item \AllApplyCase{constraint\_exact\_abnormal}
  \begin{itemize}
    \item $\vc$ is a constraint consisting of a single side-effect-free expression $\ve$, that is, $\ConstraintExact(\ve)$;
    \item evaluating $\ve$ in $\env$ results in either a dynamic error configuration or diverges;
    \item $\vd$ is the empty set.
  \end{itemize}

  \item \AllApplyCase{constraint\_range\_okay}
  \begin{itemize}
    \item $\vc$ is a range constraint consisting of a two side-effect-free expressions $\veone$ and $\vetwo$, that is, $\ConstraintRange(\veone, \vetwo)$;
    \item evaluating $\veone$ in $\env$ results in a configuration with the native integer for $a$;
    \item evaluating $\vetwo$ in $\env$ results in a configuration with the native integer for $b$;
    \item $\vd$ is the set containing all native integer values for integers greater or equal to $a$ and less than or equal to $b$.
  \end{itemize}

  \item \AllApplyCase{constraint\_range\_abnormal1}
  \begin{itemize}
    \item $\vc$ is a range constraint consisting of a two side-effect-free expressions $\veone$ and $\vetwo$, that is, $\ConstraintRange(\veone, \vetwo)$;
    \item evaluating $\veone$ in $\env$ results in either a dynamic error configuration or diverges;
    \item $\vd$ is the empty set.
  \end{itemize}

  \item \AllApplyCase{constraint\_range\_abnormal2}
  \begin{itemize}
    \item $\vc$ is a range constraint consisting of a two side-effect-free expressions $\veone$ and $\vetwo$, that is, $\ConstraintRange(\veone, \vetwo)$;
    \item evaluating $\veone$ in $\env$ results in a configuration with the native integer for $a$;
    \item evaluating $\vetwo$ in $\env$ results in either a dynamic error configuration or diverges;
    \item $\vd$ is the empty set.
  \end{itemize}

  \item \AllApplyCase{t\_int\_parameterized}
  \begin{itemize}
    \item $\vt$ is a \parameterizedintegertype\ for parameter $\id$, \\ $\TInt(\parameterized(\id))$;
    \item the \nativevalue\ associated with $\id$ in the local dynamic environment is the native integer value for $n$;
    \item $\vd$ is the set containing the single integer value for $n$.
  \end{itemize}

  \item \AllApplyCase{t\_bits\_dynamic\_error}
  \begin{itemize}
    \item $\vt$ is a bitvector type with size expression $\ve$, $\TBits(\ve, \Ignore)$;
    \item evaluating $\ve$ in $\env$ results in either a dynamic error configuration or diverges;
    \item $\vd$ is the empty set.
  \end{itemize}

  \item \AllApplyCase{t\_bits\_negative\_width\_error}
  \begin{itemize}
    \item $\vt$ is a bitvector type with size expression $\ve$, $\TBits(\ve, \Ignore)$;
    \item evaluating $\ve$ in $\env$ results in a configuration with the native integer for $k$;
    \item $k$ is negative;
    \item $\vd$ is the empty set.
  \end{itemize}

  \item \AllApplyCase{t\_bits\_empty}
  \begin{itemize}
    \item $\vt$ is a bitvector type with size expression $\ve$, $\TBits(\ve, \Ignore)$;
    \item evaluating $\ve$ in $\env$ results in a configuration with the native integer for $0$;
    \item $\vd$ is the set containing the single \nativevalue\ for an empty bitvector.
  \end{itemize}

  \item \AllApplyCase{t\_bits\_non\_empty}
  \begin{itemize}
    \item $\vt$ is a bitvector type with size expression $\ve$, $\TBits(\ve, \Ignore)$;
    \item evaluating $\ve$ in $\env$ results in a configuration with the native integer for $k$;
    \item $k$ is greater than $0$;
    \item $\vd$ is the set containing all \nativevalues\ for bitvectors of size exactly $k$.
  \end{itemize}

  \item \AllApplyCase{t\_tuple}
  \begin{itemize}
    \item $\vt$ is a \tupletypeterm{} over types $\vt_i$, for $i=1..k$, $\TTuple(\vt_{1..k})$;
    \item the domain of each element $\vt_i$ is $D_i$, for $i=1..k$;
    \item evaluating $\ve$ in $\env$ results in a configuration with the native integer for $k$;
    \item $\vd$ is the set containing all native vectors of $k$ values, where the value at position $i$
    is from $D_i$.
  \end{itemize}

  \item \AllApply
  \begin{itemize}
    \item $\vt$ is an integer-indexed array type with length expression $\ve$ and element type $\telem$, $\TArray(\ArrayLengthExpr(\ve), \telem)$;
    \item \OneApplies
      \begin{itemize}
      \item \AllApplyCase{t\_array\_dynamic\_error}
      \begin{itemize}
        \item evaluating $\ve$ in $\env$ results in either a dynamic error configuration or diverges;
        \item $\vd$ is the empty set.
      \end{itemize}

      \item \AllApplyCase{t\_array\_negative\_length\_error}
      \begin{itemize}
        \item evaluating $\ve$ in $\env$ results in a configuration with the native integer for $k$;
        \item $k$ is negative;
        \item $\vd$ is the empty set.
      \end{itemize}

      \item \AllApplyCase{t\_array\_okay}
      \begin{itemize}
        \item evaluating $\ve$ in $\env$ results in a configuration with the native integer for $k$;
        \item $k$ is greater than or equal to $0$;
        \item the domain of $\vtone$ is $D_\telem$;
        \item $\vd$ is the set of all native vectors of $k$ values taken from $D_\telem$.
      \end{itemize}
    \end{itemize}
  \end{itemize}

  \item \AllApplyCase{t\_enum\_array}
  \begin{itemize}
    \item $\vt$ is an enumeration-indexed array type with for the enumeration $\id$ with $k$ labels and element type $\telem$,
          $\TArray(\ArrayLengthEnum(\id, k), \telem)$;
    \item view $\env$ as the pair consisting of the \staticenvironmentterm{} $\tenv$ and a dynamic environment;
    \item the type bound to $\id$ in the $\declaredtypes$ map of the \staticenvironmentterm{} of $\tenv$ is the \enumerationtypeterm{}
          for the labels $\vl_{1..k}$, that is, $\TEnum(\vl_{1..k})$;
    \item the dynamic domain of $\telem$ in $\env$ is $D_\telem$;
    \item $\vd$ is the set of all native records where each $\vl_i$ is mapped to a value taken from $D_\telem$, for $i=1..k$.
  \end{itemize}

  \item \AllApplyCase{t\_structured}
  \begin{itemize}
    \item $\vt$ is a \structuredtype\ with typed fields $(\id_i, \vt_i)$, for $i=1..k$, that is $L([i=1..k: (\id_i,\vt_i))]$
    where $L\in\{\TRecord, \TException\}$;
    \item the domain of each type $\vt_i$ is $D_i$, for $i=1..k$;
    \item $\vd$ is the set containing all native records where $\id_i$ is mapped to a value taken from $D_i$, for $i=1..k$.
  \end{itemize}

  \item \AllApplyCase{t\_named}
  \begin{itemize}
    \item $\vt$ is a named type with name $\id$, $\TNamed(\id)$;
    \item the type associated with $\id$ in $\tenv$ is $\tty$;
    \item $\vd$ is the domain of $\tty$ in $\env$.
  \end{itemize}
\end{itemize}

\FormallyParagraph

\begin{mathpar}
\inferrule[t\_bool]{}{ \dynamicdomain(\env, \overname{\TBool}{\vt}) = \overname{\tbool}{\vd} }
\and
\inferrule[t\_string]{}{ \dynamicdomain(\env, \overname{\TString}{\vt}) = \overname{\tstring}{\vd} }
\and
\inferrule[t\_real]{}{ \dynamicdomain(\env, \overname{\TReal}{\vt}) = \overname{\treal}{\vd} }
\and
\inferrule[t\_enumeration]{}{
  \dynamicdomain(\env, \overname{\TEnum(\vl_{1..k})}{\vt}) = \overname{\{ \vi = 1..k: \nvlabel(\vl_\vi) \}}{\vd}
}
\end{mathpar}

\begin{mathpar}
  \inferrule[t\_int\_unconstrained]{}{
  \dynamicdomain(\env, \overname{\unconstrainedinteger}{\vt}) = \overname{\tint}{\vd}
}
\end{mathpar}

\begin{mathpar}
\inferrule[t\_int\_well\_constrained]{}{
  \dynamicdomain(\env, \overname{\TInt(\wellconstrained(\vc_{1..k}))}{\vt}) = \overname{\bigcup_{i=1}^k \dynamicdomain(\env, \vc_i)}{\vd}
}
\end{mathpar}

\begin{mathpar}
\inferrule[constraint\_exact\_okay]{
  \evalexprsef{\env, \ve} \evalarrow \ResultExprSEF(\nvint(n), \Ignore)
}{
  \dynamicdomain(\env, \overname{\ConstraintExact(\ve)}{\vc}) = \overname{\{ \nvint(n) \}}{\vd}
}
\and
\inferrule[constraint\_exact\_abnormal]{
  \evalexprsef{\env, \ve} \evalarrow C\\
  C \in \TDynError \cup \TDiverging
}{
  \dynamicdomain(\env, \overname{\ConstraintExact(\ve)}{\vc}) = \overname{\emptyset}{\vd}
}
\end{mathpar}

\begin{mathpar}
\inferrule[constraint\_range\_okay]{
  \evalexprsef{\env, \veone} \evalarrow \ResultExprSEF(\nvint(a), \Ignore)\\
  \evalexprsef{\env, \vetwo} \evalarrow \ResultExprSEF(\nvint(b), \Ignore)
}{
  \dynamicdomain(\env, \overname{\ConstraintRange(\veone, \vetwo)}{\vc}) = \overname{\{ \nvint(n) \;|\;  a \leq n \land n \leq b\}}{\vd}
}
\and
\inferrule[constraint\_range\_abnormal1]{
  \evalexprsef{\env, \veone} \evalarrow C\\
  C \in \TDynError \cup \TDiverging
}{
  \dynamicdomain(\env, \overname{\ConstraintRange(\veone, \vetwo)}{\vc}) = \overname{\emptyset}{\vd}
}
\and
\inferrule[constraint\_range\_abnormal2]{
  \evalexprsef{\env, \veone} \evalarrow \ResultExprSEF(\Ignore, \Ignore)\\
  \evalexprsef{\env, \vetwo} \evalarrow C\\
  C \in \TDynError \cup \TDiverging
}{
  \dynamicdomain(\env, \overname{\ConstraintRange(\veone, \vetwo)}{\vc}) = \overname{\emptyset}{\vd}
}
\end{mathpar}

The notation $L^\denv(\id)$ denotes the \nativevalue\ associated with the identifier $\id$
in the \emph{local dynamic environment} of $\denv$.
\begin{mathpar}
  \inferrule[t\_int\_parameterized]{
  L^\denv(\id) = \nvint(n)
}{
  \dynamicdomain(\env, \overname{\TInt(\parameterized(\id))}{\vt}) = \overname{\{ \nvint(n) \}}{\vd}
}
\end{mathpar}

\begin{mathpar}
\inferrule[t\_bits\_abnormal]{
  \evalexprsef{\env, \ve} \evalarrow C\\
  C \in \TDynError \cup \TDiverging
}{
  \dynamicdomain(\env, \overname{\TBits(\ve, \Ignore)}{\vt}) = \overname{\emptyset}{\vd}
}
\and
\inferrule[t\_bits\_negative\_width\_error]{
  \evalexprsef{\env, \ve} \evalarrow \ResultExprSEF(\nvint(k), \Ignore)\\
  k < 0
}{
  \dynamicdomain(\env, \overname{\TBits(\ve, \Ignore)}{\vt}) = \overname{\emptyset}{\vd}
}
\and
\inferrule[t\_bits\_empty]{
  \evalexprsef{\env, \ve} \evalarrow \ResultExprSEF(\nvint(0), \Ignore)
}{
  \dynamicdomain(\env, \overname{\TBits(\ve, \Ignore)}{\vt}) = \overname{\{ \nvbitvector(\emptylist) \}}{\vd}
}
\and
\inferrule[t\_bits\_non\_empty]{
  \evalexprsef{\env, \ve} \evalarrow \ResultExprSEF(\nvint(k), \Ignore)\\
  k > 0
}{
  \dynamicdomain(\env, \overname{\TBits(\ve, \Ignore)}{\vt}) = \overname{\{ \nvbitvector(\vb_{1..k}) \;|\; \vb_1,\ldots,\vb_k \in \Bit \}}{\vd}
}
\end{mathpar}

\begin{mathpar}
\inferrule[t\_tuple]{
  i=1..k: \dynamicdomain(\env, \vt_i) = D_i
}{
  \dynamicdomain(\env, \overname{\TTuple(\vt_{1..k})}{\vt}) =
  \overname{\{ \nvvector(\vv_{1..k}) \;|\; \vv_i \in D_i \}}{\vd}
}
\end{mathpar}

\begin{mathpar}
\inferrule[t\_array\_abnormal]{
  \evalexprsef{\env, \ve} \evalarrow C\\
  C \in \TDynError \cup \TDiverging
}{
  \dynamicdomain(\env, \overname{\TArray(\ArrayLengthExpr(\ve), \telem)}{\vt}) = \overname{\emptyset}{\vd}
}
\and
\inferrule[t\_array\_negative\_length\_error]{
  \evalexprsef{\env, \ve} \evalarrow \ResultExprSEF(\nvint(k), \Ignore)\\
  k < 0
}{
  \dynamicdomain(\env, \overname{\TArray(\ArrayLengthExpr(\ve), \telem)}{\vt}) = \overname{\emptyset}{\vd}
}
\and
\inferrule[t\_array\_okay]{
  \evalexprsef{\env, \ve} \evalarrow \ResultExprSEF(\nvint(k), \Ignore)\\
  k \geq 0\\
  \dynamicdomain(\env, \telem) = D_\telem
}{
  \dynamicdomain(\env, \overname{\TArray(\ArrayLengthExpr(\ve), \telem)}{\vt}) =
  \overname{\{ \nvvector(\vv_{1..k}) \;|\; \vv_{1..k} \in D_{\telem} \}}{\vd}
}
\end{mathpar}

\begin{mathpar}
\inferrule[t\_enum\_array]{
  \env \eqname (\tenv, \Ignore)\\
  G^\tenv.\declaredtypes(\id) = \TEnum(\vl_{1..k})\\
  \dynamicdomain(\env, \telem) = D_\telem
}{
  {
  \begin{array}{c}
    \dynamicdomain(\env, \overname{\TArray(\ArrayLengthEnum(\id, k), \telem)}{\vt}) =\\
    \overname{\{ \nvrecord(\{i=1..k: \vl_i\mapsto \vv_i\}) \;|\; \vv_i \in D_\telem \}}{\vd}
  \end{array}
  }
}
\end{mathpar}

\begin{mathpar}
\inferrule[structured]{
  L \in \{\TRecord, \TException\}\\
  i=1..k: \dynamicdomain(\env, \vt_i) = D_i
}{
  \dynamicdomain(\env, \overname{L([i=1..k: (\id_i,\vt_i))]}{\vt}) = \\
  \overname{\{ \nvrecord(\{i=1..k: \id_i\mapsto \vv_i\}) \;|\; \vv_i \in D_i \}}{\vd}
}
\end{mathpar}

\begin{mathpar}
\inferrule[t\_named]{
  G^\tenv.\declaredtypes(\id)=\tty
}{
  \dynamicdomain(\env, \overname{\TNamed(\id)}{\vt}) = \overname{\dynamicdomain(\env, \tty)}{\vd}
}
\end{mathpar}

\ExampleDef{Type Domains}
The domain of \texttt{integer} is the infinite set of all integers.

The domain of \verb|integer {2,16}| is the set $\{\nvint(2), \nvint(16)\}$.

The domain of \verb|integer{1..3}| is the set $\{\nvint(1), \nvint(2), \nvint(3)\}$.

The domain of \verb|integer{10..1}| is the empty set as there are no integers that are
both greater than $10$ and smaller than $1$.

The domain of \texttt{bits(2)} is the set $\{\nvbitvector(00)$, $\nvbitvector(01),$
$\nvbitvector(10)$, $\nvbitvector(11)\}$.

The domain of \verb|enumeration {GREEN, ORANGE, RED}| is the set \\
$\{\nvlabel(\texttt{GREEN}), \nvlabel(\texttt{ORANGE}), \nvlabel(\texttt{RED})\}$ and so is the domain
of \\
\verb|type TrafficLights of enumeration {GREEN, ORANGE, RED}|.

The domain of \texttt{(integer, integer)} is the set containing all pairs of native integer values.

The domain of \verb|record {a: integer;  b: boolean}| contains all native records
that map \texttt{a} to a native integer value and \texttt{b} to a native Boolean value.

The dynamic domain of a subprogram parameter \texttt{N: integer} is the (singleton) set containing
the native integer value $c$,
which is assigned to \texttt{N} by a given dynamic environment. The static domain of that parameter
is the infinite set of all native integer values.

\identd{BMGM} \identr{PHRL} \identr{PZNR}
\identr{RLQP} \identr{LYDS} \identr{SVDJ} \identi{WLPJ} \identr{FWMM}
\identi{WPWL} \identi{CDVY} \identi{KFCR} \identi{BBQR} \identr{ZWGH}
\identr{DKGQ} \identr{DHZT} \identi{HSWR} \identd{YZBQ}
