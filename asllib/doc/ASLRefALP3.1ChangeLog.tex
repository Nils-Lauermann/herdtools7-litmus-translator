\section{ALP3.1}

The following changes have been made.

\subsection{ASL-790: reserved keywords}
The following keywords are no longer reserved:
%
\texttt{SAMPLE},
\texttt{UNSTABLE},
\texttt{\_},
\texttt{any},
\texttt{assume},
\texttt{assumes},
\texttt{call},
\texttt{cast},
\texttt{class},
\texttt{dict},
\texttt{endcase},
\texttt{endcatch},
\texttt{endclass},
\texttt{endevent},
\texttt{endfor},
\texttt{endfunc},
\texttt{endgetter},
\texttt{endif},
\texttt{endmodule},
\texttt{endnamespace},
\texttt{endpackage},
\texttt{endproperty},\\
\texttt{endrule},
\texttt{endsetter},
\texttt{endtemplate},
\texttt{endtry},
\texttt{endwhile},
\texttt{event},
\texttt{export},
\texttt{extends},
\texttt{extern},
\texttt{feature},
\texttt{gives},
\texttt{iff},
\texttt{implies},
\texttt{import},
\texttt{intersect},
\texttt{intrinsic},
\texttt{invariant},
\texttt{list},
\texttt{map},
\texttt{module},
\texttt{namespace},
\texttt{newevent},
\texttt{newmap},
\texttt{original},
\texttt{package},
\texttt{parallel},
\texttt{port},
\texttt{private},
\texttt{profile},
\texttt{property},
\texttt{protected},
\texttt{public},
\texttt{requires},
\texttt{rethrow},
\texttt{rule},
\texttt{shared},
\texttt{signal},
\texttt{template},
\texttt{typeof},
\texttt{union},
\texttt{using},
\texttt{ztype}.

The following words have been added as reserved keywords:
\texttt{pure}, \texttt{readonly}, \texttt{collection}.

\subsection{ASL-758: support for string concatenation}
The binary \texttt{::} operator is overloaded to handle both bit vector and string concatenation.
For string concatenation, its operands are converted to strings (as in printing, $\literaltostring$) and the resulting strings are concatenated.

\subsection{ASL-791: typechecking of \texttt{impdef} subprograms}
Subprograms marked \texttt{impdef} that have been overridden by a corresponding \texttt{implementation} subprogram are now typechecked, and not simply discarded.
See \secref{Overriding}.

\subsection{ASL-792: discarding local/global storage elements}
The ASL grammar now forbids discarding of global storage elements.
Local storage element declarations must bind at least one name (see \RequirementRef{DiscardingLocalStorageDeclarations}).
For example:
\begin{lstlisting}
var - = ...;           // ERROR
let - = ...;           // ERROR
constant - = ...;      // ERROR
config - = ...;        // ERROR

func foo()
begin
  var - = ...;         // ERROR
  let - = ...;         // ERROR
  constant - = ...;    // ERROR
  let (-, -, -) = ...; // ERROR

  - = ...;             // OK - not a storage declaration
end;
\end{lstlisting}

\subsection{ASL-797: use of \texttt{elsif}}
The keyword \texttt{elsif} is no longer valid in expressions.
It can still be used for statements.

\subsection{ASL-800: mask syntax}
The two forms of ``don't care'' characters (\texttt{x} and parentheses) can now be mixed.
For example:
\begin{lstlisting}
  '01000111 1 xxxx (0)(0)(0)'
\end{lstlisting}

\subsection{ASL-805: base values of bit vectors}
Base values of bit vectors previously required statically known bit widths.
This has been relaxed: the base value of a bit vector of type \texttt{bits(e)} can always be recovered as \texttt{0[:e]}.

\subsection{ASL-808: standard library additions}
\texttt{IsAlignedSize()} and \texttt{IsAlignedP2()} have been added to the standard library.
Each function has both integer- and bit vector-flavoured versions.

\subsection{ASL-819: updates to paired accessor syntax}
Paired accessor syntax has been updated to reduce usage of the keyword \texttt{begin}, and to move the declaration of the additional argument of the setter to the top-level.
The new syntax looks as follows:
\begin{lstlisting}
accessor Name{params}(args) <=> value_in: return_type
begin
  getter
    ... // getter implementation
  end;

  setter
    ... // setter implementation
  end;
end;
\end{lstlisting}

\subsection{ASL-821: syntax for empty records/exceptions/collections}
Empty records/exceptions/collections are now declared and constructed with \texttt{RecordName\{-\}}, rather than \texttt{RecordName\{\}}.
Note that for declarations, the braces can still be omitted entirely.
This permits parameter elision to use any of the following equivalent syntaxes for a function with no arguments:
\begin{lstlisting}
  let x: bits(64) = Zeros{};     // NEW
  let x: bits(64) = Zeros{}();
  let x: bits(64) = Zeros{64};
  let x: bits(64) = Zeros{64}();
\end{lstlisting}

\subsection{ASL-822: signatures for \texttt{UInt} and \texttt{SInt}}
The signatures are now as follows:
\begin{lstlisting}
  func UInt{N} (x: bits(N)) => integer{0..2^N-1}
  func SInt{N} (x: bits(N))
    => integer{(if N == 0 then 0 else -(2^(N-1))) .. (if N == 0 then 0 else 2^(N-1)-1)}
\end{lstlisting}

The constraints are also removed from the signatures of \texttt{AlignDownSize} and \texttt{AlignUpSize}.
