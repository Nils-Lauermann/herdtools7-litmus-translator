\chapter{Symbolic Reduction and Equivalence Testing\label{chap:SymbolicEquivalenceTesting}}

In this chapter, we define two forms of symbolic reasoning ---
\emph{symbolic reduction} and \emph{symbolic equivalence testing}.
Symbolic reduction simplifies expressions into \emph{equivalent} expressions
that are simpler to reason about.
In our context, equivalence means that we can substitute one expression for another without
affecting the semantics of the overall specification.
%
Symbolic equivalence is a \emph{conservative} test.
By conservative, we mean that if a test for equivalence returns $\True$ then the expressions
being compared are indeed equivalent, but if the test returns $\False$ then
there are two possibilities:
\begin{itemize}
  \item the expressions are not equivalent;
  \item the expressions are equivalent, but the reasoning power of our rules
  is not enough to prove it, and so we conservatively answer negatively.
\end{itemize}
In proof-theoretic terms, we can say that our equivalence tests are \emph{sound} but \emph{incomplete}.

Notice that for a conservative test, it is always correct to return $\False$.

\hypertarget{def-unsupportedexpression}{}
In this chapter, we use the special value $\CannotBeTransformed$
to represent a failure in transforming an expression into a desired form (the specific
desired form varies according to the functions utilizing this value).

We first define symbolic expressions and operations over symbolic expressions in \secref{symbolicexpressions}
and then define the rules for symbolic reduction and equivalence testing in \secref{SymbolicReductionAndEquivalenceTestingRules}.

\section{Symbolic Expressions\label{sec:symbolicexpressions}}
Our symbolic reduction and equivalence testing rules use \emph{symbolic expressions}, defined below:
\[
  \begin{array}{rcl}
    \polynomial &\triangleq& \unitarymonomial \partialto \Q \setminus \{0\}\\
    \unitarymonomial   &\triangleq& \Identifiers \partialto \Npos\\
  \end{array}
\]

We now explain each component of a symbolic expression and how it can be interpreted as a mathematical formula
via the interpretation function $\alpha$.
We also define operations over symbolic expressions.

\hypertarget{def-unitarymonomial}{}
\hypertarget{def-monomial}{}
\begin{definition}[Unitary Monomial]
A \emph{Unitary Monomial} is a partial function from identifiers to positive integers\footnote{A unitary monomial has a unit factor,
for example $x^3$, whereas a non-unitary monomial has a non-unit factor, for example, $2 x^3$.}.

A non-empty unitary monomial, $\vm\in\unitarymonomial$ where $\vm \neq \emptyfunc$, can be interpreted as follows:
\[
  \alpha(\vm) \triangleq \prod_{\vx \in \dom(\vm)} \vx^{\vm(\vx)} \enspace.
\]

An empty unitary monomial is interpreted as the constant $1$:
\[
  \alpha(\emptyfunc) \triangleq 1 \enspace.
\]
\end{definition}
For example,
\[
  \alpha(\ \{\vx\mapsto 3, \vy\mapsto 1, \vz\mapsto2\}\ ) = x^3 \cdot y \cdot z^2 \enspace.
\]

\hypertarget{def-mulmonomials}{}
The function
\[
  \mulmonomials(\overname{\unitarymonomial}{\vmone} \aslsep \overname{\unitarymonomial}{\vmtwo}) \rightarrow
  \overname{\unitarymonomial}{\vm}
\]
multiplies two unitary monomials and returns a unitary monomial
\begin{mathpar}
  \inferrule{
    {
      \vf \eqdef \lambda \vx\in\identifier.\
      \left\{
      \begin{array}{ll}
        \vfone(\vx) & \text{if } \vx \in \dom(\vfone) \setminus \dom(\vftwo)\\
        \vftwo(\vx) & \text{if } \vx \in \dom(\vftwo) \setminus \dom(\vfone)\\
        \vfone(\vx)+\vftwo(\vx) & \text{else } \vx \in \dom(\vfone) \cap \dom(\vftwo)\\
      \end{array}
      \right.
    }
  }
  {
    \mulmonomials(\overname{\vfone}{\vmone}, \overname{\vftwo}{\vmtwo}) \typearrow \overname{\vf}{\vm}
  }
\end{mathpar}
For example,
\[
  \begin{array}{ll}
  \mulmonomials( & \{\vx\mapsto 3, \vy\mapsto 1, \vz\mapsto2\}, \{\vx\mapsto 1, \vw\mapsto 2\}\ ) =\\
                 & \{\vx\mapsto 4, \vy\mapsto 1, \vz\mapsto2, \vw\mapsto2\}
  \end{array}
\]

\hypertarget{def-polynomial}{}
\begin{definition}[Polynomial]
  \emph{Polynomials} are partial functions from monomials to rationals other than zero.
  Intuitively, each unitary monomial is mapped to its factor in the polynomial.
  A polynomial $\vp$ can be interpreted as follows:
  %
\[
  \alpha(\vp) \triangleq \sum_{\vm \in \dom(\vp)} \vp(\vm)\cdot\alpha(\vm)
\]
\end{definition}
For example,
\[
  \left(\left\{
    \begin{array}{lcl}
      \{\vx\mapsto 3, \vy\mapsto 1, \vz\mapsto2\} &\mapsto& -1,\\
      \{\vx\mapsto 2, \vy\mapsto 1\} &\mapsto& \frac{3}{4}
    \end{array} \right\}\right) =
    -1\cdot x^3 \cdot y \cdot z^2 + \frac{3}{4} \cdot \vx^2\cdot \vy \enspace.
\]

\hypertarget{def-addpolynomials}{}
The function
\[
  \addpolynomials : \polynomial \times \polynomial \rightarrow \polynomial
\]
adds two polynomials:
\begin{mathpar}
\inferrule{
  {
    \vf \eqdef \lambda \vm\in\unitarymonomial.\
    \left\{
    \begin{array}{ll}
      \vfone(\vm) & \text{if } \vm \in \dom(\vfone) \setminus \dom(\vftwo)\\
      \vftwo(\vm) & \text{if } \vm \in \dom(\vftwo) \setminus \dom(\vfone)\\
      \bot        & \text{if } \vm \in \dom(\vfone) \cap \dom(\vftwo) \text{ and } \vfone(\vm)+\vftwo(\vm) = 0\\
      \vfone(\vm)+\vftwo(\vm) & \text{else } \vm \in \dom(\vfone) \cap \dom(\vftwo) \text{ and } \vfone(\vm)+\vftwo(\vm) \neq 0\\
    \end{array}
    \right.
  }
}{
  \addpolynomials(\overname{\vfone}{\vpone}, \overname{\vftwo}{\vptwo}) \typearrow \overname{\vf}{\vp}
}
\end{mathpar}

The overloaded function
\[
  \addpolynomials : \polynomial^* \rightarrow \polynomial
\]
adds a list of polynomials:
\begin{mathpar}
\inferrule[empty]{}{ \addpolynomials(\emptylist) \typearrow \emptyfunc }
\and
\inferrule[one]{}{ \addpolynomials([ \vp ]) \typearrow \vp }
\and
\inferrule[two\_or\_more]{
  \addpolynomials(\vp_{2..k}) \typearrow \vpp\\
  \addpolynomials(\vp_1, \vpp) \typearrow \vp
}{
  \addpolynomials(\vp_{1..k}) \typearrow \vp
}
\end{mathpar}

The function
\hypertarget{def-mulpolynomials}{}
\[
  \mulpolynomials : \overname{\polynomial}{\vpone} \times \overname{\polynomial}{\vptwo} \rightarrow \overname{\polynomial}{\vp}
\]
multiplies two polynomials.
\begin{mathpar}
\inferrule{
  {
    \vps \eqdef \left\{
      \begin{array}{l}
        \{\mulmonomials(\vmone, \vmtwo) \mapsto \vfone(\vmone)\times\vftwo(\vmtwo)\}  \;|\; \\
       \vmone\in\dom(\vfone)\ \land\ \vmtwo\in\dom(\vftwo)
    \end{array}
    \right\}
  }\\
  \orderedps \eqdef \listset(\vps)\\
  \addpolynomials(i=1..k: \orderedps) \typearrow \vp\\
}{
  \mulpolynomials(\overname{\vfone}{\vpone}, \overname{\vftwo}{\vptwo}) \typearrow \vp
}
\end{mathpar}

\section{Typing Rules\label{sec:SymbolicReductionAndEquivalenceTestingRules}}
We employ the following rules:
\begin{itemize}
  \item \TypingRuleRef{Normalize}
  \item \TypingRuleRef{ReduceConstraint}
  \item \TypingRuleRef{ReduceConstraints}
  \item \TypingRuleRef{ToIR}
  \item \TypingRuleRef{ExprEqual}
  \item \TypingRuleRef{ExprEqualNorm}
  \item \TypingRuleRef{ExprEqualCase}
  \item \TypingRuleRef{TypeEqual}
  \item \TypingRuleRef{BitwidthEqual}
  \item \TypingRuleRef{BitFieldsEqual}
  \item \TypingRuleRef{BitFieldEqual}
  \item \TypingRuleRef{ConstraintEqual}
  \item \TypingRuleRef{SlicesEqual}
  \item \TypingRuleRef{SliceEqual}
  \item \TypingRuleRef{ArrayLengthEqual}
\end{itemize}

\TypingRuleDef{Normalize}
\hypertarget{def-normalize}{}
The function
\[
\normalize(\overname{\staticenvs}{\tenv} \aslsep \overname{\expr}{\ve}) \aslto \overname{\expr}{\newe} \cup \overname{\TTypeError}{\TypeErrorConfig}
\]
\hypertarget{def-symbolicallysimplifies}{symbolically simplifies} an expression $\ve$ in the static environment $\tenv$,
yielding an expression $\newe$.
\ProseOtherwiseTypeError

\hypertarget{def-normalizedterm}{}
We refer to an expression in the image of $\normalize$ as a \emph{\normalizedexpressionterm}.

\ProseParagraph
\OneApplies
\begin{itemize}
  \item \AllApplyCase{normalizable}
  \begin{itemize}
    \item applying $\toir$ to $\ve$ in $\tenv$ to obtain a symbolic expression yields a symbolic expression $\vpone$\ProseOrTypeError;
    \item applying $\polynomialtoexpr$ to $\vpone$ to transform it into an expression yields $\newe$.
  \end{itemize}

  \item \AllApplyCase{not\_normalizable}
  \begin{itemize}
    \item applying $\toir$ to $\ve$ in $\tenv$ to obtain a symbolic expression yields $\CannotBeTransformed$,
          indicating it cannot be transformed to a corresponding symbolic expression;
    \item define $\newe$ as $\ve$.
  \end{itemize}
\end{itemize}
\FormallyParagraph
\begin{mathpar}
\inferrule[normalizable]{
  \toir(\tenv, \ve) \typearrow \vpone \OrTypeError\\\\
  \vpone \neq \CannotBeTransformed\\
  \polynomialtoexpr(\vpone) \typearrow \newe
}{
  \normalize(\tenv, \ve) \typearrow \newe
}
\and
\inferrule[not\_normalizable]{
  \toir(\tenv, \ve) \typearrow \CannotBeTransformed
}{
  \normalize(\tenv, \ve) \typearrow \overname{\ve}{\newe}
}
\end{mathpar}
\CodeSubsection{\NormalizeBegin}{\NormalizeEnd}{../StaticModel.ml}

\TypingRuleDef{ReduceConstraint}
\hypertarget{def-reduceconstraint}{}
The function
\[
\reduceconstraint(
  \overname{\staticenvs}{\tenv} \aslsep
  \overname{\intconstraint}{\vc}
) \aslto
\overname{\intconstraint}{\newc}
\]
\symbolicallysimplifies\ an integer constraint $\vc$, yielding the integer constraint $\newc$

\ProseParagraph
\OneApplies
\begin{itemize}
  \item \AllApplyCase{exact}
  \begin{itemize}
    \item $\vc$ is an exact integer constraint for $\ve$, that is, $\ConstraintExact(\ve)$;
    \item applying $\normalize$ to $\ve$ in $\tenv$ yields $\vep$;
    \item define $\newc$ as the exact integer constraint for $\vep$, that is, $\ConstraintExact(\ve)$.
  \end{itemize}

  \item \AllApplyCase{range}
  \begin{itemize}
    \item $\vc$ is an range integer constraint for $\veone$ and $\vetwo$, that is, $\ConstraintRange(\veone, \vetwo)$;
    \item applying $\normalize$ to $\veone$ in $\tenv$ yields $\veonep$;
    \item applying $\normalize$ to $\vetwo$ in $\tenv$ yields $\vetwop$;
    \item define $\newc$ as the exact integer constraint for $\vep$, that is, \\ $\ConstraintRange(\veonep, \vetwop)$.
  \end{itemize}
\end{itemize}

\CodeSubsection{\ReduceConstraintBegin}{\ReduceConstraintEnd}{../StaticOperations.ml}

\FormallyParagraph
\begin{mathpar}
\inferrule[exact]{
  \normalize(\tenv, \ve) \typearrow \vep
}{
  \reduceconstraint(\tenv, \overname{\ConstraintExact(\ve)}{\vc}) \typearrow \overname{\ConstraintExact(\vep)}{\newc}
}
\end{mathpar}

\begin{mathpar}
\inferrule[range]{
  \normalize(\tenv, \veone) \typearrow \veonep\\
  \normalize(\tenv, \vetwo) \typearrow \vetwop\\
}{
  \reduceconstraint(\tenv, \overname{\ConstraintRange(\veone, \vetwo)}{\vc}) \typearrow \overname{\ConstraintRange(\veonep, \vetwop)}{\newc}
}
\end{mathpar}

\TypingRuleDef{ReduceConstraints}
\hypertarget{def-reduceconstraints}{}
The function
\[
\reduceconstraints(
  \overname{\staticenvs}{\tenv} \aslsep
  \overname{\intconstraint^*}{\cs}
) \aslto
\overname{\intconstraint^*}{\newcs}
\]
\symbolicallysimplifies\ a list of integer constraints $\cs$, yielding a list of integer constraints $\newcs$

\ProseParagraph
\AllApply
\begin{itemize}
  \item applying $\reduceconstraint$ to every constraint $\cs[\vi]$ in $\tenv$ for every $\vi$ in \\
        $\listrange(\cs)$ yields $\vc_\vi$;
  \item define $\newcs$ as the list containing $\vc_\vi$ for every $\vi$ in $\listrange(\cs)$.
\end{itemize}

\CodeSubsection{\ReduceConstraintsBegin}{\ReduceConstraintsEnd}{../StaticOperations.ml}

\FormallyParagraph
\begin{mathpar}
\inferrule{
  \vi\in\listrange(\cs): \reduceconstraint(\tenv, \cs[\vi]) \typearrow \vc_\vi\\
  \newcs \eqdef [\vi\in\listrange(\cs): \vc_\vi]
}{
  \reduceconstraints(\tenv, \cs) \typearrow \newcs
}
\end{mathpar}

\TypingRuleDef{ToIR}
\hypertarget{def-toir}{}
The function
\[
  \toir(\overname{\staticenvs}{\tenv} \aslsep \overname{\expr}{\ve}) \aslto
  \overname{\polynomial}{\vp}\ \cup\ \{\CannotBeTransformed\}\ \cup\ \overname{\TTypeError}{\TypeErrorConfig}
\]
transforms a subset of ASL expressions into symbolic expressions. If an ASL expression cannot be represented
by a symbolic expression (because, for example, it contains operations that are not available in symbolic expressions),
the special value $\CannotBeTransformed$ is returned.

\ProseParagraph
Intuitively, $\toir$ conducts a case analysis to determine whether the ASL expression corresponds to a polynomial.

\OneApplies
\begin{itemize}
  \item \AllApplyCase{literal\_int}
  \begin{itemize}
    \item $\ve$ is an integer literal expression for $\vi$, that is, $\ELiteral(\lint(\vi))$;
    \item $\vp$ is the symbolic expression for $\vi$.
  \end{itemize}

  \item \AllApplyCase{literal\_other}
  \begin{itemize}
    \item $\ve$ is a variable expression other than an integer literal;
    \item $\vp$ is $\CannotBeTransformed$.
  \end{itemize}

  \item \AllApplyCase{evar\_int\_constant}
  \begin{itemize}
    \item $\ve$ is a variable expression with identifier $\vs$, that is, $\EVar(\vs)$;
    \item looking up the constant associated with $\vs$ in $\tenv$ via $\lookupconstant$ yields the literal expression for $\vv$, that is, $\ELiteral(\vv)$;
    \item checking whether $\vv$ is an integer literal yields $\True$\ProseOrTypeError;
    \item $\vv$ is an integer literal for $\vi$;
    \item $\vp$ is the symbolic expression for $\vi$, that is, $\{ \emptyfunc\mapsto \vi \}$.
  \end{itemize}

  \item \AllApplyCase{evar\_immutable\_expr}
  \begin{itemize}
    \item $\ve$ is a variable expression with identifier $\vs$, that is, $\EVar(\vs)$;
    \item looking up the constant associated with $\vs$ in $\tenv$ via $\lookupconstant$ yields $\bot$;
    \item looking up $\vs$ in $\tenv$ for an associated immutable expression via \\
          $\lookupimmutableexpr$ yields the expression $\vep$;
    \item applying $\toir$ to $\vep$ in $\tenv$ yields $\vp$.
  \end{itemize}

  \item \AllApplyCase{evar\_exact\_constraint}
  \begin{itemize}
    \item $\ve$ is a variable expression with identifier $\vs$, that is, $\EVar(\vs)$;
    \item looking up the constant associated with $\vs$ in $\tenv$ via $\lookupconstant$ yields $\bot$;
    \item looking up $\vs$ in $\tenv$ for an associated immutable expression via \\
          $\lookupimmutableexpr$ yields $\bot$;
    \item determining the type of $\vs$ yields $\vt$\ProseOrTypeError;
    \item the \underlyingtype\ of $\vt$ is $\ttyone$\ProseOrTypeError;
    \item checking whether $\ttyone$ is an integer type yields $\True$\ProseOrTypeError;
    \item $\ttyone$ is a well-constrained integer with the exact constraint $\ve$, that is, \\ $\TInt(\wellconstrained([\ConstraintExact(\ve)]))$;
    \item converting $\ve$ to a symbolic expression yields $\vp$\ProseTerminateAs{\CannotBeTransformed}.
  \end{itemize}

  \item \AllApplyCase{int\_var}
  \begin{itemize}
    \item $\ve$ is a variable expression with identifier $\vs$, that is, $\EVar(\vs)$;
    \item looking up the constant associated with $\vs$ in $\tenv$ yields $\bot$;
    \item determining the type of $\vs$ yields $\vt$\ProseOrTypeError;
    \item the \underlyingtype\ of $\vt$ is $\ttyone$\ProseOrTypeError;
    \item checking whether $\ttyone$ is an integer type yields $\True$\ProseOrTypeError;
    \item $\ttyone$ is not a well-constrained integer with a single exact constraint;
    \item $\vp$ is the symbolic expression for the variable $\vs$, that is, $\{ \{\vs\mapsto 1\}\mapsto 1 \}$.
  \end{itemize}

  \item \AllApplyCase{ebinop\_plus}
  \begin{itemize}
    \item $\ve$ is a binary addition expression with operands $\veone$ and $\vetwo$, that is, \\ $\EBinop(\PLUS, \veone, \vetwo)$;
    \item converting $\veone$ to a symbolic expression in $\tenv$ yields $\irone$\ProseOrTypeErrorOrCannotBeTransformed;
    \item converting $\vetwo$ to a symbolic expression in $\tenv$ yields $\irtwo$\ProseOrTypeErrorOrCannotBeTransformed;
    \item $\vp$ is the symbolic expression adding up $\irone$ and $\irtwo$.
  \end{itemize}

  \item \AllApplyCase{ebinop\_minus}
  \begin{itemize}
    \item $\ve$ is a binary subtraction expression with operands $\veone$ and $\vetwo$, that is, \\ $\EBinop(\MINUS, \veone, \vetwo)$;
    \item $\vep$ is the addition expression with operands $\veone$ and the negation of $\vetwo$, that is, \\ $\EBinop(\PLUS, \veone, \EUnop(\MINUS, \vetwo))$;
    \item converting $\vep$ into a symbolic expression in $\tenv$ yields $\vp$\ProseOrTypeErrorOrCannotBeTransformed.
  \end{itemize}

  \item \AllApplyCase{ebinop\_mul\_div\_left}
  \begin{itemize}
    \item $\ve$ is a binary multiplication expression where the left operand is a binary division expression over $\veone$ and $\vetwo$
          and the right operand is $\vethree$, that is, \\ $\EBinop(\MUL, \EBinop(\DIV, \veone, \vetwo), \vethree)$;
    \item converting the binary division expression where the left operand is the binary multiplication expression  over $\veone$
          and $\vethree$ and the right operand is $\vetwo$ yields $\vp$\ProseOrTypeErrorOrCannotBeTransformed.
  \end{itemize}

  \item \AllApplyCase{ebinop\_mul\_div\_right}
  \begin{itemize}
    \item $\ve$ is a binary multiplication expression where the left operand is $\veone$ and the right operand
          is the division expression over $\vetwo$ and $\vethree$, that is, \\
          $\EBinop(\MUL, \veone, \EBinop(\DIV, \vetwo, \vethree))$;
    \item $\veone$ is not a binary division expression;
    \item converting the binary division expression where the left operand is the binary multiplication expression  over $\veone$
          and $\vetwo$ and the right operand is $\vethree$ yields $\vp$\ProseOrTypeErrorOrCannotBeTransformed.
  \end{itemize}

  \item \AllApplyCase{ebinop\_mul}
  \begin{itemize}
    \item $\ve$ is a binary multiplication expression with operands $\veone$ and $\vetwo$, that is, \\ $\EBinop(\MUL, \veone, \vetwo)$;
    \item neither $\veone$ nor $\vetwo$ are binary vision expressions;
    \item converting $\veone$ to a symbolic expression in $\tenv$ yields $\irone$\ProseOrTypeErrorOrCannotBeTransformed;
    \item converting $\vetwo$ to a symbolic expression in $\tenv$ yields $\irtwo$\ProseOrTypeErrorOrCannotBeTransformed;
    \item $\vp$ is the symbolic expression multiplying $\irone$ and $\irtwo$.
  \end{itemize}

  \item \AllApplyCase{ebinop\_div\_int\_denominator}
  \begin{itemize}
    \item $\ve$ is a binary division expression with operands $\veone$ and $\vetwo$, that is, \\ $\EBinop(\DIV, \veone, \vetwo)$;
    \item $\vetwo$ is an integer literal expression for $\vitwo$;
    \item converting $\veone$ to a symbolic expression in $\tenv$ yields $\irone$\ProseOrTypeErrorOrCannotBeTransformed;
    \item $\vftwo$ is $\frac{1}{\vitwo}$ (testing against $\vitwo = 0$ is done dynamically);
    \item $\vp$ is the polynomial $\irone$ with each monomial multiplied by $\vftwo$.
  \end{itemize}

  \item \AllApplyCase{ebinop\_div\_monomial\_denominator}
  \begin{itemize}
    \item $\ve$ is a binary division expression with operands $\veone$ and $\vetwo$, that is, \\ $\EBinop(\DIV, \veone, \vetwo)$;
    \item converting $\veone$ to a symbolic expression in $\tenv$ yields $\irone$\ProseOrTypeErrorOrCannotBeTransformed;
    \item converting $\vetwo$ to a symbolic expression in $\tenv$ yields $\irtwo$\ProseOrTypeErrorOrCannotBeTransformed;
    \item $\irtwo$ consists of a single binding between the monomial $\vmono$ and the factor $\vfactor$;
    \item applying $\polynomialdividebyterm$ to $\irone$, $\vfactor$, and $\vmono$ yields $\vp$\ProseTerminateAs{\CannotBeTransformed}.
  \end{itemize}

  \item \AllApplyCase{ebinop\_div\_non\_monomial\_denominator}
  \begin{itemize}
    \item $\ve$ is a binary division expression with operands $\veone$ and $\vetwo$, that is, \\ $\EBinop(\DIV, \veone, \vetwo)$;
    \item converting $\veone$ to a symbolic expression in $\tenv$ yields $\irone$\ProseOrTypeErrorOrCannotBeTransformed;
    \item converting $\vetwo$ to a symbolic expression in $\tenv$ yields $\irtwo$\ProseOrTypeErrorOrCannotBeTransformed;
    \item $\irtwo$ does not consist of a single binding;
    \item the result is $\CannotBeTransformed$.
  \end{itemize}

  \item \AllApplyCase{ebinop\_shl\_non\_lint\_exponent}
  \begin{itemize}
    \item $\ve$ is a binary shift-left expression with operands $\veone$ and $\vetwo$, that is, \\ $\EBinop(\SHL, \veone, \vetwo)$;
    \item $\vetwo$ is not an integer literal expression;
    \item $\vp$ is $\CannotBeTransformed$.
  \end{itemize}

  \item \AllApplyCase{ebinop\_shl\_neg\_shift}
  \begin{itemize}
    \item $\ve$ is a binary shift-left expression with operands $\veone$ and $\vetwo$, that is, \\ $\EBinop(\SHL, \veone, \vetwo)$;
    \item $\vetwo$ is an integer literal expression for $\vitwo$;
    \item $\vitwo$ is negative;
    \item $\vp$ is $\CannotBeTransformed$.
  \end{itemize}

  \item \AllApplyCase{ebinop\_shl\_okay}
  \begin{itemize}
    \item $\ve$ is a binary shift-left expression with operands $\veone$ and $\vetwo$, that is, \\ $\EBinop(\SHL, \veone, \vetwo)$;
    \item $\vetwo$ is an integer literal expression for $\vitwo$;
    \item converting $\veone$ to a symbolic expression in $\tenv$ yields $\irone$\ProseOrTypeErrorOrCannotBeTransformed;
    \item $\vitwo$ is non-negative;
    \item $\vftwo$ is $2^{\vitwo}$;
    \item $\vp$ is the polynomial $\irone$ with each monomial multiplied by $\vftwo$.
  \end{itemize}

  \item \AllApplyCase{ebinop\_other\_non\_literals}
  \begin{itemize}
    \item $\ve$ is a binary expression with an operator $\op$ that is other than $\PLUS$, $\MINUS$, $\MUL$, or $\SHL$,
          applied to the operand expressions $\veone$ and $\vetwo$;
    \item at least one of $\veone$ and $\vetwo$ is not a literal expression;
    \item $\vp$ is $\CannotBeTransformed$.
  \end{itemize}

  \item \AllApplyCase{ebinop\_other\_literals\_non\_int\_result}
  \begin{itemize}
    \item $\ve$ is a binary expression with an operator $\op$ that is other than $\PLUS$, $\MINUS$, $\MUL$, $\DIV$, or $\SHL$,
          applied to the operand expressions $\veone$ and $\vetwo$;
    \item $\veone$ is the literal expression for literal $\vlone$;
    \item $\vetwo$ is the literal expression for literal $\vltwo$;
    \item statically applying $\op$ to $\vlone$ and $\vltwo$ yields the literal $\vl$, which is not an integer literal;
    \item $\vp$ is $\CannotBeTransformed$.
  \end{itemize}

  \item \AllApplyCase{ebinop\_other\_literals\_int\_result}
  \begin{itemize}
    \item $\ve$ is a binary expression with an operator $\op$ that is other than $\PLUS$, $\MINUS$, $\MUL$, or $\SHL$,
          applied to the operand expressions $\veone$ and $\vetwo$;
    \item $\veone$ is the literal expression for literal $\vlone$;
    \item $\vetwo$ is the literal expression for literal $\vltwo$;
    \item statically applying $\op$ to $\vlone$ and $\vltwo$ yields the integer literal for $k$;
    \item $\vp$ is the symbolic expression for the integer $k$, that is, $\{ \emptyfunc\mapsto k \}$.
  \end{itemize}

  \item \AllApplyCase{eunop\_neg}
  \begin{itemize}
    \item $\ve$ is a unary expression with the negation operator $\NEG$ and operand $\veone$;
    \item converting the binary expression with operator $\MUL$ and left-hand-side operand for the integer literal $-1$ and
    right-hand-side operand $\veone$ in $\tenv$ yields $\vp$\ProseOrTypeErrorOrCannotBeTransformed.
  \end{itemize}

  \item \AllApplyCase{eunop\_other}
  \begin{itemize}
    \item $\ve$ is a unary expression with an operator other than $\NEG$;
    \item $\vp$ is $\CannotBeTransformed$.
  \end{itemize}

  \item \AllApplyCase{atc}
  \begin{itemize}
    \item $\ve$ is an asserting type conversion for the subexpression $\vep$, that is, \\
          $\EATC(\vep, \Ignore)$;
    \item applying $\toir$ to $\vep$ yields $\vp$\ProseOrTypeErrorOrCannotBeTransformed.
  \end{itemize}

  \item \AllApplyCase{other}
  \begin{itemize}
    \item $\ve$ is an expression with a label other than $\ELiteral$, $\EVar$, $\EBinop$, $\EUnop$, and $\EATC$;
    \item $\vp$ is $\CannotBeTransformed$.
  \end{itemize}
\end{itemize}

\FormallyParagraph
\begin{mathpar}
\inferrule[literal\_int]{}
{
  \toir(\tenv, \overname{\ELiteral(\lint(\vi))}{\ve}) \typearrow \overname{\{ \emptyfunc\mapsto \vi \}}{\vp}
}
\end{mathpar}

\begin{mathpar}
\inferrule[literal\_other]{
  \astlabel(\vv) \neq \lint
}{
  \toir(\tenv, \overname{\ELiteral(\vv)}{\ve}) \typearrow \CannotBeTransformed
}
\end{mathpar}

\begin{mathpar}
\inferrule[evar\_int\_constant]{
  \lookupconstant(\tenv, \vs) \typearrow \ELiteral(\vv)\\
  \checktrans{\astlabel(\vv) = \lint}{ExpectedIntegerLiteral} \typearrow \True \OrTypeError\\\\
  \vv \eqname \lint(\vi)
}{
  \toir(\tenv, \overname{\EVar(\vs)}{\ve}) \typearrow \overname{\{ \emptyfunc\mapsto \vi \}}{\vp}
}
\end{mathpar}

\begin{mathpar}
\inferrule[evar\_immutable\_expr]{
  \lookupconstant(\tenv, \vs) \typearrow \CannotBeTransformed\\
  \lookupimmutableexpr(\tenv, \vs) \typearrow \vep\\
  \toir(\vep) \typearrow \vp
}{
  \toir(\tenv, \overname{\EVar(\vs)}{\ve}) \typearrow \vp
}
\end{mathpar}

\begin{mathpar}
\inferrule[evar\_exact\_constraint]{
  \lookupconstant(\tenv, \vs) \typearrow \CannotBeTransformed\\
  \lookupimmutableexpr(\tenv, \vs) \typearrow \bot\\
  \typeof(\vs) \typearrow \vt \OrTypeError\\\\
  \makeanonymous(\vt) \typearrow \ttyone \OrTypeError\\\\
  \checktrans{\astlabel(\ttyone) = \TInt}{ExpectedIntegerType} \typearrow \True \OrTypeError\\\\
  \ttyone = \TInt(\wellconstrained([\ConstraintExact(\ve)]))\\
  \toir(\ve) \typearrow \vp \terminateas \CannotBeTransformed
}{
  \toir(\tenv, \overname{\EVar(\vs)}{\ve}) \typearrow \vp
}
\end{mathpar}

\begin{mathpar}
\inferrule[int\_var]{
  \lookupconstant(\tenv, \vs) \typearrow \CannotBeTransformed\\
  \lookupimmutableexpr(\tenv, \vs) \typearrow \bot\\
  \typeof(\vs) \typearrow \vt\\
  \makeanonymous(\vt) \typearrow \ttyone\\
  \checktrans{\astlabel(\ttyone) = \TInt}{ExpectedIntegerType} \typearrow \True\\
  \ttyone \neq \TInt(\wellconstrained([\ConstraintExact(\Ignore)]))
}{
  \toir(\tenv, \overname{\EVar(\vs)}{\ve}) \typearrow \overname{\{ \{\vs\mapsto 1\}\mapsto 1 \}}{\vp}
}
\end{mathpar}

\begin{mathpar}
\inferrule[ebinop\_plus]{
  \toir(\tenv, \veone) \typearrow \irone \OrTypeError, \CannotBeTransformed\\\\
  \toir(\tenv, \vetwo) \typearrow \irtwo \OrTypeError, \CannotBeTransformed\\\\
  \vp \eqdef \addpolynomials(\irone, \irtwo)
}{
  \toir(\tenv, \overname{\EBinop(\PLUS, \veone, \vetwo)}{\ve}) \typearrow \vp
}
\end{mathpar}

\begin{mathpar}
\inferrule[ebinop\_minus]{
  \vep \eqdef \EBinop(\PLUS, \veone, \EUnop(\MINUS, \vetwo))\\
  \toir(\tenv, \vep) \typearrow \vp \OrTypeError, \CannotBeTransformed\\\\
}{
  \toir(\tenv, \overname{\EBinop(\MINUS, \veone, \vetwo)}{\ve}) \typearrow \vp
}
\end{mathpar}

\begin{mathpar}
\inferrule[ebinop\_mul\_div\_left]{
  \toir(\tenv, \EBinop(\DIV, \EBinop(\MUL, \veone, \vethree), \vetwo)) \typearrow \vp \OrTypeError, \CannotBeTransformed\\\\
}{
  \toir(\tenv, \overname{\EBinop(\MUL, \EBinop(\DIV, \veone, \vetwo), \vethree)}{\ve}) \typearrow \vp
}
\end{mathpar}

\begin{mathpar}
\inferrule[ebinop\_mul\_div\_right]{
  \veone \neq \EBinop(\DIV, \Ignore, \Ignore)\\
  \toir(\tenv, \EBinop(\DIV, \EBinop(\MUL, \veone, \vetwo), \vethree)) \typearrow \vp \OrTypeError, \CannotBeTransformed\\\\
}{
  \toir(\tenv, \overname{\EBinop(\MUL, \veone, \EBinop(\DIV, \vetwo, \vethree))}{\ve}) \typearrow \vp
}
\end{mathpar}

\begin{mathpar}
\inferrule[ebinop\_mul]{
  \veone \neq \EBinop(\DIV, \Ignore, \Ignore)\\
  \vetwo \neq \EBinop(\DIV, \Ignore, \Ignore)\\
  \toir(\tenv, \veone) \typearrow \irone \OrTypeError, \CannotBeTransformed\\\\
  \toir(\tenv, \vetwo) \typearrow \irtwo \OrTypeError, \CannotBeTransformed\\\\
  \vp \eqdef \mulpolynomials(\irone, \irtwo)
}{
  \toir(\tenv, \overname{\EBinop(\MUL, \veone, \vetwo)}{\ve}) \typearrow \vp
}
\end{mathpar}

\begin{mathpar}
\inferrule[ebinop\_div\_int\_denominator]{
  \toir(\tenv, \veone) \typearrow \irone \OrTypeError, \CannotBeTransformed\\\\
  \vftwo \eqdef \frac{1}{\vitwo}\\
  \irone \eqname [i=1..k: \vm_\vi \mapsto \vc_\vi]\\
  \vp \eqdef [i=1..k: \vm_\vi \mapsto \vc_\vi \times \vftwo]\\
}{
  \toir(\tenv, \overname{\EBinop(\DIV, \veone, \overname{\ELiteral(\lint(\vitwo))}{\vetwo})}{\ve}) \typearrow \vp
}
\end{mathpar}

\begin{mathpar}
\inferrule[ebinop\_div\_monomial\_denominator]{
  \toir(\tenv, \veone) \typearrow \irone \OrTypeError, \CannotBeTransformed\\\\
  \toir(\tenv, \vetwo) \typearrow \irtwo \OrTypeError, \CannotBeTransformed\\\\
  \irtwo = [\vmono \mapsto \vfactor]\\
  \polynomialdividebyterm(\irone, \vfactor, \vmono) \typearrow \vp \terminateas \CannotBeTransformed
}{
  \toir(\tenv, \overname{\EBinop(\DIV, \veone, \vetwo)}{\ve}) \typearrow \vp
}
\end{mathpar}
\begin{mathpar}
\inferrule[ebinop\_div\_non\_monomial\_denominator]{
  \toir(\tenv, \veone) \typearrow \irone \OrTypeError, \CannotBeTransformed\\\\
  \toir(\tenv, \vetwo) \typearrow \irtwo \OrTypeError, \CannotBeTransformed\\\\
  \irtwo \neq [\vmono \mapsto \vfactor]
}{
  \toir(\tenv, \overname{\EBinop(\DIV, \veone, \vetwo)}{\ve}) \typearrow \CannotBeTransformed
}
\end{mathpar}

\begin{mathpar}
\inferrule[ebinop\_shl\_non\_lint\_exponent]{
    \vetwo \neq \ELiteral(\lint(\Ignore))
}{
  \toir(\tenv, \overname{\EBinop(\SHL, \Ignore, \vetwo)}{\ve}) \typearrow \CannotBeTransformed
}
\end{mathpar}

\begin{mathpar}
\inferrule[ebinop\_shl\_neg\_shift]{
  \vitwo < 0
}{
  \toir(\tenv, \overname{\EBinop(\SHL, \veone, \ELiteral(\lint(\vitwo)))}{\ve}) \typearrow \CannotBeTransformed
}
\and
  \inferrule[ebinop\_shl\_okay]{
    \toir(\tenv, \veone) \typearrow \irone \OrTypeError, \CannotBeTransformed\\\\
    \vitwo \geq 0\\
    \vftwo \eqdef 2^{\vitwo}\\
    \irone \eqname [i=1..k: \vm_\vi \mapsto \vc_\vi]\\
    \vp \eqdef [i=1..k: \vm_\vi \mapsto \vc_\vi \times \vftwo]\\
}{
  \toir(\tenv, \overname{\EBinop(\SHL, \veone, \ELiteral(\lint(\vitwo)))}{\ve}) \typearrow \vp
}
\end{mathpar}

\begin{mathpar}
\inferrule[ebinop\_other\_non\_literals]{
  \op \not\in \{\PLUS, \MINUS, \MUL, \DIV, \SHL\}\\
  (\veone \neq \ELiteral(\Ignore) \lor \vetwo \neq \ELiteral(\Ignore))
}{
  \toir(\tenv, \overname{\EBinop(\op, \veone, \vetwo)}{\ve}) \typearrow \CannotBeTransformed
}
\and
\inferrule[ebinop\_other\_literals\_non\_int\_result]{
  \op \not\in \{\PLUS, \MINUS, \MUL, \SHL\}\\
  \binopliterals(\op, \vlone, \vltwo) \typearrow \vl\\
  \vl \neq \lint(\Ignore)
}{
  \toir(\tenv, \overname{\EBinop(\op, \ELiteral(\vlone), \ELiteral(\vltwo))}{\ve}) \typearrow \CannotBeTransformed
}
\and
\inferrule[ebinop\_other\_literals\_int\_result]{
  \op \not\in \{\PLUS, \MINUS, \MUL, \SHL\}\\
  \binopliterals(\op, \vlone, \vltwo) \typearrow \lint(k)\\
  \vp \eqdef \{ \emptyfunc\mapsto k \}
}{
  \toir(\tenv, \overname{\EBinop(\op, \ELiteral(\vlone), \ELiteral(\vltwo))}{\ve}) \typearrow \vp
}
\end{mathpar}

\begin{mathpar}
\inferrule[eunop\_neg]{
  \toir(\tenv, \EBinop(\MUL, \ELiteral(\lint(-1)),\veone )) \typearrow \vp \OrTypeError, \CannotBeTransformed\\\\
}{
  \toir(\tenv, \overname{\EUnop(\NEG, \veone)}{\ve}) \typearrow \vp
}
\end{mathpar}

\begin{mathpar}
\inferrule[eunop\_other]{
  \op \neq \NEG
}{
  \toir(\tenv, \overname{\EUnop(\op, \Ignore)}{\ve}) \typearrow \CannotBeTransformed
}
\end{mathpar}

\begin{mathpar}
\inferrule[atc]{
  \toir(\tenv, \vep) \typearrow \vp \OrTypeError, \CannotBeTransformed\\\\
}{
  \toir(\tenv, \overname{\EATC(\vep, \Ignore)}{\ve}) \typearrow \vp
}
\end{mathpar}

\begin{mathpar}
\inferrule[other]{
  \astlabel(\ve) \not\in \{\ELiteral, \EVar, \EBinop, \EUnop, \EATC\}
}{
  \toir(\tenv, \ve) \typearrow \CannotBeTransformed
}
\end{mathpar}
\CodeSubsection{\ToIRBegin}{\ToIREnd}{../StaticModel.ml}

\TypingRuleDef{ExprEqual}
\hypertarget{def-exprequal}{}
The function
\[
  \exprequal(\overname{\staticenvs}{\tenv} \aslsep \overname{\expr}{\veone} \aslsep \overname{\expr}{\vetwo}) \aslto
  \overname{\Bool}{\vb} \cup \overname{\TTypeError}{\TypeErrorConfig}
\]
conservatively checks whether the expression $\veone$ is equivalent to the expression $\vetwo$ in environment $\tenv$.
The result is given in $\vb$ or a \typingerrorterm{}, if one is detected.

\ProseParagraph
\OneApplies
\begin{itemize}
  \item \AllApplyCase{norm\_true}
  \begin{itemize}
    \item comparing $\veone$ to $\vetwo$ in $\tenv$ via $\exprequalnorm$ yields $\True$\ProseOrTypeError;
    \item $\vb$ is $\True$.
  \end{itemize}

  \item \AllApplyCase{norm\_false}
  \begin{itemize}
    \item comparing $\veone$ to $\vetwo$ in $\tenv$ via $\exprequalnorm$ yields $\False$;
    \item comparing $\veone$ to $\vetwo$ by case analysis via $\exprequalcase$ yields $\vb$\ProseOrTypeError.
  \end{itemize}
\end{itemize}

\FormallyParagraph
\begin{mathpar}
\inferrule[norm\_true]{
  \exprequalnorm(\tenv, \veone, \vetwo) \typearrow \True \OrTypeError
}{
  \exprequal(\tenv, \veone, \vetwo) \typearrow \True
}
\and
\inferrule[norm\_false]{
  \exprequalnorm(\tenv, \veone, \vetwo) \typearrow \False\\
  \exprequalcase(\tenv, \veone, \vetwo) \typearrow \vb \OrTypeError
}{
  \exprequal(\tenv, \veone, \vetwo) \typearrow \vb
}
\end{mathpar}

\TypingRuleDef{ExprEqualNorm}
\hypertarget{def-exprequalnorm}{}
The helper function
\[
  \exprequalnorm(\overname{\staticenvs}{\tenv} \aslsep \overname{\expr}{\veone} \aslsep \overname{\expr}{\vetwo})
  \aslto \overname{\Bool}{\vb} \cup \overname{\TTypeError}{\TypeErrorConfig}
\]
conservatively tests whether the expression $\veone$ is equivalent to the expression $\vetwo$ in environment $\tenv$
by attempting to transform both expressions to their symbolic expression form
and, if successful, comparing the resulting normal forms for equality.
The result is given in $\vb$ or a \typingerrorterm{}, if one is detected.

\ProseParagraph
\OneApplies
\begin{itemize}
  \item \AllApplyCase{all\_supported}
  \begin{itemize}
    \item transforming $\veone$ into a symbolic expression in $\tenv$ yields $\irone$\ProseOrTypeError;
    \item transforming $\vetwo$ into a symbolic expression in $\tenv$ yields $\irtwo$\ProseOrTypeError;
    \item $\vb$ is the result of equating $\irone$ and $\irtwo$.
  \end{itemize}

  \item \AllApplyCase{unsupported1}
  \begin{itemize}
    \item transforming $\veone$ into a symbolic expression in $\tenv$ yields $\CannotBeTransformed$;
    \item $\vb$ is $\False$;
  \end{itemize}

  \item \AllApplyCase{unsupported2}
  \begin{itemize}
    \item transforming $\veone$ into a symbolic expression in $\tenv$ yields $\irone$;
    \item transforming $\vetwo$ into a symbolic expression in $\tenv$ yields $\CannotBeTransformed$;
    \item $\vb$ is $\False$;
  \end{itemize}
\end{itemize}

\FormallyParagraph
\begin{mathpar}
\inferrule[all\_supported]{
  \toir(\veone) \typearrow \irone \OrTypeError\\\\
  \toir(\vetwo) \typearrow \irtwo \OrTypeError
}{
  \exprequalnorm(\tenv, \veone, \vetwo) \typearrow \overname{\irone = \irtwo}{\vb}
}
\and
\inferrule[unsupported1]{
  \toir(\veone) \typearrow \CannotBeTransformed
}{
  \exprequalnorm(\tenv, \veone, \vetwo) \typearrow \overname{\False}{\vb}
}
\and
\inferrule[unsupported2]{
  \toir(\veone) \typearrow \irone\\
  \toir(\vetwo) \typearrow \CannotBeTransformed
}{
  \exprequalnorm(\tenv, \veone, \vetwo) \typearrow \overname{\False}{\vb}
}
\end{mathpar}

\TypingRuleDef{ExprEqualCase}
\hypertarget{def-exprequalcase}{}
The helper function
\[
  \exprequalcase(\overname{\staticenvs}{\tenv} \aslsep \overname{\expr}{\veone} \aslsep \overname{\expr}{\vetwo})
  \aslto \overname{\Bool}{\vb} \cup \overname{\TTypeError}{\TypeErrorConfig}
\]
specializes the equivalence test for expressions $\veone$ and $\vetwo$ in $\tenv$
for the different types of expressions.
The result is given in $\vb$ or a \typingerrorterm{}, if one is detected.

\ProseParagraph
\OneApplies
\begin{itemize}
  \item \AllApplyCase{different\_labels}
  \begin{itemize}
    \item the AST labels of $\veone$ and $\vetwo$ are different;
    \item $\vb$ is $\False$.
  \end{itemize}

  \item \AllApplyCase{e\_binop}
  \begin{itemize}
    \item $\veone$ is a binary expression with operator $\opone$ and operands $\veoneone$ and $\veonetwo$,
          that is, $\EBinop(\opone, \veoneone, \veonetwo)$;
    \item $\vetwo$ is a binary expression with operator $\optwo$ and operands $\vetwoone$ and $\vetwotwo$,
          that is, $\EBinop(\optwo, \vetwoone, \vetwotwo)$;
    \item testing the equivalence of $\veoneone$ and $\vetwoone$ in $\tenv$ yields $\vbone$\ProseOrTypeError;
    \item testing the equivalence of $\veonetwo$ and $\vetwotwo$ in $\tenv$ yields $\vbtwo$\ProseOrTypeError;
    \item $\vb$ is $\True$ if and only if $\opone$ is equal to $\optwo$ and both $\vbone$ and $\vbtwo$ are $\True$.
  \end{itemize}

  \item \AllApplyCase{e\_call}
  \begin{itemize}
    \item $\veone$ is a call expression with subprogram name $\nameone$ and list of arguments $\vargsone$,
          that is, $\ECall(\nameone, \vargsone, \Ignore)$;
    \item $\vetwo$ is a call expression with subprogram name $\nametwo$ and list of arguments $\vargstwo$,
          that is, $\ECall(\nametwo, \vargstwo, \Ignore)$;
    \item checking whether $\nameone$ is equal to $\nametwo$ either yields $\True$ or $\False$, which short-circuits the entire rule;
    \item checking whether the lists of arguments $\vargsone$ and $\vargstwo$ have the same length yields
          $\True$ or $\False$, which short-circuits the entire rule;
    \item for each index $i$ in the list of indices for $\vargsone$, testing whether $\vargsone[i]$ is equivalent to $\vargstwo[i]$
          in $\tenv$ yields $\vb_i$\ProseOrTypeError;
    \item $\vb$ is $\True$ if and only if $\vb_i$ is $\True$ for each index $i$ in the list of indices for $\vargsone$.
  \end{itemize}

  \item \AllApplyCase{e\_cond}
  \begin{itemize}
    \item $\veone$ is a conditional expression with expressions $\veoneone$, $\veonetwo$, and $\veonethree$,
          that is, $\ECond(\veoneone, \veonetwo, \veonethree)$;
    \item $\vetwo$ is a conditional expression with expressions $\vetwoone$, $\vetwotwo$, and $\vetwothree$,
          that is, $\ECond(\vetwoone, \vetwotwo, \vetwothree)$;
    \item testing whether $\veoneone$ is equivalent to $\vetwoone$ yields $\vbone$\ProseOrTypeError;
    \item testing whether $\veonetwo$ is equivalent to $\vetwotwo$ yields $\vbtwo$\ProseOrTypeError;
    \item testing whether $\veonethree$ is equivalent to $\vetwothree$ yields $\vbthree$\ProseOrTypeError;
    \item $\vb$ is $\True$ if and only if all of $\vbone$, $\vbtwo$, and $\vbthree$ are $\True$.
  \end{itemize}

  \item \AllApplyCase{e\_slice}
  \begin{itemize}
    \item $\veone$ is a slicing expression with expression $\veoneone$ and list of slices $\slicesone$,
          that is, $\ESlice(\veoneone, \slicesone)$;
    \item $\veone$ is a slicing expression with expression $\vetwoone$ and list of slices $\slicestwo$,
          that is, $\ESlice(\vetwoone, \slicestwo)$;
    \item testing whether $\veoneone$ is equivalent to $\vetwoone$ yields $\vbone$\ProseOrTypeError;
    \item testing whether the lists of slices $\slicesone$ and $\slicestwo$ are equivalent in $\tenv$ yields $\vbtwo$\ProseOrTypeError;
    \item $\vb$ is $\True$ if and only both $\vbone$ and $\vbtwo$ are $\True$.
  \end{itemize}

  \item \AllApplyCase{e\_getarray}
  \begin{itemize}
    \item $\veone$ is an \arrayaccess\ expression with array expression $\veoneone$ and position expression $\veonetwo$,
          that is, $\EGetArray(\veoneone, \veonetwo)$;
    \item $\vetwo$ is an \arrayaccess\ expression with array expression $\vetwoone$ and position expression $\vetwotwo$,
          that is, $\EGetArray(\vetwoone, \vetwotwo)$;
    \item testing whether $\veoneone$ is equivalent to $\vetwoone$ yields $\vbone$\ProseOrTypeError;
    \item testing whether $\veonetwo$ is equivalent to $\vetwotwo$ yields $\vbtwo$\ProseOrTypeError;
    \item $\vb$ is $\True$ if and only both $\vbone$ and $\vbtwo$ are $\True$.
  \end{itemize}

  \item \AllApplyCase{e\_getfield}
  \begin{itemize}
    \item $\veone$ is a field access expression with subexpression $\veoneone$ and field name $\vfieldone$,
          that is, $\EGetField(\veoneone, \vfieldone)$;
    \item $\vetwo$ is a field access expression with subexpression $\vetwoone$ and field name $\vfieldtwo$,
          that is, $\EGetField(\vetwoone, \vfieldtwo)$;
    \item $\vbone$ is $\True$ if and only if $\vfieldone$ is equal to $\vfieldtwo$;
    \item testing whether $\veoneone$ is equivalent to $\vetwoone$ yields $\vbtwo$\ProseOrTypeError;
    \item $\vb$ is $\True$ if and only both $\vbone$ and $\vbtwo$ are $\True$.
  \end{itemize}

  \item \AllApplyCase{e\_getfields}
  \begin{itemize}
    \item $\veone$ is a fields access expression with subexpression $\veoneone$ and list of field names $\vfieldsone$,
          that is, $\EGetFields(\veoneone, \vfieldsone)$;
    \item $\vetwo$ is a fields access expression with subexpression $\vetwoone$ and list of field names $\vfieldstwo$,
          that is, $\EGetFields(\vetwoone, \vfieldstwo)$;
    \item $\vbone$ is $\True$ if and only if $\vfieldsone$ is equal to $\vfieldstwo$;
    \item testing whether $\veoneone$ is equivalent to $\vetwoone$ yields $\vbtwo$\ProseOrTypeError;
    \item $\vb$ is $\True$ if and only both $\vbone$ and $\vbtwo$ are $\True$.
  \end{itemize}

  \item \AllApplyCase{e\_getitem}
  \begin{itemize}
    \item $\veone$ is a tuple access expression with subexpression $\veoneone$ and position $\vione$,
          that is, $\EGetItem(\veoneone, \vione)$;
    \item $\vetwo$ is a tuple access expression with subexpression $\vetwoone$ and position $\vitwo$,
          that is, $\EGetItem(\vetwoone, \vitwo)$;
    \item $\vbone$ is $\True$ if and only if $\vione$ is equal to $\vitwo$;
    \item testing whether $\veoneone$ is equivalent to $\vetwoone$ yields $\vbtwo$\ProseOrTypeError;
    \item $\vb$ is $\True$ if and only both $\vbone$ and $\vbtwo$ are $\True$.
  \end{itemize}

  \item \AllApplyCase{e\_literal}
  \begin{itemize}
    \item $\veone$ is the literal expression with literal $\vvone$;
    \item $\vetwo$ is the literal expression with literal $\vvtwo$;
    \item $\vb$ is $\True$ if and only if $\vvone$ is equal to $\vvtwo$ in $\tenv$.
  \end{itemize}

  \item \AllApplyCase{e\_pattern}
  \begin{itemize}
    \item both $\veone$ and $\vetwo$ are pattern expressions;
    \item $\vb$ is $\False$.
  \end{itemize}

  \item \AllApplyCase{e\_record}
  \begin{itemize}
    \item both $\veone$ and $\vetwo$ are record expressions;
    \item $\vb$ is $\False$.
  \end{itemize}

  \item \AllApplyCase{e\_tuple}
  \begin{itemize}
    \item $\veone$ is a tuple expression with subexpression list $\vlone$,
          that is, $\ETuple(\vlone)$;
    \item $\vetwo$ is a tuple expression with subexpression list $\vltwo$,
          that is, $\ETuple(\vltwo)$;
    \item checking whether the lengths of $\vlone$ and $\vltwo$ are equal yields either $\True$ or $\False$, which short-circuits
          the entire rule;
    \item for each index $i$ in the list of indices for $\vlone$, testing whether $\vlone[i]$ is equivalent to $\vltwo[i]$
          in $\tenv$ yields $\vb_i$\ProseOrTypeError;
    \item $\vb$ is $\True$ if and only if $\vb_i$ is $\True$ for each index $i$ in the list of indices for $\vlone$.
  \end{itemize}

  \item \AllApplyCase{e\_array}
  \begin{itemize}
    \item $\veone$ is an array construction expression with length expression $\vlone$ and value expression $\vvone$,
          that is, $\EArray\{\EArrayLength: \vlone, \EArrayValue: \vvone\}$;
    \item $\vetwo$ is an array construction expression with length expression $\vltwo$ and value expression $\vvtwo$,
          that is, $\EArray\{\EArrayLength: \vltwo, \EArrayValue: \vvtwo\}$;
    \item applying $\exprequal$ to $\vlone$ and $\vltwo$ in $\tenv$ yields $\vbone$\ProseOrTypeError;
    \item applying $\exprequal$ to $\vvone$ and $\vvtwo$ in $\tenv$ yields $\vbone$\ProseOrTypeError;
    \item $\vb$ is $\True$ if and only if both $\vbone$ and $\vbtwo$ are $\True$.
  \end{itemize}

  \item \AllApplyCase{e\_unop}
  \begin{itemize}
    \item $\veone$ is a unary operator expression with operator $\opone$ and operand expressions $\veoneone$,
          that is, $\EUnop(\opone, \veoneone)$;
    \item $\vetwo$ is a unary operator expression with operator $\optwo$ and operand expressions $\vetwoone$,
          that is, $\EUnop(\optwo, \vetwoone)$;
    \item testing whether $\veoneone$ is equivalent to $\vetwoone$ in $\tenv$ yields $\vbone$;
    \item $\vb$ is $\True$ if and only if $\opone$ is equal to $\optwo$ and $\vbone$ is $\True$.
  \end{itemize}

  \item \AllApplyCase{e\_arbitrary}
  \begin{itemize}
    \item both $\veone$ and $\vetwo$ are $\ARBITRARY$ expressions;
    \item $\vb$ is $\False$.
  \end{itemize}

  \item \AllApplyCase{e\_atc}
  \begin{itemize}
    \item $\veone$ is a type assertion with subexpression with operator $\veoneone$ and type $\vtone$,
          that is, $\EATC(\veoneone, \vtone)$;
    \item $\vetwo$ is a type assertion with subexpression with operator $\vetwoone$ and type $\vttwo$,
          that is, $\EATC(\vetwoone, \vttwo)$;
    \item testing whether $\veoneone$ is equivalent to $\vetwoone$ in $\tenv$ yields $\vbone$;
    \item testing whether $\vtone$ is equivalent to $\vttwo$ in $\tenv$ yields $\vbtwo$;
    \item $\vb$ is $\True$ if and only if both $\vbone$ and $\vbtwo$ are $\True$.
  \end{itemize}

  \item \AllApplyCase{e\_var}
  \begin{itemize}
    \item $\veone$ is a variable expression with identifier $\nameone$, that is, $\EVar(\nameone)$;
    \item $\vetwo$ is a variable expression with identifier $\nametwo$, that is, $\EVar(\nametwo)$;
    \item $\vb$ is $\True$ if and only if both $\nameone$ is equal to $\nametwo$.
  \end{itemize}
\end{itemize}

\FormallyParagraph
\begin{mathpar}
\inferrule[different\_labels]{
  \astlabel(\veone) \neq \astlabel(\vetwo)
}{
  \exprequalcase(\tenv, \veone, \vetwo) \typearrow \False
}
\end{mathpar}

\begin{mathpar}
\inferrule[e\_binop]{
  \veone = \EBinop(\opone, \veoneone, \veonetwo)\\
  \vetwo = \EBinop(\optwo, \vetwoone, \vetwotwo)\\
  \exprequal(\veoneone, \vetwoone) \typearrow \vbone \OrTypeError\\\\
  \exprequal(\veonetwo, \vetwotwo) \typearrow \vbtwo \OrTypeError\\\\
  \vb \eqdef (\opone = \optwo) \land \vbone \land \vbtwo
}{
  \exprequalcase(\tenv, \veone, \vetwo) \typearrow \vb
}
\end{mathpar}

(Recall that a conjunction over an empty set equals $\True$.)
\begin{mathpar}
\inferrule[e\_call]{
  \veone = \ECall(\nameone, \vargsone, \Ignore)\\
  \vetwo = \ECall(\nametwo, \vargstwo, \Ignore)\\\\
  \booltrans{\nameone = \nametwo} \booltransarrow \True \terminateas \False\\\\
  \equallength(\vargsone, \vargstwo) \typearrow \True \terminateas \False\\\\
  i \in \listrange(\vargsone): \exprequal(\tenv, \vargsone[i], \vargstwo[i]) \typearrow \vb_i \OrTypeError\\\\
  \vb \eqdef \bigwedge_{i \in \listrange(\vargsone)} \vb_i
}{
  \exprequalcase(\tenv, \veone, \vetwo) \typearrow \vb
}
\end{mathpar}

\begin{mathpar}
\inferrule[e\_cond]{
  \veone = \ECond(\veoneone, \veonetwo, \veonethree)\\
  \vetwo = \ECond(\vetwoone, \vetwotwo, \vetwothree)\\\\
  \exprequal(\tenv, \veoneone, \vetwoone) \typearrow \vbone \OrTypeError\\\\
  \exprequal(\tenv, \veonetwo, \vetwotwo) \typearrow \vbtwo \OrTypeError\\\\
  \exprequal(\tenv, \veonethree, \vetwothree) \typearrow \vbthree \OrTypeError\\\\
  \vb \eqdef \vbone \land \vbtwo \land \vbthree
}{
  \exprequalcase(\tenv, \veone, \vetwo) \typearrow \True
}
\end{mathpar}

\begin{mathpar}
  \inferrule[e\_slice]{
  \veone = \ESlice(\veoneone, \slicesone)\\
  \vetwo = \ESlice(\vetwoone, \slicestwo)\\\\
  \exprequal(\tenv, \veoneone, \vetwoone) \typearrow \vbone \OrTypeError\\\\
  \slicesequal(\tenv, \slicesone, \slicestwo) \typearrow \vbtwo \OrTypeError\\\\
  \vb \eqdef \vbone \land \vbtwo
}{
  \exprequalcase(\tenv, \veone, \vetwo) \typearrow \vb
}
\end{mathpar}

\begin{mathpar}
  \inferrule[e\_getarray]{
  \veone = \EGetArray(\veoneone, \veonetwo)\\
  \vetwo = \EGetArray(\vetwoone, \vetwotwo)\\\\
  \exprequal(\tenv, \veoneone, \vetwoone) \typearrow \vbone \OrTypeError\\\\
  \exprequal(\tenv, \veonetwo, \vetwotwo) \typearrow \vbtwo \OrTypeError\\\\
  \vb \eqdef \vbone \land \vbtwo
}{
  \exprequalcase(\tenv, \veone, \vetwo) \typearrow \vb
}
\end{mathpar}

\begin{mathpar}
  \inferrule[e\_getfield]{
  \veone = \EGetField(\veoneone, \vfieldone)\\
  \vetwo = \EGetField(\vetwoone, \vfieldtwo)\\\\
  \vbone \eqdef \vfieldone = \vfieldtwo\\
  \exprequal(\tenv, \veoneone, \vetwoone) \typearrow \vbtwo \OrTypeError\\\\
  \vb \eqdef \vbone \land \vbtwo
}{
  \exprequalcase(\tenv, \veone, \vetwo) \typearrow \vb
}
\end{mathpar}

\begin{mathpar}
  \inferrule[e\_getfields]{
  \veone = \EGetFields(\veoneone, \vfieldsone)\\
  \vetwo = \EGetFields(\vetwoone, \vfieldstwo)\\\\
  \vbone \eqdef \vfieldsone = \vfieldstwo\\
  \exprequal(\tenv, \veoneone, \vetwoone) \typearrow \vbtwo \OrTypeError\\\\
  \vb \eqdef \vbone \land \vbtwo
}{
  \exprequalcase(\tenv, \veone, \vetwo) \typearrow \vb
}
\end{mathpar}

\begin{mathpar}
  \inferrule[e\_getitem]{
  \veone = \EGetItem(\veoneone, \vione)\\
  \vetwo = \EGetItem(\vetwoone, \vitwo)\\\\
  \vbone \eqdef \vione = \vitwo\\
  \exprequal(\tenv, \veoneone, \vetwoone) \typearrow \vbtwo \OrTypeError\\\\
  \vb \eqdef \vbone \land \vbtwo
}{
  \exprequalcase(\tenv, \veone, \vetwo) \typearrow \vb
}
\end{mathpar}

\begin{mathpar}
  \inferrule[e\_literal]{
  \veone = \ELiteral(\vvone)\\
  \vetwo = \ELiteral(\vvtwo)\\\\
  \vb \eqdef \vvone = \vvtwo
}{
  \exprequalcase(\tenv, \veone, \vetwo) \typearrow \vb
}
\end{mathpar}

\begin{mathpar}
\inferrule[e\_pattern]{
  \astlabel(\veone) = \EPattern \land \astlabel(\vetwo) = \EPattern
}{
  \exprequalcase(\tenv, \veone, \vetwo) \typearrow \False
}
\end{mathpar}

\begin{mathpar}
\inferrule[e\_record]{
  \astlabel(\veone) = \ERecord \land \astlabel(\vetwo) = \ERecord
}{
  \exprequalcase(\tenv, \veone, \vetwo) \typearrow \False
}
\end{mathpar}

\begin{mathpar}
\inferrule[e\_tuple]{
  \veone = \ETuple(\vlone)\\
  \vetwo = \ETuple(\vltwo)\\
  \equallength(\vlone, \vltwo) \typearrow \True \terminateas \False\\\\
  i \in \listrange(\vlone): \exprequal(\tenv, \vlone[i], \vltwo[i]) \typearrow \vb_i \OrTypeError\\\\
  \vb \eqdef \bigwedge_{i \in \listrange(\vlone)} \vb_i
}{
  \exprequalcase(\tenv, \veone, \vetwo) \typearrow \vb
}
\end{mathpar}

\begin{mathpar}
\inferrule[e\_array]{
  \veone = \EArray\{\EArrayLength: \vlone, \EArrayValue: \vvone\}\\
  \vetwo = \EArray\{\EArrayLength: \vltwo, \EArrayValue: \vvtwo\}\\
  \exprequal(\tenv, \vlone, \vltwo) \typearrow \vbone \OrTypeError\\\\
  \exprequal(\tenv, \vvone, \vvtwo) \typearrow \vbone \OrTypeError\\\\
  \vb \eqdef \vbone \land \vbtwo
}{
  \exprequalcase(\tenv, \veone, \vetwo) \typearrow \vb
}
\end{mathpar}

\begin{mathpar}
\inferrule[e\_unop]{
  \veone = \EUnop(\opone, \veoneone)\\
  \vetwo = \EUnop(\optwo, \vetwoone)\\\\
  \exprequal(\tenv, \veoneone, \vetwoone) \typearrow \vbone \OrTypeError\\\\
  \vb \eqdef (\opone = \optwo) \land \vbone
}{
  \exprequalcase(\tenv, \veone, \vetwo) \typearrow \vb
}
\end{mathpar}

\begin{mathpar}
\inferrule[e\_arbitrary]{
  (\astlabel(\veone) = \EArbitrary \land \astlabel(\vetwo) = \EArbitrary)
}{
  \exprequalcase(\tenv, \veone, \vetwo) \typearrow \False
}
\end{mathpar}

\begin{mathpar}
\inferrule[e\_atc]{
  \veone = \EATC(\veoneone, \vtone)\\
  \vetwo = \EATC(\vetwoone, \vttwo)\\
  \exprequal(\tenv, \veoneone, \vetwoone) \typearrow \vbone \OrTypeError\\\\
  \typeequal(\tenv, \vtone, \vttwo) \typearrow \vbtwo \OrTypeError\\\\
  \vb \eqdef \vbone \land \vbtwo
}{
  \exprequalcase(\tenv, \veone, \vetwo) \typearrow \vb
}
\end{mathpar}

\begin{mathpar}
\inferrule[e\_var]{
  \veone = \EVar(\nameone)\\
  \vetwo = \EVar(\nametwo)\\\\
  \vb \eqdef \nameone = \nametwo
}{
  \exprequalcase(\tenv, \veone, \vetwo) \typearrow \vb
}
\end{mathpar}

\TypingRuleDef{TypeEqual}
\hypertarget{def-typeequal}{}
The function
\[
  \typeequal(\overname{\ty}{\vtone} \aslsep \overname{\ty}{\vttwo}) \aslto
   \overname{\Bool}{\vb} \cup \overname{\TTypeError}{\TypeErrorConfig}
\]
conservatively tests whether the type $\vtone$ is equivalent to the type $\vttwo$ in environment $\tenv$
and yields the result in $\vb$.  \ProseOtherwiseTypeError

\ProseParagraph
\OneApplies
\begin{itemize}
  \item \AllApplyCase{different\_labels}
  \begin{itemize}
    \item the AST labels of $\vtone$ and $\vttwo$ are different;
    \item $\vb$ is $\False$.
  \end{itemize}

  \item \AllApplyCase{tbool\_treal\_tstring}
  \begin{itemize}
    \item both $\vtone$ and $\vttwo$ are both either $\TBool$, $\TReal$, or $\TString$;
    \item $\vb$ is $\True$.
  \end{itemize}

  \item \AllApplyCase{tint\_unconstrained}
  \begin{itemize}
    \item both $\vtone$ and $\vttwo$ are the unconstrained integer type $\unconstrainedinteger$;
    \item $\vb$ is $\True$.
  \end{itemize}

  \item \AllApplyCase{tint\_parameterized}
  \begin{itemize}
    \item $\vtone$ is the \parameterizedintegertype\  with identifier $\vione$, that is, \\ $\TInt(\parameterized(\vione))$;
    \item $\vttwo$ is the \parameterizedintegertype\ with identifier $\vitwo$, that is, \\ $\TInt(\parameterized(\vitwo))$;
    \item $\vb$ is $\True$ if and only if $\vione$ is equal to $\vitwo$.
  \end{itemize}

  \item \AllApplyCase{tint\_wellconstrained}
  \begin{itemize}
    \item $\vtone$ is the well-constrained integer type with list of constraints $\vcone$, that is, \\ $\TInt(\wellconstrained(\vcone))$;
    \item $\vttwo$ is the well-constrained integer type with list of constraints $\vctwo$, that is, \\ $\TInt(\wellconstrained(\vctwo))$;
    \item testing whether $\vcone$ and $\vctwo$ are equivalent in $\tenv$ yields $\vb$\ProseOrTypeError.
  \end{itemize}

  \item \AllApplyCase{tbits}
  \begin{itemize}
    \item $\vtone$ is the bitvector type with width expression $\vwone$ and list of bitfields $\bfone$, that is, $\TBits(\vwone, \bfone)$;
    \item $\vttwo$ is the bitvector type with width expression $\vwtwo$ and list of bitfields $\bftwo$, that is, $\TBits(\vwtwo, \bftwo)$;
    \item testing whether $\vwone$ and $\vwtwo$ are equivalent bitwidths in $\tenv$ yields $\vbone$\ProseOrTypeError;
    \item testing whether $\bfone$ and $\bftwo$ are equivalent lists of bitfields in $\tenv$ yields $\vbtwo$\ProseOrTypeError;
    \item $\vb$ is $\True$ if and only if both $\vbone$ and $\vbtwo$ are $\True$.
  \end{itemize}

  \item \AllApplyCase{tarray}
  \begin{itemize}
    \item $\vtone$ is an array type with index $\vlone$ and element type $\vtone$, that is, $\TArray(\vlone, \vtone)$;
    \item $\vttwo$ is an array type with index $\vltwo$ and element type $\vttwo$, that is, $\TArray(\vltwo, \vttwo)$;
    \item testing whether $\vlone$ is equivalent to $\vltwo$ in $\tenv$ yields $\vbone$\ProseOrTypeError;
    \item testing whether $\vtone$ is equivalent to $\vttwo$ in $\tenv$ yields $\vbtwo$\ProseOrTypeError;
    \item $\vb$ is $\True$ if and only if both $\vbone$ and $\vbtwo$ are $\True$.
  \end{itemize}

  \item \AllApplyCase{tnamed}
  \begin{itemize}
    \item $\vtone$ is a named type with identifier $\vsone$, that is $\TNamed(\vsone)$;
    \item $\vttwo$ is a named type with identifier $\vstwo$, that is $\TNamed(\vstwo)$;
    \item $\vb$ is $\True$ if and only if $\vsone$ is equal to $\vstwo$.
  \end{itemize}

  \item \AllApplyCase{tenum}
  \begin{itemize}
    \item $\vtone$ is an \enumerationtypeterm{} with identifier $\vlone$, that is $\TEnum(\vlone)$;
    \item $\vttwo$ is an \enumerationtypeterm{} with identifier $\vltwo$, that is $\TEnum(\vltwo)$;
    \item $\vb$ is $\True$ if and only if $\vlone$ is equal to $\vltwo$.
  \end{itemize}

  \item \AllApplyCase{tstructured}
  \begin{itemize}
    \item $L$ is either $\TRecord$ or $\TException$;
    \item $\vtone$ is a \structuredtype\ with list of fields $\vfieldsone$, that is $L(\vfieldsone)$;
    \item $\vttwo$ is a \structuredtype\ with list of fields $\vfieldstwo$, that is $L(\vfieldstwo)$;
    \item checking whether the set of field names in $\vfieldsone$ is equal to the set of field names in $\vfieldstwo$
          yields $\True$ or $\False$, which short-circuits the entire rule;
    \item for each field $\vf$ in the set of fields of $\vfieldsone$, testing whether the type associated with
          $\vf$ in $\vfieldsone$ is equivalent to the type associated with
          $\vf$ in $\vfieldstwo$ in $\tenv$ yields $\vb_\vf$\ProseOrTypeError;
    \item $\vb$ is $\True$ if and only if $\vb_\vf$ is $\True$ for each field $\vf$ in the set of fields of $\vfieldsone$.
  \end{itemize}

  \item \AllApplyCase{ttuple}
  \begin{itemize}
    \item $\vtone$ is a \tupletypeterm{} with list of types $\vtsone$, that is $\TTuple(\vtsone)$;
    \item $\vttwo$ is a \tupletypeterm{} with list of types $\vtstwo$, that is $\TTuple(\vtstwo)$;
    \item checking whether the list of types $\vtsone$ has the same length as the list of types $\vtstwo$ yields $\True$
          or $\False$, which short-circuits the entire rule;
    \item for each index $i$ in the list $\vtsone$, testing whether $\vtsone[i]$ is equivalent to $\vtstwo[i]$ in $\tenv$
          yields $\vb_i$\ProseOrTypeError;
    \item $\vb$ is $\True$ if and only if $\vb_i$ is $\True$ for each index $i$ in the list $\vtsone$.
  \end{itemize}
\end{itemize}

\FormallyParagraph
\begin{mathpar}
\inferrule[different\_labels]{
  \astlabel(\vtone) \neq \astlabel(\vttwo)
}{
  \typeequal(\tenv, \vtone, \vttwo) \typearrow \False
}
\end{mathpar}

\begin{mathpar}
\inferrule[tbool\_treal\_tstring]{
  \astlabel(\vtone) = \astlabel(\vttwo)\\
  \astlabel(\vtone) \in \{\TBool, \TReal, \TString\}
}{
  \typeequal(\tenv, \vtone, \vttwo) \typearrow \True
}
\and
\end{mathpar}

\begin{mathpar}
\inferrule[tint\_unconstrained]{}
{
  \typeequal(\tenv, \unconstrainedinteger, \unconstrainedinteger) \typearrow \True
}
\and
\inferrule[tint\_parameterized]{
  \vb \eqdef \vione = \vitwo
}{
  \typeequal(\tenv, \TInt(\parameterized(\vione)), \TInt(\parameterized(\vitwo))) \typearrow \vb
}
\and
\inferrule[tint\_wellconstrained]{
  \constraintsequal(\tenv, \vcone, \vctwo) \typearrow \vb \OrTypeError
}{
  \typeequal(\tenv, \TInt(\wellconstrained(\vcone)), \TInt(\wellconstrained(\vctwo))) \typearrow \vb
}
\end{mathpar}

\begin{mathpar}
\inferrule[tbits]{
  \bitwidthequal(\tenv, \vwone, \vwtwo) \typearrow \vbone \OrTypeError\\\\
  \bitfieldsequal(\tenv, \bfone, \bftwo) \typearrow \vbtwo \OrTypeError\\\\
  \vb \eqdef \vbone \land \vbtwo
}{
  \typeequal(\tenv, \TBits(\vwone, \bfone), \TBits(\vwtwo, \bftwo)) \typearrow \vb
}
\end{mathpar}

\begin{mathpar}
\inferrule[tarray]{
  \exprequal(\tenv, \vlone, \vltwo) \typearrow \vbone \OrTypeError\\\\
  \typeequal(\tenv, \vtone, \vttwo) \typearrow \vbtwo \OrTypeError\\\\
  \vb \eqdef \vbone \land \vbtwo
}{
  \typeequal(\tenv, \TArray(\vlone, \vtone), \TArray(\vltwo, \vttwo)) \typearrow \vb
}
\end{mathpar}

\begin{mathpar}
\inferrule[tnamed]{
  \vb \eqdef \vsone = \vstwo
}{
  \typeequal(\tenv, \TNamed(\vsone), \TNamed(\vstwo)) \typearrow \vb
}
\end{mathpar}

\begin{mathpar}
\inferrule[tenum]{
  \vb \eqdef \vlone = \vltwo
}{
  \typeequal(\tenv, \TEnum(\vlone), \TEnum(\vltwo)) \typearrow \vb
}
\end{mathpar}

\begin{mathpar}
\inferrule[tstructured]{
  L \in \{\TRecord, \TException\}\\
  \booltrans{\fieldnames(\vfieldsone) = \fieldnames(\vfieldstwo)} \booltransarrow \True \terminateas \False\\\\
  {
    \begin{array}{l}
  \vf \in \fieldnames(\vfieldsone): \\ \typeequal(\tenv, \fieldtype(\vfieldsone, \vf), \fieldtype(\vfieldstwo, \vf)) \typearrow \vb_\vf \OrTypeError
    \end{array}
  }\\\\
  \vb \eqdef \bigwedge_{\vf \in \fieldnames(\vfieldsone)} \vb_\vf
}{
  \typeequal(\tenv, L(\vfieldsone), L(\vfieldstwo)) \typearrow \vb
}
\end{mathpar}

\begin{mathpar}
\inferrule[ttuple]{
  \equallength(\vtsone, \vtstwo) \typearrow \True \terminateas \False\\
  i \in \listrange(\vtsone): \typeequal(\tenv, \vtsone[i], \vtstwo[i]) \typearrow \vb_i \OrTypeError\\\\
  \vb \eqdef \bigwedge_{i \in \listrange(\vtsone)} \vb_i
}{
  \typeequal(\tenv, \TTuple(\vtsone), \TTuple(\vtstwo)) \typearrow \vb
}
\end{mathpar}

\TypingRuleDef{BitwidthEqual}
\hypertarget{def-bitwidthequal}{}
The function
\[
  \bitwidthequal(\overname{\staticenvs}{\tenv} \aslsep \overname{\expr}{\vwone} \aslsep \overname{\expr}{\vwtwo})
  \aslto \overname{\Bool}{\vb} \cup \overname{\TTypeError}{\TypeErrorConfig}
\]
conservatively tests whether the bitwidth expression $\vwone$ is equivalent to the bitwidth expression $\vwtwo$
in environment $\tenv$ and yields the result in $\vb$.  \ProseOtherwiseTypeError

\ExampleDef{Equivalence of Bitwidth Expressions}
\listingref{BitwidthEqual} shows examples of equivalent bitwidth expressions on both sides
of the respective local declaration statements.
\ASLListing{Equivalence of bitwidth expressions}{BitwidthEqual}{\typingtests/TypingRule.BitwidthEqual.asl}

\ProseParagraph
Testing whether the expressions $\vwone$ and $\vwtwo$ are equivalent in $\tenv$ yields $\vb$\ProseOrTypeError.

\FormallyParagraph
\begin{mathpar}
\inferrule{
  \exprequal(\tenv, \vwone, \vwtwo) \typearrow \vb \OrTypeError
}{
  \bitwidthequal(\tenv, \vwone, \vwtwo) \typearrow \vb
}
\end{mathpar}

\TypingRuleDef{BitFieldsEqual}
\hypertarget{def-bitfieldsequal}{}
The function
\[
  \bitfieldsequal(\overname{\staticenvs}{\tenv} \aslsep \overname{\bitfield^*}{\bfone} \aslsep \overname{\bitfield^*}{\bftwo})
  \aslto \overname{\Bool}{\vb} \cup \overname{\TTypeError}{\TypeErrorConfig}
\]
conservatively tests whether the list of bitfields $\bfone$ is equivalent to the list of bitfields $\bftwo$
in environment $\tenv$ and yields the result in $\vb$.  \ProseOtherwiseTypeError

\ProseParagraph
\OneApplies
\begin{itemize}
  \item \AllApplyCase{different\_lengths}
  \begin{itemize}
    \item the number of bitfields in $\bfone$ is different from the number of bitfields in $\bftwo$;
    \item $\vb$ is $\False$.
  \end{itemize}

  \item \AllApplyCase{same\_lengths}
  \begin{itemize}
    \item the number of bitfields in $\bfone$ is the same as the number of bitfields in $\bftwo$;
    \item testing whether the bitfield $\bfone[i]$ is equivalent to $\bftwo[i]$ in $\tenv$ for every index
          of $\bfone$ yields $\vb_i$\ProseOrTypeError;
    \item $\vb$ is $\True$ if and only if $\vb_i$ is $\True$ for every index of $\bfone$.
  \end{itemize}
\end{itemize}

\FormallyParagraph
\begin{mathpar}
\inferrule[different\_lengths]{
  \equallength(\bfone, \bftwo) \typearrow \False
}{
  \bitfieldsequal(\tenv, \bfone, \bftwo) \typearrow \False
}
\and
\inferrule[same\_lengths]{
  \equallength(\bfone, \bftwo) \typearrow \True\\
  i\in\listrange(\bfone): \bitfieldequal(\tenv, \bfone[i], \bftwo[i]) \typearrow \vb_i\\\\
  \vb \eqdef \bigwedge_{i\in\listrange(\bfone)} \vb_i
}{
  \bitfieldsequal(\tenv, \bfone, \bftwo) \typearrow \vb
}
\end{mathpar}

\TypingRuleDef{BitFieldEqual}
\hypertarget{def-bitfieldequal}{}
The function
\[
  \bitfieldequal(\overname{\staticenvs}{\tenv} \aslsep \overname{\bitfield}{\bfone} \aslsep \overname{\bitfield}{\bftwo})
  \aslto \overname{\Bool}{\vb} \cup \overname{\TTypeError}{\TypeErrorConfig}
\]
conservatively tests whether the bitfield $\bfone$ is equivalent to the bitfield $\bftwo$ in environment $\tenv$
and yields the result in $\vb$.  \ProseOtherwiseTypeError

\ProseParagraph
\OneApplies
\begin{itemize}
  \item \AllApplyCase{different\_labels}
  \begin{itemize}
    \item the AST labels of $\bfone$ and $\bftwo$ are different;
    \item $\vb$ is $\False$.
  \end{itemize}

  \item \AllApplyCase{bitfield\_simple}
  \begin{itemize}
    \item $\bfone$ is a simple bitfield with name $\nameone$ and list of slices $\slicesone$, that is, \\ $\BitFieldSimple(\nameone, \slicesone)$;
    \item $\bftwo$ is a simple bitfield with name $\nametwo$ and list of slices $\slicestwo$, that is, \\ $\BitFieldSimple(\nametwo, \slicestwo)$;
    \item checking whether $\nameone$ is equal to $\nametwo$ yields $\vbone$;
    \item testing whether $\slicesone$ and $\slicestwo$ are equivalent in $\tenv$ yields $\vbtwo$\ProseOrTypeError;
    \item $\vb$ is $\True$ if and only if both $\vbone$ and $\vbtwo$ are $\True$.
  \end{itemize}

  \item \AllApplyCase{bitfield\_nested}
  \begin{itemize}
    \item $\bfone$ is a nested bitfield with name $\nameone$, list of slices $\slicesone$, and nested bitfields $\bfoneone$, that is,
          $\BitFieldNested(\nameone, \slicesone, \bfoneone)$;
    \item $\bftwo$ is a nested bitfield with name $\nametwo$, list of slices $\slicestwo$, and nested bitfields $\bftwoone$, that is,
          $\BitFieldNested(\nametwo, \slicestwo, \bftwoone)$;
    \item checking whether $\nameone$ is equal to $\nametwo$ yields $\vbone$;
    \item testing whether $\slicesone$ and $\slicestwo$ are equivalent in $\tenv$ yields $\vbtwo$\ProseOrTypeError;
    \item testing whether the bitfields $\bfoneone$ and $\bftwoone$ are equivalent in $\tenv$ yields $\vbtwo$\ProseOrTypeError;
    \item $\vb$ is $\True$ if and only if both $\vbone$ and $\vbtwo$ are $\True$.
  \end{itemize}

  \item \AllApplyCase{bitfield\_typed}
  \begin{itemize}
    \item $\bfone$ is a typed bitfield with name $\nameone$, list of slices $\slicesone$, and type $\vtone$, that is,
          $\BitFieldType(\nameone, \slicesone, \vtone)$;
    \item $\bftwo$ is a typed bitfield with name $\nametwo$, list of slices $\slicestwo$, and type $\vttwo$, that is,
          $\BitFieldType(\nametwo, \slicestwo, \vttwo)$;
    \item checking whether $\nameone$ is equal to $\nametwo$ yields $\True$\ProseTerminateAs{\False};
    \item testing whether $\slicesone$ and $\slicestwo$ are equivalent in $\tenv$ yields $\vbone$\ProseOrTypeError;
    \item testing whether the types $\vtone$ and $\vttwo$ are equivalent in $\tenv$ yields $\vbtwo$\ProseOrTypeError;
    \item $\vb$ is $\True$ if and only if both $\vbone$ and $\vbtwo$ are $\True$.
  \end{itemize}
\end{itemize}

\FormallyParagraph
\begin{mathpar}
\inferrule[different\_labels]{
  \astlabel(\bfone) \neq \astlabel(\bftwo)
}{
  \bitfieldequal(\tenv, \bfone, \bftwo) \typearrow \False
}
\and
\inferrule[bitfield\_simple]{
  \bfone = \BitFieldSimple(\nameone, \slicesone)\\
  \bftwo = \BitFieldSimple(\nametwo, \slicestwo)\\
  \booltrans{\nameone = \nametwo} \booltransarrow \vbone\\
  \slicesequal(\tenv, \slicesone, \slicestwo) \typearrow \vbtwo \OrTypeError\\\\
  \vb \eqdef \vbone \land \vbtwo
}{
  \bitfieldequal(\tenv, \bfone, \bftwo) \typearrow \vb
}
\and
\inferrule[bitfield\_nested]{
  \bfone = \BitFieldNested(\nameone, \slicesone, \bfoneone)\\
  \bftwo = \BitFieldNested(\nametwo, \slicestwo, \bftwoone)\\
  \booltrans{\nameone = \nametwo} \booltransarrow \True \terminateas \False\\\\
  \slicesequal(\tenv, \slicesone, \slicestwo) \typearrow \vbone \terminateas \TypeErrorConfig,\\\\
  \bitfieldsequal(\tenv, \bfoneone, \bftwoone) \typearrow \vbtwo
}{
  \bitfieldequal(\tenv, \bfone, \bftwo) \typearrow \vb
}
\and
\inferrule[bitfield\_typed]{
  \bfone = \BitFieldType(\nameone, \slicesone, \vtone)\\
  \bftwo = \BitFieldType(\nametwo, \slicestwo, \vttwo)\\
  \booltrans{\nameone = \nametwo} \booltransarrow \True \terminateas \False \\\\
  \slicesequal(\tenv, \slicesone, \slicestwo) \typearrow \vbone \OrTypeError\\\\
  \typeequal(\tenv, \vtone, \vttwo) \typearrow \vbtwo \OrTypeError\\\\
  \vb \eqdef \vbone \land \vbtwo
}{
  \bitfieldequal(\tenv, \bfone, \bftwo) \typearrow \vb
}
\end{mathpar}

\TypingRuleDef{ConstraintsEqual}
\hypertarget{def-constraintsequal}{}
The function
\[
  \constraintsequal(\overname{\staticenvs}{\tenv} \aslsep \overname{\intconstraint^*}{\csone} \aslsep \overname{\intconstraint^*}{\cstwo})
  \aslto \overname{\Bool}{\vb} \cup \overname{\TTypeError}{\TypeErrorConfig}
\]
conservatively tests whether the constraint list $\csone$ is equivalent to the constraint list $\cstwo$ in environment $\tenv$
and yields the result in $\vb$.  \ProseOtherwiseTypeError

\ProseParagraph
\AllApply
\begin{itemize}
  \item checking whether the number of constraints in $\csone$ is the same as the number of constraints in $\cstwo$
        yields $\True$\ProseTerminateAs{\False};
  \item testing whether the constraint $\csone[i]$ is equivalent to the constraint $\cstwo[i]$ in $\tenv$
        yields $\vb_i$ for each index in the index in the indices for $\csone$ ($i\in\listrange(\csone)$)\ProseOrTypeError;
  \item $\vb$ is $\True$ if and only if all $\vb_i$ are $\True$ for each index in the indices for $\csone$.
\end{itemize}

\FormallyParagraph
\begin{mathpar}
\inferrule{
  \equallength(\csone, \cstwo) \typearrow \True \terminateas \False\\
  i\in\listrange(\csone): \constraintequal(\tenv, \csone[i], \cstwo[i]) \typearrow \vb_i \OrTypeError\\\\
  \vb \eqdef \bigwedge_{i\in\listrange(\csone)} \vb_i
}{
  \constraintsequal(\tenv, \csone, \cstwo) \typearrow \vb
}
\end{mathpar}

\TypingRuleDef{ConstraintEqual}
\hypertarget{def-constraintequal}{}
The function
\[
  \constraintequal(\overname{\staticenvs}{\tenv} \aslsep \overname{\intconstraint}{\vcone} \aslsep \overname{\intconstraint}{\vstwo})
  \aslto \overname{\Bool}{\vb} \cup \overname{\TTypeError}{\TypeErrorConfig}
\]
conservatively tests whether the constraint $\vcone$ is equivalent to the constraint $\vctwo$ in environment $\tenv$
and yields the result in $\vb$.  \ProseOtherwiseTypeError

\ProseParagraph
\OneApplies
\begin{itemize}
  \item \AllApplyCase{different\_labels}
  \begin{itemize}
    \item the AST labels of $\vcone$ and $\vctwo$ are different;
    \item define $\vb$ as $\False$.
  \end{itemize}

  \item \AllApplyCase{constraint\_exact}
  \begin{itemize}
    \item $\vcone$ is an exact constraint with subexpression $\veone$, that is, $\ConstraintExact(\veone)$;
    \item $\vctwo$ is an exact constraint with subexpression $\vetwo$, that is, $\ConstraintExact(\vetwo)$;
    \item applying $\exprequal$ to $\veone$ and $\vetwo$ yields $\vb$\ProseOrTypeError.
  \end{itemize}

  \item \AllApplyCase{constraint\_range}
  \begin{itemize}
    \item $\vcone$ is a range constraint with subexpressions $\veoneone$ and $\veonetwo$, that is, \\ $\ConstraintRange(\veoneone, \veonetwo)$;
    \item $\vctwo$ is a range constraint with subexpressions $\vetwoone$ and $\vetwotwo$, that is, \\ $\ConstraintRange(\vetwoone, \vetwotwo)$;
    \item applying $\exprequal$ to $\veoneone$ and $\vetwoone$ yields $\vbone$\ProseOrTypeError;
    \item applying $\exprequal$ to $\veonetwo$ and $\vetwotwo$ yields $\vbtwo$\ProseOrTypeError;
    \item define $\vb$ as $\True$ if and only if both $\vbone$ and $\vbtwo$ are $\True$.
  \end{itemize}
\end{itemize}

\FormallyParagraph
\begin{mathpar}
\inferrule[different\_labels]{
  \astlabel(\vcone) \neq \astlabel(\vctwo)
}{
  \constraintequal(\tenv, \vcone, \vctwo) \typearrow \False
}
\and
\inferrule[constraint\_exact]{
  \vcone = \ConstraintExact(\veone)\\
  \vctwo = \ConstraintExact(\vetwo)\\
  \exprequal(\tenv, \veone, \vetwo) \typearrow \vb \OrTypeError
}{
  \constraintequal(\tenv, \vcone, \vctwo) \typearrow \vb
}
\and
\inferrule[constraint\_range]{
  \bfone = \ConstraintRange(\veoneone, \veonetwo)\\
  \bftwo = \ConstraintRange(\vetwoone, \vetwotwo)\\
  \exprequal(\tenv, \veoneone, \vetwoone) \typearrow \vbone \OrTypeError\\\\
  \exprequal(\tenv, \veonetwo, \vetwotwo) \typearrow \vbtwo \OrTypeError\\\\
  \vb \eqdef \vbone \land \vbtwo
}{
  \constraintequal(\tenv, \bfone, \bftwo) \typearrow \vb
}
\end{mathpar}

\TypingRuleDef{SlicesEqual}
\hypertarget{def-slicesequal}{}
The function
\[
  \slicesequal(\overname{\staticenvs}{\tenv} \aslsep \overname{\slice^*}{\slicesone} \aslsep \overname{\slice^*}{\slicestwo})
  \aslto \overname{\Bool}{\vb} \cup \overname{\TTypeError}{\TypeErrorConfig}
\]
conservatively tests whether the list of slices $\slicesone$ is equivalent to the list of slices $\slicestwo$
in environment $\tenv$ and yields the result in $\vb$.  \ProseOtherwiseTypeError

\ExampleDef{Equivalent Lists of Slices}
\listingref{SlicesEqual} shows an example of two equivalent lists of slices.
\ASLListing{Equivalent lists of slices}{SlicesEqual}{\typingtests/TypingRule.SlicesEqual.asl}

\ProseParagraph
\OneApplies
\begin{itemize}
  \item \AllApplyCase{different\_lengths}
  \begin{itemize}
    \item checking whether the number of slices in $\slicesone$ is equal to the number of slice in $\slicestwo$ yields $\False$;
    \item $\vb$ is $\False$.
  \end{itemize}

  \item \AllApplyCase{same\_lengths}
  \begin{itemize}
    \item checking whether the number of slices in $\slicesone$ is equal to the number of slice in $\slicestwo$ yields $\True$;
    \item determining whether the expression $\slicesone[i]$ is equivalent to $\slicestwo[i]$ in $\tenv$
          for each index in the indices for $\slicesone$ ($i \in \listrange(\slicesone$)), yields $\vb_i$\ProseOrTypeError;
    \item $\vb$ is $\True$ if and only if all $\vb_i$ are $\True$ for each index in the indices for $\slicesone$.
  \end{itemize}
\end{itemize}

\FormallyParagraph
\begin{mathpar}
\inferrule[different\_lengths]{
  \equallength(\slicesone, \slicestwo) \typearrow \False
}{
  \slicesequal(\tenv, \slicesone, \slicestwo) \typearrow \False
}
\and
\inferrule[same\_lengths]{
  \equallength(\slicesone, \slicestwo) \typearrow \True\\
  i\in\listrange(\slicesone): \sliceequal(\tenv, \slicesone[i], \slicestwo[i]) \typearrow \vb_i \OrTypeError\\\\
  \vb \eqdef \bigwedge_{i\in\listrange(\slicesone)} \vb_i
}{
  \slicesequal(\tenv, \slicesone, \slicestwo) \typearrow \vb
}
\end{mathpar}

\TypingRuleDef{SliceEqual}
\hypertarget{def-sliceequal}{}
The function
\[
  \sliceequal(\overname{\staticenvs}{\tenv} \aslsep \overname{\slice}{\sliceone} \aslsep \overname{\slice}{\slicetwo})
  \aslto \overname{\Bool}{\vb} \cup \overname{\TTypeError}{\TypeErrorConfig}
\]
conservatively tests whether the slice $\sliceone$ is equivalent to the slice $\slicetwo$
in environment $\tenv$ and yields the result in $\vb$. \ProseOtherwiseTypeError

This function operates both on \typedast{} nodes, where both slices are only
of the form $\SliceLength(\ \cdot\ ,\ \cdot\ )$, as well as \untypedast{} nodes where all slice forms
are possible. The latter is used during checking overriding subprograms (\secref{Overriding}).

\ExampleDef{Equivalent Slices}
The specification in \listingref{SliceEqual} shows examples of equivalent slices.
\ASLListing{Equivalent slices}{SliceEqual}{\typingtests/TypingRule.SliceEqual.asl}

In \listingref{overriding}, the slice for the bitfield \verb|lsb| in both the \texttt{impdef} version
of \verb|Foo| and the overridden version of \verb|Foo| are equivalent.

\ProseParagraph
\OneApplies
\begin{itemize}
  \item \AllApplyCase{single\_expr}
  \begin{itemize}
    \item $\sliceone$ is a slice for the single expression $\veone$, that is, $\SliceSingle(\veone)$;
    \item $\slicetwo$ is a slice for the single expression $\vetwo$, that is, $\SliceSingle(\vetwo)$;
    \item testing $\veone$ and $\vetwo$ for equivalence yields $\vb$\ProseOrTypeError;
  \end{itemize}

  \item \AllApplyCase{two\_exprs}
  \begin{itemize}
    \item $\sliceone$ is a slice consisting of the configuration domain $\vLone$ and two expressions $\veoneone$ and $\vetwoone$, that is, $\vLone(\veoneone, \vetwoone)$;
    \item $\slicetwo$ is a slice consisting of the configuration domain $\vLtwo$ and two expressions $\veonetwo$ and $\vetwotwo$, that is, $\vLtwo(\veonetwo, \vetwotwo)$;
    \item $\vLone$ is equal to $\vLtwo$;
    \item testing $\veoneone$ and $\vetwoone$ for equivalence yields $\vbone$\ProseOrTypeError;
    \item testing $\veonetwo$ and $\vetwotwo$ for equivalence yields $\vbtwo$\ProseOrTypeError;
    \item \Proseeqdef{$\vb$}{$\True$ if and only if both $\vbone$ and $\vbtwo$ are $\True$}.
  \end{itemize}

  \item \AllApplyCase{different\_labels}
  \begin{itemize}
    \item $\sliceone$ is a slice consisting of the configuration domain $\vLone$;
    \item $\slicetwo$ is a slice consisting of the configuration domain $\vLtwo$;
    \item $\vLone$ is different to $\vLtwo$;
    \item \Proseeqdef{$\vb$}{$\False$}.
  \end{itemize}
\end{itemize}

\FormallyParagraph
\begin{mathpar}
\inferrule[single\_expr]{
  \exprequal(\tenv, \veone, \vetwo) \typearrow \vb \OrTypeError
}{
  \slicesequal(\tenv, \overname{\SliceSingle(\veone)}{\sliceone}, \overname{\SliceSingle(\vetwo)}{\slicetwo}) \typearrow \vb
}
\end{mathpar}

\begin{mathpar}
\inferrule[two\_exprs]{
  \vLone, \vLtwo \in \{\SliceLength, \SliceRange, \SliceStar\}\\
  \vLone = \vLtwo\\
  \exprequal(\tenv, \veoneone, \veonetwo) \typearrow \vbone \OrTypeError\\\\
  \exprequal(\tenv, \vetwoone, \vetwotwo) \typearrow \vbtwo \OrTypeError\\\\
  \vb \eqdef \vbone \land \vbtwo
}{
  \slicesequal(\tenv, \overname{\vLone(\veoneone, \vetwoone)}{\sliceone}, \overname{\vLtwo(\veonetwo, \vetwotwo)}{\slicetwo}) \typearrow \vb
}
\end{mathpar}

\begin{mathpar}
\inferrule[different\_labels]{
  \configdomain{\sliceone} \neq \configdomain{\slicetwo}
}{
  \slicesequal(\tenv, \sliceone, \slicetwo) \typearrow \overname{\False}{\vb}
}
\end{mathpar}

\TypingRuleDef{ArrayLengthEqual}
\hypertarget{def-arraylengthequal}{}
The function
\[
  \arraylengthequal(\overname{\arrayindex}{\vlone} \aslsep \overname{\arrayindex}{\vltwo})
  \aslto \overname{\Bool}{\vb} \cup \overname{\TTypeError}{\TypeErrorConfig}
\]
tests whether the array lengths $\vlone$ and $\vltwo$ are equivalent and yields the result
in $\vb$. \ProseOtherwiseTypeError

\ProseParagraph
\OneApplies
\begin{itemize}
  \item \AllApplyCase{different\_labels}
  \begin{itemize}
    \item $\vlone$ and $\vltwo$ have different AST labels;
    \item $\vb$ is $\False$.
  \end{itemize}

  \item \AllApplyCase{expr\_expr}
  \begin{itemize}
    \item $\vlone$ is an integer type length expression with subexpression $\veoneone$, that is, \\ $\ArrayLengthExpr(\veoneone)$;
    \item $\vltwo$ is an integer type length expression with subexpression $\vetwoone$, that is, \\ $\ArrayLengthExpr(\vetwoone)$;
    \item testing whether $\veoneone$ and $\vetwoone$ are equivalent in $\tenv$ yields $\vb$\ProseOrTypeError.
  \end{itemize}

  \item \AllApplyCase{enum\_enum}
  \begin{itemize}
    \item $\vlone$ is an \enumerationtypeterm{} index for the identifier $\venumone$ and a list of labels, that is,
          $\ArrayLengthEnum(\venumone, \Ignore)$;
    \item $\vltwo$ is an \enumerationtypeterm{} index for the identifier $\venumtwo$ and a list of labels, that is,
           $\ArrayLengthEnum(\venumtwo, \Ignore)$;
    \item $\vb$ is $\True$ if and only if $\venumone$ is equal to $\venumtwo$.
  \end{itemize}
\end{itemize}

\FormallyParagraph
\begin{mathpar}
\inferrule[different\_labels]{
  \astlabel(\vlone) \neq \astlabel(\vltwo)
}{
  \arraylengthequal(\vlone, \vltwo) \typearrow \False
}
\end{mathpar}

\begin{mathpar}
\inferrule[expr\_expr]{
  \exprequal(\veoneone, \vetwoone) \typearrow \vb \OrTypeError
}{
  \arraylengthequal(\ArrayLengthExpr(\veoneone), \ArrayLengthExpr(\vetwoone)) \typearrow \vb
}
\end{mathpar}

\begin{mathpar}
\inferrule[enum\_enum]{
  \vb \eqdef (\venumone = \venumtwo)
}{
  \arraylengthequal(\ArrayLengthEnum(\venumone, \Ignore), \ArrayLengthEnum(\venumtwo, \Ignore)) \typearrow \vb
}
\end{mathpar}

\TypingRuleDef{PolynomialToExpr}
\hypertarget{def-polynomialtoexpr}{}
The function
\[
\polynomialtoexpr(\overname{\polynomial}{\vp}) \typearrow \overname{\expr}{\ve}
\]
transforms a polynomial $\vp$ into the corresponding expression $\ve$.

\ProseParagraph
\OneApplies
\begin{itemize}
  \item \AllApplyCase{empty}
  \begin{itemize}
    \item $\vp$ is the polynomial with an empty list of monomials, that is, $\emptyfunc$;
    \item define $\ve$ as the literal expression for $0$.
  \end{itemize}

  \item \AllApplyCase{non\_empty}
  \begin{itemize}
    \item $\vp$ is the polynomial $f$;
    \item sorting (see $\sort$ for details) the graph of $f$ (see $\funcgraph$ for details)
          yields $\monoms$ --- a list consisting of pairs of unitary monomials and rationals.
          In principle, any total order of the graph of $f$ is acceptable for sorting.
          The function $\comparemonomialbindings$ provides one such way of ordering
          the graph of $f$;
    \item transforming $\monoms$ to an expression and sign via $\monomialstoexpr$ yields the expression $\veone$
          and sign $\vsone$;
    \item define $\ve$ as $\veone$ if $\vsone$ is $1$, the integer literal expression for $0$ if $\vsone$ is $0$,
          and the unary expression negating $\veone$, that is,
          $\EUnop(\NEG, \veone)$, if $\vsone$ is $-1$.
  \end{itemize}
\end{itemize}

\FormallyParagraph
\begin{mathpar}
\inferrule[empty]{}
{
  \polynomialtoexpr(\overname{\emptyfunc}{\vp}) \typearrow \overname{\ELiteral(\lint(0))}{\ve}
}
\and
\inferrule[non\_empty]{
  \sort(\funcgraph(f), \comparemonomialbindings) = \monoms\\
  \monomialstoexpr(\monoms) \typearrow (\veone, \vsone)\\
  {
  \ve \eqdef \begin{cases}
    \ELiteral(\lint(0))   & \text{if }\vsone = 0\\
    \veone                & \text{if }\vsone = 1\\
    \EUnop(\NEG, \veone)  & \text{if }\vsone = -1
  \end{cases}
  }
}{
  \polynomialtoexpr(\overname{f}{\vp}) \typearrow \ve
}
\end{mathpar}

\TypingRuleDef{CompareMonomialBindings}
\hypertarget{def-comparemonomialbindings}{}
The function
\[
\comparemonomialbindings(
  (\overname{\monomial}{\vmone}\times\overname{\Q}{\vqone}) \aslsep
  (\overname{\monomial}{\vmtwo}\times\overname{\Q}{\vqtwo})
) \aslto \overname{\{-1,0,1\}}{\vs}
\]
compares two monomial bindings given by $(\vmone, \vqone)$ and $(\vmtwo, \vqtwo)$
and yields in $\vs$ $-1$ to mean that the first monomial binding should be ordered before the second,
$0$ to mean that any ordering of the monomial bindings is acceptable,
and $1$ to mean that the second monomial binding should be ordered before the first.

\ProseParagraph
\OneApplies
\begin{itemize}
  \item \AllApplyCase{equal\_monomials}
  \begin{itemize}
    \item $\vmone$ is $f$ and $\vmtwo$ is $g$;
    \item $f$ is equal to $g$;
    \item $\vs$ is the sign of $\vqtwo - \vqone$.
  \end{itemize}

  \item \AllApplyCase{different\_monomials}
  \begin{itemize}
    \item $\vmone$ is $f$ and $\vmtwo$ is $g$;
    \item $f$ is different from to $g$;
    \item $\ids$ is the list obtained by taking the set of identifiers in the domain of $f$ and in the domain of $g$,
          and sorting them according to the lexical order for identifiers (ASCII string order);
    \item $\vv$ is the first identifier in $\ids$ for which $f$ and $g$ behave differently (either one of them is defined
          for $\vv$ and the other is not, or they both bind $\vv$ to a different value);
    \item $s$ is determined as follows: $1$ if $\vv$ is not in the domain of $f$ and is in the domain of $g$;
          $-1$ if $\vv$ is not in the domain of $f$ and is in the domain of $g$;
          otherwise it is the sign of $g(\vv)-f(\vv)$.
  \end{itemize}
\end{itemize}

\FormallyParagraph
\hypertarget{def-compareidentifier}{}
The function $\compareidentifier$ compares two identifiers, which are lists of ASCII characters,
via the lexicographic ordering.

\begin{mathpar}
\inferrule[equal\_monomials]{
  f = g\\
  \vs \eqdef \sign(\vqtwo - \vqone)
}{
  \comparemonomialbindings((\overname{f}{\vmone}, \vqone), (\overname{g}{\vmtwo}, \vqtwo)) \typearrow \vs
}
\and
\inferrule[different\_monomials]{
  f \neq g\\
  \ids \eqdef \sort(\dom(f) \cup \dom(g), \compareidentifier)\\
  \ids \eqname \idsone \concat \idstwo\\
  \vi\in\listrange(\idsone): f(\idsone[\vi]) = g(\idsone[\vi])\\
  \vv \eqdef \idstwo[1]\\
  \vs \eqdef {
    \begin{cases}
    1 & f(\vv) = \bot \land g(\vv) \neq \bot\\
    -1  & f(\vv) \neq \bot \land g(\vv) = \bot\\
    \sign(g(\vv) - f(\vv))  & f(\vv) \neq \bot \land g(\vv) \neq \bot\\
    \end{cases}
}
}{
  \comparemonomialbindings((\overname{f}{\vmone}, \vqone), (\overname{g}{\vmtwo}, \vqtwo)) \typearrow \vs
}
\end{mathpar}

\TypingRuleDef{MonomialsToExpr}
\hypertarget{def-monomialstoexpr}{}
The function
\[
\monomialstoexpr(\overname{(\overname{\unitarymonomial}{\vm}\times\overname{\Q}{q})^*}{\monoms}) \aslto
(\overname{\expr}{\ve},\overname{\{-1,0,1\}}{\vs})
\]
transforms a list consisting of pairs of unitary monomials and rational factors $\monoms$ (so, general monomials),
into an expression $\ve$, which represents the absolute value of the sum of all the monomials,
and a sign value $\vs$, which indicates the sign of the resulting sum.

\ProseParagraph
\OneApplies
\begin{itemize}
  \item \AllApplyCase{empty}
  \begin{itemize}
    \item $\monoms$ is an empty list;
    \item $\ve$ is the literal expression for the integer $0$ and $\vs$ is $0$.
  \end{itemize}

  \item \AllApplyCase{non\_empty}
  \begin{itemize}
    \item $\monoms$ is a list with $(\vm,q)$ as its \head\ and $\monomsone$ as its \tail;
    \item transforming the unitary monomial $\vm$ to an expression via \\ $\unitarymonomialstoexpr$ yields $\veonep$;
    \item transforming $\veonep$ and $q$ via $\monomialtoexpr$ yields the expression $\veone$ and sign $\vsone$;
    \item transforming $\monoms$ to an expression and sign via $\monomialstoexpr$ yields $(\vetwo, \vstwo)$;
    \item symbolically adding $\veone, \vsone, \vetwo, \vstwo$ via $\symaddexpr$ yields $(\ve, \vs)$.
  \end{itemize}
\end{itemize}

\FormallyParagraph
\begin{mathpar}
\inferrule[empty]{}
{
  \monomialstoexpr(\overname{\emptylist}{\monoms}) \typearrow (\overname{\ELiteral(\lint(0))}{\ve}, \overname{0}{\vs})
}
\and
\inferrule[non\_empty]{
  \unitarymonomialstoexpr(\vm) \typearrow \veonep\\
  \monomialtoexpr(\veonep, q) \typearrow (\veone,\vsone)\\
  \monomialstoexpr(\monoms) \typearrow (\vetwo, \vstwo)\\
  \symaddexpr(\veone, \vsone, \vetwo, \vstwo) \typearrow (\ve, \vs)
}{
  \monomialstoexpr(\overname{[(\vm,q)] \concat \monomsone}{\monoms}) \typearrow (\ve, \vs)
}
\end{mathpar}

\TypingRuleDef{MonomialToExpr}
\hypertarget{def-monomialtoexpr}{}
The function
\[
\monomialtoexpr(\overname{\expr}{\ve} \aslsep \overname{\Q}{q})
\aslto (\overname{\expr}{\newe} \times \overname{\{-1,0,1\}}{\vs})
\]
transforms an expression $\ve$ and rational $q$ into the expression $\newe$,
which represents the absolute value of $\ve$ multiplied by $q$, and the sign of $q$ as $\vs$.

\ProseParagraph
\OneApplies
\begin{itemize}
  \item \AllApplyCase{q\_zero}
  \begin{itemize}
    \item $q$ is $0$;
    \item $\newe$ is the literal expression for $0$;
    \item $\vs$ is $0$.
  \end{itemize}

  \item \AllApplyCase{q\_natural}
  \begin{itemize}
    \item $q$ a strictly positive;
    \item symbolically multiplying the literal expression for $q$ and $\ve$ via $\symmulexpr$ yields $\newe$;
    \item $\vs$ is $1$.
  \end{itemize}

  \item \AllApplyCase{q\_positive\_fraction}
  \begin{itemize}
    \item $q$ a strictly positive fraction, that is, not an integer;
    \item the reduced representation of the fraction $q$ is $\frac{d}{n}$;
    \item symbolically multiplying the literal expression for $q$ and $\ve$ via $\symmulexpr$ yields $\vetwo$;
    \item $\ve$ is the binary expression with operator $\DIV$ and operands $\vetwo$ and the literal expression for $n$;
    \item $\vs$ is $1$.
  \end{itemize}

  \item \AllApplyCase{q\_negative}
  \begin{itemize}
    \item $q$ a strictly negative;
    \item transforming $\ve$ with $-q$ to an expression and a sign via $\monomialtoexpr$ yields $(\newe, 1)$;
    \item $\vs$ is $-1$.
  \end{itemize}
\end{itemize}

\FormallyParagraph
\begin{mathpar}
\inferrule[q\_zero]{
  q = 0\\
}{
  \monomialtoexpr(\ve, q) \typearrow (\overname{\ELiteral(\lint(0))}{\newe}, \overname{0}{\vs})
}
\and
\inferrule[q\_natural]{
  q > 0\\
  q \in \N\\
  \symmulexpr(\ELiteral(\lint(q)), \ve) \typearrow \newe
}{
  \monomialtoexpr(\ve, q) \typearrow (\newe, \overname{1}{\vs})
}
\and
\inferrule[q\_positive\_fraction]{
  q > 0\\
  q \not\in \N\\\\
  q \eqname \frac{d}{n} \;\;\;\;\text{ is the reduced fraction for }q\\\\
  \symmulexpr(\ELiteral(\lint(d)), \ve) \typearrow \vetwo\\
  \newe \eqdef \EBinop(\DIV, \vetwo, \ELiteral(\lint(n)))
}{
  \monomialtoexpr(\ve, q) \typearrow (\newe, \overname{1}{\vs})
}
\and
\inferrule[q\_negative]{
  q < 0\\
  \monomialtoexpr(\ve, -q) \typearrow (\newe, 1)
}{
  \monomialtoexpr(\ve, q) \typearrow (\newe, \overname{-1}{\vs})
}
\end{mathpar}

\TypingRuleDef{SymAddExpr}
\hypertarget{def-symaddexpr}{}
The function
\[
\symaddexpr(
  \overname{\expr}{\veone} \aslsep
  \overname{\{-1,0,1\}}{\vsone} \aslsep
  \overname{\expr}{\vetwo} \aslsep
  \overname{\{-1,0,1\}}{\vstwo}
  )
  \typearrow (\overname{\expr}{\ve}, \overname{\{-1,0,1\}}{\vs})
\]
symbolically sums the expressions $\veone$ and $\vetwo$
with respective signs $\vsone$ and $\vstwo$
yielding the expression $\ve$ and sign $\vs$.

The effect of the function can be summarized by the following table:
\begin{center}
\begin{tabular}{|c|c|c|c|}
\hline
& \multicolumn{3}{c|}{$\vsone$}\\
\hline
$\vstwo$                      & -1                         & 0               & 1\\
\hline
-1                            &  $(\veone+\vetwo, \vsone)$ & (\vetwo,\vstwo) & $(\veone-\vetwo, \vsone)$ \\
\hline
 0                            &  $(\veone, \vsone)$        & (\veone,\vsone) & $(\veone, \vsone)$ \\
 \hline
 1                            &  $(\veone-\vetwo, \vsone)$ & (\vetwo,\vstwo) & $(\veone+\vetwo, \vsone)$ \\
\hline
\end{tabular}
\end{center}

\ExampleDef{Symbolically Summing Signed Expressions}
The following are some examples of inputs and outputs for symbolic addition:
\begin{center}
\begin{tabular}{rrrrrr}
\textbf{$\vsone$} & \textbf{$\veone$} & \textbf{$\vstwo$} & \textbf{$\vetwo$} & \textbf{$\ve$} & \textbf{$\vs$}\\
\hline
-1 & \verb|x| & -1 & \verb|y| & \verb|x+y| & -1\\
-1 & \verb|x| & 0 & \verb|y| & \verb|x| & -1\\
-1 & \verb|x| & 1 & \verb|y| & \verb|x-y| & -1\\
1 & \verb|x| & -1 & \verb|y| & \verb|x-y| & 1\\
\end{tabular}
\end{center}

\ProseParagraph
\OneApplies
\begin{itemize}
  \item \AllApplyCase{zero}
  \begin{itemize}
    \item either $\vsone$ is $0$ or $\vstwo$ is $0$;
    \item the result is $(\vetwo, \vstwo)$ if $\vsone$ is $0$ and $(\veone, \vsone)$, otherwise.
  \end{itemize}

  \item \AllApplyCase{same\_sign}
  \begin{itemize}
    \item both $\vsone$ and $\vstwo$ are not $0$;
    \item $\vsone$ is equal to $\vstwo$;
    \item $\ve$ is the binary expression with operator $\PLUS$ and operands $\veone$ and $\vetwo$,
          that is, $\EBinop(\PLUS, \veone, \vetwo)$;
    \item $\vs$ is $\vsone$;
  \end{itemize}

  \item \AllApplyCase{different\_signs}
  \begin{itemize}
    \item both $\vsone$ and $\vstwo$ are not $0$;
    \item $\vsone$ is different from $\vstwo$;
    \item $\ve$ is the binary expression with operator $\MINUS$ and operands $\veone$ and $\vetwo$,
          that is, $\EBinop(\MINUS, \veone, \vetwo)$;
    \item $\vs$ is $\vsone$;
  \end{itemize}
\end{itemize}

\FormallyParagraph
\begin{mathpar}
\inferrule[zero]{
  (\vsone = 0 \lor \vstwo = 0)\\
  (\ve, \vs) \eqdef \choice{\vsone = 0}{(\vetwo, \vstwo)}{(\veone, \vsone)}
}{
  \symaddexpr(\veone, \vsone, \vetwo, \vstwo) \typearrow (\ve, \vs)
}
\and
\inferrule[same\_sign]{
  \vsone \neq 0 \land \vstwo \neq 0\\
  \vsone = \vstwo
}{
  \symaddexpr(\veone, \vetwo) \typearrow (\overname{\EBinop(\PLUS, \veone, \vetwo)}{\ve}, \overname{\vsone}{\vs})
}
\and
\inferrule[different\_signs]{
  \vsone \neq 0 \land \vstwo \neq 0\\
  \vsone \neq \vstwo
}{
  \symaddexpr(\veone, \vetwo) \typearrow (\overname{\EBinop(\MINUS, \veone, \vetwo)}{\ve}, \overname{\vsone}{\vs})
}
\end{mathpar}

\TypingRuleDef{UnitaryMonomialsToExpr}
\hypertarget{def-unitarymonomialstoexpr}{}
The function
\[
\unitarymonomialstoexpr(\overname{(\identifier\times\N)^*}{\monoms}) \aslto \overname{\expr}{\ve}
\]
transforms a list of single-variable unitary monomials $\monoms$ into an expression $\ve$.
Intuitively, $\monoms$ represented a multiplication of the single-variable unitary monomials.

\ProseParagraph
\OneApplies
\begin{itemize}
  \item \AllApplyCase{empty}
  \begin{itemize}
    \item $\monoms$ is the empty list;
    \item $\ve$ is the literal expression for $1$.
  \end{itemize}

  \item \AllApplyCase{exp\_zero}
  \begin{itemize}
    \item $\monoms$ is a list where the first element is $(\vv, 0)$ and its tail is $\monoms$;
    \item transforming $\monomsone$ to an expression yields $\ve$.
  \end{itemize}

  \item \AllApplyCase{exp\_one}
  \begin{itemize}
    \item $\monoms$ is a list where the first element is $(\vv, 1)$ and its tail is $\monoms$;
    \item $\veone$ is the variable expression for $\vv$;
    \item transforming $\monomsone$ to an expression yields $\vetwo$;
    \item symbolically multiplying $\veone$ and $\vetwo$ via $\symmulexpr$ yields $\ve$.
  \end{itemize}

  \item \AllApplyCase{exp\_two}
  \begin{itemize}
    \item $\monoms$ is a list where the first element is $(\vv, 2)$ and its tail is $\monoms$;
    \item $\veone$ is the binary expression with operator $\MUL$ and operands $\EVar(\vv)$ and $\EVar(\vv)$
          (that is, $\vv$ squared);
    \item transforming $\monomsone$ to an expression yields $\vetwo$;
    \item symbolically multiplying $\veone$ and $\vetwo$ via $\symmulexpr$ yields $\ve$.
  \end{itemize}

  \item \AllApplyCase{exp\_gt\_two}
  \begin{itemize}
    \item $\monoms$ is a list where the first element is $(\vv, n)$ and its tail is $\monoms$;
    \item $n$ is greater than $1$;
    \item $\veone$ is the binary expression with operator $\POW$ and base operand being the variable expression for $\vv$
          and the exponent operand being the variable expression for $n$;
    \item transforming $\monomsone$ to an expression yields $\vetwo$;
    \item symbolically multiplying $\veone$ and $\vetwo$ via $\symmulexpr$ yields $\ve$.
  \end{itemize}
\end{itemize}

\FormallyParagraph
\begin{mathpar}
\inferrule[empty]{}
{
  \unitarymonomialstoexpr(\overname{\emptylist}{\monoms}) \typearrow \overname{\ELiteral(\lint(1))}{\ve}
}
\end{mathpar}

\begin{mathpar}
\inferrule[exp\_zero]{
  \unitarymonomialstoexpr(\monomsone) \typearrow \ve
}{
  \unitarymonomialstoexpr(\overname{[(\vv, 0)] \concat \monomsone}{\monoms}) \typearrow \ve
}
\end{mathpar}

\begin{mathpar}
\inferrule[exp\_one]{
  \veone \eqdef \EVar(\vv)\\
  \unitarymonomialstoexpr(\monomsone) \typearrow \vetwo\\
  \symmulexpr(\veone, \vetwo) \typearrow \ve
}{
  \unitarymonomialstoexpr(\overname{[(\vv, 1)] \concat \monomsone}{\monoms}) \typearrow \ve
}
\end{mathpar}

\begin{mathpar}
\inferrule[exp\_two]{
  \veone \eqdef \AbbrevEBinop{\MUL}{\EVar(\vv)}{\EVar(\vv)}\\
  \unitarymonomialstoexpr(\monomsone) \typearrow \vetwo\\
  \symmulexpr(\veone, \vetwo) \typearrow \ve
}{
  \unitarymonomialstoexpr(\overname{[(\vv, 2)] \concat \monomsone}{\monoms}) \typearrow \ve
}
\end{mathpar}

\begin{mathpar}
\inferrule[exp\_gt\_two]{
  n \geq 2\\
  \veone \eqdef \AbbrevEBinop{\POW}{\EVar(\vv)}{\ELiteral(n)}\\
  \unitarymonomialstoexpr(\monomsone) \typearrow \vetwo\\
  \symmulexpr(\veone, \vetwo) \typearrow \ve
}{
  \unitarymonomialstoexpr(\overname{[(\vv, n)] \concat \monomsone}{\monoms}) \typearrow \ve
}
\end{mathpar}

\TypingRuleDef{SymMulExpr}
\hypertarget{def-symmulexpr}{}
The function
$
\symmulexpr(\overname{\expr}{\veone} \aslsep \overname{\expr}{\vetwo}) \typearrow \overname{\expr}{\ve}
$
produces an expression representing the multiplication of expressions $\veone$ and $\vetwo$,
simplifying away the case where one of the operands is the literal one.

\ExampleDef{Symbolic Multiplication}
The following table shows examples of symbolically multiplying the expression $\veone$ by the expression $\vetwo$,
yielding the result in $\ve$:
\begin{center}
\begin{tabular}{lll}
\textbf{$\veone$} & \textbf{$\vetwo$} & \textbf{$\ve$}\\
\hline
\verb|x*0|  & \verb|1|    & \verb|x*0|\\
\verb|1|    & \verb|x*0|  & \verb|x*0|\\
\verb|x*0|  & \verb|x*0|  & \verb|(x*0)*(x*0)|
\end{tabular}
\end{center}

\ProseParagraph
\OneApplies
\begin{itemize}
  \item \AllApplyCase{one\_operand}
  \begin{itemize}
    \item either $\veone$ or $\vetwo$ is the literal expression for $1$;
    \item $\ve$ is $\vetwo$ if $\veone$ is the literal expression for $1$ and $\veone$, otherwise.
  \end{itemize}

  \item \AllApplyCase{no\_one\_operand}
  \begin{itemize}
    \item both $\veone$ and $\vetwo$ are not the literal expression for $1$;
    \item $\ve$ is the binary expression for multiplying $\vetwo$ and $\veone$.
  \end{itemize}
\end{itemize}

\FormallyParagraph
\begin{mathpar}
\inferrule[one\_operand]{
  (\veone = \ELiteral(\lint(1)) \lor \vetwo = \ELiteral(\lint(1)))\\
  \ve \eqdef \choice{\veone = \ELiteral(\lint(1))}{\vetwo}{\veone}
}{
  \symmulexpr(\veone, \vetwo) \typearrow \ve
}
\end{mathpar}

\begin{mathpar}
\inferrule[no\_one\_operand]{
  (\veone \neq \ELiteral(\lint(1)) \land \vetwo \neq \ELiteral(\lint(1)))
}{
  \symmulexpr(\veone, \vetwo) \typearrow \overname{\EBinop(\MUL, \veone, \vetwo)}{\ve}
}
\end{mathpar}

\TypingRuleDef{TypeOf}
\hypertarget{def-typeof}{}
The function
\[
  \typeof(\overname{\staticenvs}{\tenv} \aslsep \overname{\identifier}{\vs})
  \;\aslto\; \overname{\ty}{\tty}\ \cup\ \overname{\TTypeError}{\TypeErrorConfig}
\]
looks up the environment $\tenv$ for a type $\tty$ associated with an identifier
$\vs$. The result is \typingerrorterm{} if $\vs$ is not associated with any type.

\ExampleDef{The Type Bound to an Identifier}
In \listingref{expressions-evar}, the type bound to the global storage element
\verb|global_non_constant| after annotating the global storage element declarations
is \verb|integer{19}|,
%
the type bound to the local storage element \verb|var_x| in the static environment
after annotating the local declaration \verb|var var_x = LOCAL_CONSTANT;| is \verb|integer{7}|.
%
Invoking $\typeof$ for the static environment just before annotating the local
declaration \verb|var x = t;| in \listingref{expressions-evar-undefined} yields a \typingerrorterm.

\ProseParagraph
\OneApplies
\begin{itemize}
  \item \AllApplyCase{local}
  \begin{itemize}
    \item $\vs$ is associated with a type $\tty$ in the local environment of $\tenv$;
  \end{itemize}

  \item \AllApplyCase{global}
  \begin{itemize}
    \item $\vs$ is not associated with a type in the local environment of $\tenv$;
    \item $\vs$ is associated with a type $\tty$ in the global environment of $\tenv$;
  \end{itemize}

  \item \AllApplyCase{error}
  \begin{itemize}
    \item $\vs$ is not associated with a type in the local environment of $\tenv$;
    \item $\vs$ is not associated with a type in the global environment of $\tenv$;
    \item the result is a \typingerrorterm{} indicating that $\vs$ was expected to be associated
          with a type.
  \end{itemize}
\end{itemize}

\FormallyParagraph
\begin{mathpar}
\inferrule[local]{
  L^\tenv.\localstoragetypes(\vs) = \tty
}{
  \typeof(\tenv, \vs) \typearrow \tty
}
\and
\inferrule[global]{
  L^\tenv.\localstoragetypes(\vs) = \bot\\
  G^\tenv.\globalstoragetypes(\vs) = \tty
}{
  \typeof(\tenv, \vs) \typearrow \tty
}
\and
\inferrule[error]{
  L^\tenv.\localstoragetypes(\vs) = \bot\\
  G^\tenv.\globalstoragetypes(\vs) = \bot
}{
  \typeof(\tenv, \vs) \typearrow \TypeErrorVal{\UndefinedIdentifier}
}
\end{mathpar}
\CodeSubsection{\TypeOfBegin}{\TypeOfEnd}{../StaticModel.ml}

\TypingRuleDef{NormalizeOpt}
\hypertarget{def-normalizeopt}{}
The helper function
\[
\normalizeopt(\overname{\staticenvs}{\tenv} \aslsep \overname{\expr}{\ve}) \aslto
\overname{\langle\expr\rangle}{\neweopt} \cup \overname{\TTypeError}{\TypeErrorConfig}
\]
is similar to $\normalize$, except that it returns $\None$ when $\ve$ is not an
expression that can be symbolically simplified.
\ProseOtherwiseTypeError

\ProseParagraph
\OneApplies
\begin{itemize}
  \item \AllApplyCase{normalizable}
  \begin{itemize}
    \item applying $\toir$ to $\ve$ in $\tenv$ to obtain a symbolic expression yields a symbolic expression $\vpone$
    (that is, not $\CannotBeTransformed$)\ProseOrTypeError;
    \item applying $\normalize$ to $\ve$ in $\tenv$ yields $\newe$;
    \item \Proseeqdef{$\neweopt$}{$\langle\newe\rangle$}.
  \end{itemize}

  \item \AllApplyCase{not\_normalizable}
  \begin{itemize}
    \item applying $\toir$ to $\ve$ in $\tenv$ to obtain a symbolic expression yields $\CannotBeTransformed$;
    \item \Proseeqdef{$\neweopt$}{$\None$}.
  \end{itemize}
\end{itemize}

\FormallyParagraph
\begin{mathpar}
\inferrule[normalizable]{
  \toir(\tenv, \ve) \typearrow \vpone \OrTypeError\\\\
  \vpone \neq \CannotBeTransformed\\
  \normalize(\tenv, \ve) \typearrow \newe
}{
  \normalize(\tenv, \ve) \typearrow \overname{\langle\newe\rangle}{\neweopt}
}
\and
\inferrule[not\_normalizable]{
  \toir(\tenv, \ve) \typearrow \CannotBeTransformed
}{
  \normalize(\tenv, \ve) \typearrow \overname{\None}{\neweopt}
}
\end{mathpar}
