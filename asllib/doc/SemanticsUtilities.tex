%%%%%%%%%%%%%%%%%%%%%%%%%%%%%%%%%%%%%%%%%%%%%%%%%%%%%%%%%%%%%%%%%%%%%%%%%%%%%%%
\chapter{Semantics Utility Rules\label{chap:SemanticsUtilityRules}}
%%%%%%%%%%%%%%%%%%%%%%%%%%%%%%%%%%%%%%%%%%%%%%%%%%%%%%%%%%%%%%%%%%%%%%%%%%%%%%%

This chapter defines the following helper relations for operating on \nativevalues,
\hyperlink{def-envs}{environments}, and operations involving values and types:
\begin{itemize}
  \item \SemanticsRuleRef{GetStackSize}
  \item \SemanticsRuleRef{SetStackSize}
  \item \SemanticsRuleRef{IncrStackSize}
  \item \SemanticsRuleRef{DecrStackSize}
  \item \SemanticsRuleRef{RemoveLocal};
  \item \SemanticsRuleRef{ReadIdentifier};
  \item \SemanticsRuleRef{WriteIdentifier};
  \item \SemanticsRuleRef{ConcatBitvectors};
  \item \SemanticsRuleRef{ReadFromBitvector};
  \item \SemanticsRuleRef{WriteToBitvector};
  \item \SemanticsRuleRef{GetIndex};
  \item \SemanticsRuleRef{SetIndex};
  \item \SemanticsRuleRef{GetField};
  \item \SemanticsRuleRef{SetField};
  \item \SemanticsRuleRef{DeclareLocalIdentifier};
  \item \SemanticsRuleRef{DeclareLocalIdentifierM};
  \item \SemanticsRuleRef{DeclareLocalIdentifierMM};
\end{itemize}

\SemanticsRuleDef{GetStackSize}
\hypertarget{def-getstacksize}{}
The function
\[
\getstacksize(\overname{\denv}{\dynamicenvs} \aslsep \overname{\name}{\identifier}) \aslto \overname{\vs}{\N}
\]
retrieves the value associated with $\name$ in $\denv.\stacksize$ or $0$ if no value is associated with it.

\ExampleDef{Retrieving the Stack Size}
In \listingref{CheckRecurseLimit}, the stack size upon the call to \verb|factorial| from \verb|main|
with the environment $\env_0 \eqdef (\tenv_0, \denv_0)$ is $0$.
That is, $\getstacksize(\denv_0, \factorial) \evalarrow 0$.

\ProseParagraph
define $\vs$ is $0$ if no value is associated with $\name$ in $\denv.\stacksize$ and the value bound to
$\name$ in $\denv.\stacksize$ otherwise.

\FormallyParagraph
\begin{mathpar}
\inferrule{
  \vs \eqdef \choice{\name \in \dom(\denv.\stacksize)}{\denv.\stacksize(\name)}{0}
}{
  \getstacksize(\denv, \name) \evalarrow \vs
}
\end{mathpar}

\SemanticsRuleDef{SetStackSize}
\hypertarget{def-setstacksize}{}
The function
\[
\setstacksize(\overname{\genv}{\globaldynamicenvs} \aslsep \overname{\name}{\identifier} \aslsep \overname{\vv}{\N}) \aslto
\overname{\newgenv}{\dynamicenvs}
\]
updates the value bound to $\name$ in $\genv.\storage$ to $\vv$, yielding the new global dynamic environment $\newgenv$.

\ExampleDef{Setting the Stack Size}
In \listingref{CheckRecurseLimit}, the stack size upon the call to \verb|factorial| from \verb|main|
with the environment $\env_0 \eqdef (\tenv_0, \denv_0)$ is $0$, and it is then set to $1$.
That is, \\
$\setstacksize(\denv_0, \factorial, 1) \evalarrow \denv_1$
and\\
$\getstacksize(\denv_1, \factorial) \evalarrow 1$.

\ProseParagraph
define $\newdenv$ as $\genv$ updated to bind $\name$ to $\vv$ in $\genv.\stacksize$.

\FormallyParagraph
\begin{mathpar}
\inferrule{}{
  \setstacksize(\genv, \name, \vv) \evalarrow \overname{\genv.\stacksize[\name\mapsto\vv]}{\newgenv}
}
\end{mathpar}

\SemanticsRuleDef{IncrStackSize}
\hypertarget{def-incrstacksize}{}
The function
\[
\incrstacksize(\overname{\genv}{\globaldynamicenvs} \aslsep \overname{\name}{\identifier}) \aslto
\overname{\newgenv}{\globaldynamicenvs}
\]
increments the value associated with $\name$ in $\genv.\stacksize$, yielding the updated global dynamic environment $\newgenv$.

\ExampleDef{Incrementing the Stack Size}
In \listingref{CheckRecurseLimit}, the stack size upon the call to \verb|factorial| from \verb|main|
with the environment $\env_0 \eqdef (\tenv_0, \denv_0)$ is $0$, and it is incremented to $1$.
That is, \\
$\incrstacksize(\denv_0, \factorial) \evalarrow \denv_1$
and\\
$\getstacksize(\denv_1, \factorial) \evalarrow 1$.

\ProseParagraph
\AllApply
\begin{itemize}
  \item applying $\getstacksize$ to $\name$ in $(\genv, \emptyfunc)$ yields $\vprev$;
  \item applying $\setstacksize$ to $\name$ and $\vprev + 1$ in $\genv$ yields $\newgenv$.
\end{itemize}

\FormallyParagraph
\begin{mathpar}
\inferrule{
  \getstacksize((\genv, \emptyfunc), \name) \evalarrow \vprev\\
  \setstacksize(\genv, \name, \vprev + 1) \evalarrow \newgenv
}{
  \incrstacksize(\genv, \name) \evalarrow \newgenv
}
\end{mathpar}

\SemanticsRuleDef{DecrStackSize}
\hypertarget{def-decrstacksize}{}
The function
\[
\decrstacksize(\overname{\genv}{\globaldynamicenvs} \aslsep \overname{\name}{\identifier}) \aslto
\overname{\newgenv}{\globaldynamicenvs} \cup \overname{\newdenv}{\dynamicenvs}
\]
decrements the value associated with $\name$ in $\genv.\stacksize$, yielding the updated global dynamic environment $\newgenv$.
It is assumed that $\getstacksize((\genv, \emptyfunc), \name)$ yields a positive value.

\ExampleDef{Decrementing the Stack Size}
In \listingref{CheckRecurseLimit}, the stack size upon returning from \verb|main| to \verb|factorial|
with the environment $\env_1 \eqdef (\tenv_1, \denv_1)$ is $1$, and it is decremented to $0$.
That is, \\
$\decrstacksize(\denv_1, \factorial) \evalarrow \denv_0$
and\\
$\getstacksize(\denv_0, \factorial) \evalarrow 0$.

\ProseParagraph
\AllApply
\begin{itemize}
  \item applying $\getstacksize$ to $\name$ in $(\genv, \emptyfunc)$ yields $\vprev$;
  \item applying $\setstacksize$ to $\name$ and $\vprev - 1$ in $\genv$ yields $\newgenv$.
\end{itemize}

\FormallyParagraph
\begin{mathpar}
\inferrule{
  \getstacksize((\genv, \emptyfunc), \name) \evalarrow \vprev\\
  \setstacksize(\genv, \name, \vprev - 1) \evalarrow \newgenv
}{
  \decrstacksize(\genv, \name) \evalarrow \newgenv
}
\end{mathpar}

\SemanticsRuleDef{RemoveLocal}
\ProseParagraph
The relation
\hypertarget{def-removelocal}{}
\[
  \removelocal(\overname{\envs}{\env} \aslsep \overname{\Identifiers}{\name}) \;\aslrel\; \overname{\envs}{\newenv}
\]
removes the binding of the identifier $\name$ from the local storage of the environment $\env$,
yielding the environment $\newenv$.

This relation is used to maintain the invariant that the dynamic environment
maintains bindings only for identifiers that are in scope. That is, only identifiers
that would be in the local static environment.
%
This invariant is not necessary for correctness.
Removing identifiers from the dynamic environment does not have any
observable effect.

\ExampleDef{Removing Local Storage Elements from Dynamic Environments}
In \listingref{RemoveLocal}, the local storage element \verb|i| is removed after the (only) \forstatementterm,
and the local storage element \verb|exn| is removed after the (only) \trystatementterm.

\ASLListing{Removing local storage elements from dynamic environments}{RemoveLocal}{\semanticstests/SemanticsRule.RemoveLocal.asl}
The output to the console is the following:
% CONSOLE_BEGIN aslref \semanticstests/SemanticsRule.RemoveLocal.asl
\begin{Verbatim}[fontsize=\footnotesize, frame=single]
i=0
i=11/2
exn=TRUE
\end{Verbatim}
% CONSOLE_END

\AllApply
\begin{itemize}
  \item $\env$ consists of the static environment $\tenv$ and dynamic environment $\denv$;
  \item $\newenv$ consists of the static environment $\tenv$ and the dynamic environment
  with the same global component as $\denv$ --- $G^\denv$, and local component $L^\denv$,
  with the identifier $\name$ removed from its domain.
\end{itemize}

\FormallyParagraph
(Recall that $[\name\mapsto\bot]$ means that $\name$ is not in the domain of the resulting function.)
\begin{mathpar}
  \inferrule{
    \env \eqname (\tenv, (G^\denv, L^\denv))\\
    \newenv \eqdef (\tenv, (G^\denv, L^\denv[\name \mapsto \bot]))
  }
  {
    \removelocal(\env, \name) \evalarrow \newenv
  }
\end{mathpar}

\SemanticsRuleDef{ReadIdentifier}
\ProseParagraph
The relation
\hypertarget{def-readidentifier}{}
\[
  \readidentifier(\overname{\Identifiers}{\name}\aslsep\overname{\vals}{\vv}) \;\aslrel\; \XGraphs
\]
creates an \executiongraph{} that represents the reading of the value $\vv$ into a storage element
given by the identifier $\name$.
The result is an execution graph containing a single Read Effect,
which denotes reading from $\name$.
%
The value $\vv$ is ignored, as execution graphs do not contain values.

\ExampleDef{The Effect of Reading from an Identifier}
In \listingref{ReadIdentifier}, the first iteration of the (only) \forstatementterm{}
generates (see \SemanticsRuleRef{EvalFor}) the following transition at the first iteration:
\[
\readidentifier(\vi, \nvint(0)) \evalarrow \ReadEffect(\vi) \enspace.
\]

\ASLListing{A simple for loop}{ReadIdentifier}{\semanticstests/SemanticsRule.ReadIdentifier.asl}

\FormallyParagraph
\begin{mathpar}
\inferrule{}
{
  \readidentifier(\name, \vv) \evalarrow \ReadEffect(\name)
}
\end{mathpar}

\SemanticsRuleDef{WriteIdentifier}
\ProseParagraph
The relation
\hypertarget{def-writeidentifier}{}
\[
  \writeidentifier(\overname{\Identifiers}{\name}\aslsep\overname{\vals}{\vv}) \;\aslrel\; \XGraphs
\]
creates an \executiongraph{} that represents the writing of the value $\vv$ into
the storage element given by an identifier $\name$.
The result is an execution graph containing a single Write Effect,
which denotes writing into $\name$.
%
The value $\vv$ is ignored, as execution graphs do not contain values.

\ExampleDef{The Effect of Writing to an Identifier}
In \listingref{ReadIdentifier}, the first iteration of the (only) \forstatementterm{}
generates (see \SemanticsRuleRef{EvalFor}) the following transition at the first iteration,
as the value of the index variable $\vi$ is incremented from $\nvint(0)$ to $\nvint(1)$:
\[
\writeidentifier(\vi, \nvint(1)) \evalarrow \WriteEffect(\vi) \enspace.
\]

\FormallyParagraph
\begin{mathpar}
\inferrule{}
{
  \writeidentifier(\name, \vv) \evalarrow \WriteEffect(\name)
}
\end{mathpar}

\SemanticsRuleDef{ConcatBitvectors}
\hypertarget{def-concatbitvector}{}
The relation
\[
  \concatbitvectors(\overname{\tbitvector^*}{\vvs}) \;\aslrel\; \overname{\tbitvector}{\newvs}
\]
transforms a (possibly empty) list of bitvector \nativevalues\ $\vvs$ into a single bitvector
$\newvs$.

\ExampleDef{Concatenating Bitvectors}
The specification in \listingref{ConcatBitvectors} shows examples of concatenating
the \bitvectortypeterm{} fields of a \collectiontypeterm{} and of a \recordtypeterm.
\ASLListing{Concatenating bitvectors}{ConcatBitvectors}{\semanticstests/SemanticsRule.ConcatBitvectors.asl}

\ProseParagraph
\OneApplies
\begin{itemize}
  \item \AllApplyCase{empty}
  \begin{itemize}
    \item \Proseemptylist{$\vvs$};
    \item \Proseeqdef{$\newvs$}{the \nativevalue{} bitvector for the empty sequence of bits}.
  \end{itemize}

  \item \AllApplyCase{non\_empty}
  \begin{itemize}
    \item $\vvs$ is a \Proselist{$\vv$}{$\vvs'$};
    \item view $\vv$ as the \nativevalue{} bitvector for the sequence of bits $\bv$;
    \item applying $\concatbitvectors$ to $\vvs'$ yields the
          \nativevalue{} bitvector for the sequence of bits $\bv$;
    \item \Proseeqdef{$\vres$}{the concatenation of $\bv$ and $\bv'$};
    \item \Proseeqdef{$\newvs$}{the \nativevalue{} bitvector for sequence of bits $\vres$}.
  \end{itemize}
\end{itemize}

Define $\newvs$ as the concatenation of bitvectors listed in $\vvs$.

\FormallyParagraph
\begin{mathpar}
\inferrule[empty]{}
{
  \concatbitvectors(\overname{\emptylist}{\vvs}) \evalarrow \overname{\nvbitvector(\emptylist)}{\newvs}
}
\end{mathpar}

\begin{mathpar}
\inferrule[non\_empty]{
  \vvs = [\vv] \concat \vvs'\\
  \vv\eqname\nvbitvector(\bv)\\
  \concatbitvectors(\vvs') \evalarrow \nvbitvector(\bv')\\
  \vres \eqdef \bv \concat \bv'
}{
  \concatbitvectors(\vvs) \evalarrow \nvbitvector(\vres)
}
\end{mathpar}

\SemanticsRuleDef{SlicesToPositions}
\hypertarget{def-slicestopositions}{}
The relation
\[
  \slicestopositions(\overname{(\overname{\tint}{\vs_i}\times\overname{\tint}{\vl_i})^+}{\slices}) \;\aslrel\;
  (\overname{\N^*}{\positions} \cup\ \overname{\TDynError}{\BadIndex})
\]
returns the list of positions (indices) specified by the slices $\slices$,
if all slices are consist of only non-negative integers.
\ProseOtherwiseDynamicError

\ExampleDef{Converting Slices to Lists of Indices}
In \listingref{SlicesToPositions}, the slices \verb|1+:3, 7:5|
are converted into the list of position $3, 2, 1, 7, 6, 5$.
\ASLListing{Converting slices to lists of indices}{SlicesToPositions}{\semanticstests/SemanticsRule.SlicesToPositions.asl}

The specification in \listingref{SlicesToPositions-bad} terminates with a \dynamicerrorterm,
since the slice \verb|from+:6| evaluates to $(\nvint(-1), \nvint(6))$ and the starting position $-1$
is illegal for a slice.
\ASLListing{An illegal slice}{SlicesToPositions-bad}{\semanticstests/SemanticsRule.SlicesToPositions.bad.asl}

% Transliteration note: there are two slices_to_positions functions:
% the one in Native.ml, which calls the one in ASTUtils.ml.
% This rule transliterates the version in Native.ml, inlining the version
% in ASTUtils.ml.

\ProseParagraph
\OneApplies
\begin{itemize}
  \item \AllApplyCase{empty}
  \begin{itemize}
    \item $\slices$ is empty;
    \item \Proseeqdef{$\positions$}{the empty list}.
  \end{itemize}

  \item \AllApplyCase{non\_empty}
  \begin{itemize}
    \item $\slices$ is the \Proselist{the range starting at $\vs$ of length $\vl$}{$\slicestwo$};
    \item checking that both $\vs$ and $\vl$ are non-negative yields $\True$\ProseTerminateAs{\BadIndex};
    \item \Proseeqdef{$\positionsone$}{the range of indices starting at $\vs + \vl - 1$ and going down to $\vs$};
    \item applying $\slicestopositions$ to $\slicestwo$ yields $\positionstwo$\ProseOrError;
    \item \Proseeqdef{$\positions$}{the concatenation of $\slicesone$ and $\slicestwo$}.
  \end{itemize}
\end{itemize}

\FormallyParagraph
\begin{mathpar}
\inferrule[empty]{}{
  \slicestopositions(\overname{\emptylist}{\slices}) \evalarrow \overname{\emptylist}{\positions}
}
\end{mathpar}

\begin{mathpar}
\inferrule[non\_empty]{
  \checktrans{\vs \geq 0 \land \vl \geq 0}{\BadIndex} \checktransarrow \DynErrorConfig\\
  \positionsone \eqdef (\vs + \vl - 1)..\vs\\
  \slicestopositions(\slicestwo) \evalarrow \positionstwo \OrDynError
}{
  \slicestopositions(\overname{[(\nvint(\vs), \nvint(\vl))] \concat \slicestwo}{\slices}) \evalarrow
  \overname{\positionsone \concat \positionstwo}{\positions}
}
\end{mathpar}

\SemanticsRuleDef{ReadFromBitvector}
\hypertarget{def-readfrombitvector}{}
The relation
\[
  \readfrombitvector(\overname{\Nvalue}{\vv} \aslsep \overname{(\tint\times\tint)^*}{\slices}) \;\aslrel\;
  \overname{\tbitvector}{\vnew} \cup \overname{\TDynError}{\DynErrorConfig}
\]
reads from a bitvector $\bv$, or an integer seen as a bitvector, the indices specified by the list of slices $\slices$,
thereby concatenating their values.

Notice that the bits of a bitvector go from the least significant bit being on the right to the most significant bit being on the left,
which is reflected by how the rules list the bits.
The effect of placing the bits in sequence is that of concatenating the results
from all of the given slices.
Also notice that bitvector bits are numbered from 1 and onwards, which is why we add 1 to the indices specified
by the slices when accessing a bit.

\ExampleDef{Reading From a Bitvector}
The specification in \listingref{ReadFromBitvector} shows examples of reading slices from a bitvector
and reading slices from an integer (first converted to a bitvector), followed by the output to the console.

\ASLListing{Reading from a bitvector}{ReadFromBitvector}{\semanticstests/SemanticsRule.ReadFromBitvector.asl}
% CONSOLE_BEGIN aslref \semanticstests/SemanticsRule.ReadFromBitvector.asl
\begin{Verbatim}[fontsize=\footnotesize, frame=single]
empty_bv_slice = 0x, empty_i_slice = 0x
slice_bv = 0x9c5, slice_i = 0x9c5
\end{Verbatim}
% CONSOLE_END

\ExampleDef{Conversion of Integers to Bitvectors}
The specification in \listingref{AsBitvector} shows an example of converting a negative
integer into a bitvector, followed by the output to the console.

\ASLListing{Converting a signed integer into a bitvector}{AsBitvector}{\semanticstests/SemanticsRule.AsBitvector.asl}
% CONSOLE_BEGIN aslref \semanticstests/SemanticsRule.AsBitvector.asl
\begin{Verbatim}[fontsize=\footnotesize, frame=single]
bv = 0xfdf0, bv_i = 0xfdf0
\end{Verbatim}
% CONSOLE_END

\ProseParagraph
\OneApplies
\begin{itemize}
  \item \AllApplyCase{bitvector}
  \begin{itemize}
    \item $\vv$ is a native bitvector for the list of bits $\bv$;
    \item applying $\slicestopositions$ to $\slices$ yields the list $\positions$\ProseOrError;
    \item \Proseeqdef{$\maxpos$}{the maximal index in $\positions$};
    \item let $\bits$ be the sequence of bits $\bv[n] ... \bv[1]$;
    \item checking that $\maxpos$ is less than $n$ yields $\True$\ProseTerminateAs{\BadIndex};
    \item \Proseeqdef{$\newbits$}{the list of bits $\bits[j]$ for every \Proselistrange{$j$}{$\bits$}};
    \item \Proseeqdef{$\vv$}{the native bitvector for the list of bits $\newbits$}.
  \end{itemize}

  \item \AllApplyCase{integer}
  \begin{itemize}
    \item $\vv$ is a native integer for the integer $\vi$;
    \item applying $\slicestopositions$ to $\slices$ yields the list $\positions$\ProseOrError;
    \item \Proseeqdef{$\maxpos$}{the maximal index in $\positions$};
    \item \Proseeqdef{$\bits$}{$\vi$  given as two's complement little endian form of $\maxpos + 1$ bits};
    \item checking that $\maxpos$ is less than $n$ yields $\True$\ProseTerminateAs{\BadIndex};
    \item \Proseeqdef{$\newbits$}{the list of bits $\bits[j]$ for every \Proselistrange{$j$}{$\bits$}};
    \item \Proseeqdef{$\vv$}{the native bitvector for the list of bits $\newbits$}.
  \end{itemize}
\end{itemize}

\FormallyParagraph
\begin{mathpar}
\inferrule[bitvector]{
  \slicestopositions(\slices) \evalarrow \positions \OrDynError\\\\
  \maxpos \eqdef \max(\ \{ \vj\in\listrange(\positions) : \positions[\vj] \}\ )\\\\
  \commonprefixline\\\\
  \bits \eqname \bv[n] ... \bv[1]\\
  \checktrans{\maxpos < n}{\BadIndex} \checktransarrow \OrDynError\\\\
  \commonsuffixline\\\\
  % The following premise is essentially Bitvector.extract_slice
  \newbits \eqdef [\ \vj \in \listrange(\positions): \bits[\vj]\ ]
}{
  \readfrombitvector(\overname{\nvbitvector(\bv)}{\vv}, \slices) \evalarrow \overname{\nvbitvector(\newbits)}{\newv}
}
\end{mathpar}

\begin{mathpar}
\inferrule[integer]{
  \slicestopositions(\slices) \evalarrow \positions \OrDynError\\\\
  \maxpos \eqdef \max(\ \{ \vj\in\listrange(\positions) : \positions[\vj] \}\ )\\\\
  \commonprefixline\\\\
  \bits \eqdef \vi\text{ given as two's complement little endian form of }(\maxpos + 1)\text{ bits}\\
  \commonsuffixline\\\\
  % The following premise is essentially Bitvector.extract_slice
  \newbits \eqdef [\ \vj \in \listrange(\positions): \bits[\vj]\ ]
}{
  \readfrombitvector(\overname{\nvint(\vi)}{\vv}, \slices) \evalarrow \overname{\nvbitvector(\newbits)}{\newv}
}
\end{mathpar}

\SemanticsRuleDef{WriteToBitvector}
\hypertarget{def-writetobitvector}{}
The relation
\[
  \writetobitvector(\overname{(\tint\times\tint)^*}{\slices} \aslsep \overname{\tbitvector}{\src} \aslsep \overname{\tbitvector}{\dst})
  \;\aslrel\; \overname{\tbitvector}{\vv} \cup \overname{\TDynError}{\DynErrorConfig}
\]
overwrites the bits of $\dst$ at the positions given by $\slices$ with the bits of $\src$.

See \ExampleRef{Writing to a Bitvector}, following the definition of $\writetobitvector$.

\newcommand\bitfunc[0]{\textfunc{bit}}

\ProseParagraph
\AllApply
\begin{itemize}
  \item $\src$ is a native bitvector consisting of bits $\vs_m \ldots \vs_0$;
  \item $\dst$ is a native bitvector consisting of bits $\vd_n \ldots \vd_0$;
  \item applying $\slicestopositions$ to $\slices$ yields the list of indices $\positions$;
  \item view $\positions$ as the list $I_m \ldots I_0$;
  \item define the function $\bitfunc$ as mapping an index $i$ in $0$ to $n$ to
        $\vs_j$, if there exists an index $I_j$ in $\positions$ such that $I_j$ is equal to $i$,
        and $\vd_i$, otherwise.
  \item \Proseeqdef{$\vbits$}{the list of bits defined as
        $\bitfunc(n)\ldots\bitfunc(0)$};
  \item \Proseeqdef{$\vv$}{the native bitvector for $\vbits$}.
\end{itemize}

\FormallyParagraph
\begin{mathpar}
\inferrule{
  \src \eqname \nvbitvector(\vs_m \ldots \vs_0)\\
  \dst \eqname \nvbitvector(\vd_n \ldots \vd_0)\\
  \slicestopositions(\slices) \evalarrow \positions \OrDynError\\\\
  \positions \eqname I_m \ldots I_0 \\
  {\bitfunc = \lambda i \in 0..n.\left\{ \begin{array}{ll}
    \vs_j & \exists j\in 1..m.\ i = I_j\\
    \vd_i & \text{otherwise}
  \end{array} \right.}\\\\
  \vbits\eqdef [i=n..0: \bitfunc(i)]\\
}{
  \writetobitvector(\slices, \src, \dst) \evalarrow \overname{\nvbitvector(\vbits)}{\vv}
}
\end{mathpar}

\ExampleDef{Writing to a Bitvector}
In reference to \listingref{semantics-leslice}, we have the following application of the current rule:
\begin{mathpar} % SUPPRESS_TEXTTT_LINTER
\inferrule{
  \src = \overname{0}{\vs_5}\overname{0}{\vs_4}\overname{0}{\vs_3}\overname{0}{\vs_2}\overname{0}{\vs_1}\overname{0}{\vs_0}\\
  \dst = \overname{1}{\vd_7}\overname{1}{\vd_6}\overname{1}{\vd_5}\overname{1}{\vd_4}\overname{1}{\vd_3}\overname{1}{\vd_2}\overname{1}{\vd_1}\overname{1}{\vd_0}\\
  \slicestopositions(8, [\overname{(0, 4)}{\texttt{3:0}}, \overname{(6, 2)}{\texttt{7:6}}]) \evalarrow
  [3, 2, 1, 0, 7, 6]\\
  \positions \eqdef [\overname{3}{I_5}, \overname{2}{I_4}, \overname{1}{I_3}, \overname{0}{I_2}, \overname{7}{I_1}, \overname{6}{I_0}]\\
  {\bitfunc = \lambda i \in 0..7.\left\{ \begin{array}{ll}
    \vs_j & \exists j\in 1..5.\ i = I_j\\
    \vd_i & \text{otherwise}
  \end{array} \right.}\\
  \vbits \eqdef \bitfunc(7)\ \bitfunc(6)\ \bitfunc(5)\ \bitfunc(4)\ \bitfunc(3)\ \bitfunc(2)\ \bitfunc(1)\ \bitfunc(0)
}{
  {
  \begin{array}{r}
    \writetobitvector(
      [\overname{(0, 4)}{\texttt{3:0}}, \overname{(6, 2)}{\texttt{7:6}}],
      \overname{\nvbitvector(000000)}{\src},
      \overname{\nvbitvector(11111111)}{\dst}) \evalarrow\\
    \nvbitvector(
      \overname{0}{\vs_1}
      \overname{0}{\vs_0}
      \overname{1}{\vd_5}
      \overname{1}{\vd_4}
      \overname{0}{\vs_5}
      \overname{0}{\vs_4}
      \overname{0}{\vs_3}
      \overname{0}{\vs_2})
  \end{array}
  }
}
\end{mathpar}

\SemanticsRuleDef{GetIndex}
\ProseParagraph
The relation
\hypertarget{def-getindex}{}
\[
  \getindex(\overname{\N}{\vi} \aslsep \overname{\tvector}{\vvec}) \;\aslrel\; \overname{\tvector}{\vv_{\vi}}
\]
reads the value $\vv_i$ from the vector of values $\vvec$ at the index $\vi$.

\ExampleDef{Reading a Vector at an Index}
The specification in \listingref{GetIndex} shows examples of accessing indices of native vectors:
\begin{itemize}
  \item evaluating \verb|assert arr[[3]] == 3| involves \\
        $\getindex(3, \nvvector{\nvint(0), \nvint(3), \nvint(0)}) \evalarrow \nvint(3)$;
  \item evaluating \verb|assert (5, 7).item0 == 5| involves \\
        $\getindex(0, \nvvector{\nvint(5), \nvint(7)}) \evalarrow \nvint(5)$;
  \item evaluating \verb|(5, 7) as (integer, integer)| involves \\
        $\getindex(0, \nvvector{\nvint(5), \nvint(7)}) \evalarrow \nvint(5)$ and\\
        $\getindex(1, \nvvector{\nvint(5), \nvint(7)}) \evalarrow \nvint(7)$ to check the asserting type conversion.
\end{itemize}
\ASLListing{Reading a vector at an index}{GetIndex}{\semanticstests/SemanticsRule.GetIndex.asl}

\FormallyParagraph
\begin{mathpar}
\inferrule{
  \vvec \eqname \vv_{0..k}\\
  \vi \leq k\\
}{
  \getindex(\vi, \vvec) \evalarrow \vv_{\vi}
}
\end{mathpar}

Notice that there is no rule to handle the case where the index is out of range ---
this is guaranteed by the typechecker not to happen. Specifically,
\begin{itemize}
  \item \TypingRuleRef{EGetArray} ensures that an index is within the bounds of the array
  being accessed via a check that the type of the index satisfies the type of the array size.
  \item Typing rules \TypingRuleRef{LEDestructuring}, \TypingRuleRef{PTuple},
  and \\ \TypingRuleRef{LDTuple} use the same index sequences for the tuples
  involved and the corresponding lists of expressions.
\end{itemize}
If the rules listed above do not hold the typechecker fails.

\SemanticsRuleDef{SetIndex}
\ProseParagraph
The relation
\hypertarget{def-setindex}{}
\[
  \setindex(\overname{\N}{\vi} \aslsep \overname{\vals}{\vv} \aslsep \overname{\tvector}{\vvec}) \;\aslrel\; \overname{\tvector}{\vres}
\]
overwrites the value at the given index $\vi$ in a vector of values $\vvec$ with the new value $\vv$.

\ExampleDef{Writing a Vector at an Index}
In \listingref{GetIndex}, evaluating the statement \verb|arr[[1]] = 3;|
involves
\[
\begin{array}{r}
\setindex(1, \nvint(3), \nvvector{\nvint(0), \nvint(0), \nvint(0)}) \evalarrow\\
\nvvector{\nvint(0), \nvint(3), \nvint(0)} \enspace.
\end{array}
\]

\FormallyParagraph
\begin{mathpar}
\inferrule{
  \vvec \eqname \vu_{0..k}\\
  \vi \leq k\\
  \vres \eqname \vw_{0..k}\\
  \vv \eqdef \vw_{\vi} \\
  j \in \{0..k\} \setminus \{\vi\}.\ \vw_{j} = \vu_j\\
}{
  \setindex(\vi, \vv, \vvec) \evalarrow \vres
}
\end{mathpar}
Similar to $\getindex$, there is no need to handle the out-of-range index case.

\SemanticsRuleDef{GetField}
\ProseParagraph
The relation
\hypertarget{def-getfield}{}
\[
  \getfield(\overname{\Identifiers}{\name} \aslsep \overname{\trecord}{\record}) \;\aslrel\; \vals
\]
retrieves the value corresponding to the field name $\name$ from the record value $\record$.

\ExampleDef{Reading a Record Field}
In \listingref{GetField}, there are the following examples of reading a record field:
\begin{itemize}
  \item evaluating the expression \verb|color_to_int[[RED]]| involves\\
        $\getfield(\RED, \nvrecord{\RED\mapsto \nvint(0), \GREEN\mapsto \nvint(1), \BLUE\mapsto \nvint(2)}) \evalarrow \nvint(0)$;
  \item evaluating the expression \verb|r.RED| involves\\
        $\getfield(\RED, \nvrecord{\RED\mapsto \nvint(0), \GREEN\mapsto \nvint(1), \BLUE\mapsto \nvint(2)}) \evalarrow \nvint(0)$.
\end{itemize}
\ASLListing{Reading a record field}{GetField}{\semanticstests/SemanticsRule.GetField.asl}

\FormallyParagraph
\begin{mathpar}
\inferrule{
  \record \eqname \nvrecord{\fieldmap}
}{
  \getfield(\name, \record) \evalarrow \fieldmap(\name)
}
\end{mathpar}
The typechecker ensures, via \TypingRuleRef{EGetRecordField}, that the field $\name$ exists in $\record$.

\SemanticsRuleDef{SetField}
\ProseParagraph
The function
\hypertarget{def-setfield}{}
\[
  \setfield(\overname{\Identifiers}{\name} \aslsep \overname{\vals}{\vv} \aslsep \overname{\trecord}{\record}) \;\aslto\; \trecord
\]
overwrites the value corresponding to the field name $\name$ in the record value $\record$ with the value $\vv$.

\ExampleDef{Writing to a Record Field}
In \listingref{GetField}, there are the following examples of writing to a record field:
\begin{itemize}
  \item evaluating the statement \verb|color_to_int[[GREEN]] = 1;| involves\\
        $
        \begin{array}{r}
        \setfield\left(
          \begin{array}{l}
          \GREEN, \\
          \nvint(1),\\
          \nvrecord{\RED\mapsto \nvint(0), \GREEN\mapsto \nvint(0), \BLUE\mapsto \nvint(0)}
          \end{array}
          \right) \evalarrow\\
        \nvrecord{\RED\mapsto \nvint(0), \GREEN\mapsto \nvint(1), \BLUE\mapsto \nvint(0)} \enspace;
        \end{array}
        $
  \item evaluating the statement \verb|r.BLUE = -1;| involves\\
        $
        \begin{array}{r}
        \setfield\left(
          \begin{array}{l}
          \BLUE, \\
          \nvint(-1),\\
          \nvrecord{\RED\mapsto \nvint(0), \GREEN\mapsto \nvint(1), \BLUE\mapsto \nvint(2)}
          \end{array}
          \right) \evalarrow\\
        \nvrecord{\RED\mapsto \nvint(0), \GREEN\mapsto \nvint(1), \BLUE\mapsto \nvint(-1)} \enspace.
        \end{array}
        $
\end{itemize}

\FormallyParagraph
\begin{mathpar}
\inferrule{
  \record \eqname \nvrecord{\fieldmap}\\
  \fieldmapp \eqdef \fieldmap[\name\mapsto\vv]
}{
  \setfield(\name, \vv, \record) \evalarrow \nvrecord{\fieldmapp}
}
\end{mathpar}
The typechecker ensures that the field $\name$ exists in $\record$.

\SemanticsRuleDef{DeclareLocalIdentifier}
\ProseParagraph
The relation
\hypertarget{def-declarelocalidentifier}{}
\[
  \declarelocalidentifier(\overname{\envs}{\env} \aslsep \overname{\Identifiers}{\name} \aslsep \overname{\vals}{\vv}) \;\aslrel\;
  (\overname{\envs}{\newenv}\times\overname{\XGraphs}{\vg})
\]
associates $\vv$ to $\name$ as a local storage element in the environment $\env$ and
returns the updated environment $\newenv$ with the execution graph consisting of a Write Effect to $\name$.

\ExampleDef{Evaluating Local Declarations}
In \listingref{semantics-levar}, evaluating \verb|var x: integer = 3;|
binds \verb|x| to $\nvint(3)$ in the environment where \verb|x| is unbound
as well as producing the \executiongraph{} $\WriteEffect(\vx)$.

\FormallyParagraph
\begin{mathpar}
  \inferrule{
    \vg \eqdef \WriteEffect(\name)\\
    \env \eqname (\tenv, (G^\denv, L^\denv))\\
    \newenv \eqdef (\tenv, (G^\denv, L^\denv[\name\mapsto \vv]))
  }
  { \declarelocalidentifier(\env, \name, \vv) \evalarrow (\newenv, \vg)  }
\end{mathpar}

\SemanticsRuleDef{DeclareLocalIdentifierM}
\hypertarget{def-declarelocalidentifierm}{}
The relation
\[
  \declarelocalidentifierm(\overname{\envs}{\env} \aslsep
   \overname{\Identifiers}{\vx} \aslsep
   \overname{(\overname{\vals}{\vv}\times\overname{\XGraphs}{\vg})}{\vm}) \;\aslrel\;
  (\overname{\envs}{\newenv} \times \overname{\XGraphs}{\newg})
\]
declares the local identifier $\vx$ in the environment $\env$, in the context
of the value-graph pair $(\vv, \vg)$, yielding a pair consisting
of the environment $\newenv$ and \executiongraph{} $\newg$.

See \ExampleRef{Evaluating Local Declarations}.

\ProseParagraph
\AllApply
\begin{itemize}
  \item \newenv\ is the environment $\env$ modified to declare the variable $\vx$ as a local storage element;
  \item $\vgone$ is the execution graph resulting from the declaration of $\vx$;
  \item \Proseeqdef{$\newg$}{\executiongraph{} resulting from the ordered composition
        of $\vg$ and $\vgone$ with the $\asldata$ edge}.
\end{itemize}

\FormallyParagraph
\begin{mathpar}
  \inferrule{
    \vm \eqname (\vv, \vg)\\
    \declarelocalidentifier(\env, \vx, \vv) \evalarrow (\newenv, \vgone)\\
    \newg \eqdef \ordered{\vg}{\asldata}{\vgone}
  }
  {
    \declarelocalidentifierm(\env, \vx, \vm) \evalarrow (\newenv, \newg)
  }
\end{mathpar}

\SemanticsRuleDef{DeclareLocalIdentifierMM}
\hypertarget{def-declarelocalidentifermm}{}
The relation
\[
  \declarelocalidentifiermm(\overname{\envs}{\env} \aslsep
   \overname{\Identifiers}{\vx} \aslsep
   \overname{(\overname{\vals}{\vv}\times\overname{\XGraphs}{\vg})}{\vm}) \;\aslrel\;
  (\overname{\envs}{\newenv} \times \overname{\XGraphs}{\newg})
\]
declares the local identifier $\vx$ in the environment $\env$,
in the context of the value-graph pair $(\vv, \vg)$,
yielding a pair consisting of an environment $\newenv$
and an \executiongraph{} $\vgtwo$.

See \ExampleRef{Evaluating Local Declarations}.

\ProseParagraph
\AllApply
\begin{itemize}
  \item \newenv\ is the environment $\env$ modified to declare the variable $\vx$ as a local storage element;
  \item $\vgone$ is the execution graph resulting from the declaration of $\vx$;
  \item \Proseeqdef{$\newg$}{the execution graph resulting from the ordered composition
        of $\vg$ and $\vgone$ with the $\aslpo$ edge.}
\end{itemize}

\FormallyParagraph
\begin{mathpar}
\inferrule{
  \declarelocalidentifierm(\env, \vm) \evalarrow (\newenv, \vgone)\\
  \newg \eqdef \ordered{\vg}{\aslpo}{\vgone}
}{
  \declarelocalidentifiermm(\env, \vx, \vm) \evalarrow (\newenv, \newg)
}
\end{mathpar}
