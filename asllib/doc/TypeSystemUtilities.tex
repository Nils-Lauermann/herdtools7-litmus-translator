\chapter{Type System Utility Rules\label{chap:TypeSystemUtilityRules}}

\subsection{Checked Transitions}
\hypertarget{def-checktrans}{}
We define the following rules to allow us asserting that a condition holds,
returning a \typingerrorterm{} otherwise:
\begin{mathpar}
\inferrule[check\_trans\_true]{}{ \checktrans{\True}{\vcode} \checktransarrow \True }
\end{mathpar}

\begin{mathpar}
\inferrule[check\_trans\_false]{}{ \checktrans{\False}{\vcode} \checktransarrow \TypeErrorVal{\vcode} }
\end{mathpar}

\subsection{Converting a List of Pairs to a Map \label{sec:PairsToMap}}
\hypertarget{def-pairstomap}{}
The parametric function
\[
  \pairstomap(\overname{(\identifier\times T)^*}{\pairs}) \aslto \overname{(\identifier\partialto T)}{f} \cup \TTypeError
\]
converts a list of pairs --- $\pairs$ --- where each pair consists of an identifier and a value
of type $T$ into a function mapping each identifier to its respective value in the list.
If a duplicate identifier exists in $\pairs$ then a \typingerrorterm{} is returned.

\ProseParagraph
\OneApplies
\begin{itemize}
  \item \AllApplyCase{empty}
  \begin{itemize}
    \item $\pairs$ is empty;
    \item $f$ is the empty function.
  \end{itemize}

  \item \AllApplyCase{error}
  \begin{itemize}
    \item there exist two different positions in the list where the identifier is the same;
    \item the result is a \typingerrorterm{} indicating the existence of a duplicate identifier.
  \end{itemize}

  \item \AllApplyCase{okay}
  \begin{itemize}
    \item all identifiers occurring in the list are unique;
    \item $f$ is a function that associates to each identifier the value appearing with it in $\pairs$.
  \end{itemize}
\end{itemize}

\FormallyParagraph
\begin{mathpar}
\inferrule[empty]{}{ \pairstomap(\emptylist) \typearrow \emptyfunc }
\and
\inferrule[error]{
  i,j \in 1..k\\
  i \neq j\\
  \id_i = \id_j
}{
  \pairstomap([i=1..k: (\id_i,t_i)]) \typearrow \TypeErrorVal{\IdentifierAlreadyDeclared}
}
\end{mathpar}

\begin{mathpar}
\inferrule[okay]{
  \forall i,j \in 1..k. \id_i \neq \id_j\\
  {
  f \eqdef \lambda \id.\ \begin{cases}
    t_i & \text{ if }i\in1..k \land \id = \id_i\\
    \bot & \text{ otherwise}
  \end{cases}
  }
}{
  \pairstomap([i=1..k: (\id_i,t_i)]) \typearrow f
}
\end{mathpar}

\TypingRuleDef{CheckNoDuplicates}
\hypertarget{def-checknoduplicates}{}
The function
\[
  \checknoduplicates(\overname{\identifier^*}{\id_{1..k}}) \aslto \{\True\} \cup \TTypeError
\]
checks whether a non-empty list of identifiers contains a duplicate identifier. If it does not, the result
is $\True$ and otherwise the result is a \typingerrorterm{}.

\ExampleDef{Checking for Absence of Duplicates in an Identifier List}
In \listingref{typing-tenum}, annotating the \enumerationtypeterm{} \verb|Color|
involves checking that the list of its labels --- \verb|GREEN, ORANGE, RED| --- does not contain duplicates,
which is the case.

Similarly, in \listingref{typing-trecord}, annotating the record type \verb|MyRecord|
involves checking that the list of fields \verb|a, b| does not contain duplicates, which is the case.
In contrast, annotating the record type \verb|MyRecord| in \listingref{typing-trecord-bad},
involves checking that the list of fields \verb|v, b, v| does not contain duplicates, which is not the case,
thus resulting in a \typingerrorterm.

\ProseParagraph
\OneApplies
\begin{itemize}
  \item \AllApplyCase{okay}
  \begin{itemize}
    \item the set containing all identifiers in the list has the same cardinality as the length of the list;
    \item the result is $\True$.
  \end{itemize}

  \item \AllApplyCase{error}
  \begin{itemize}
    \item there exist two different positions in the list where the identifier is the same;
    \item the result is a \typingerrorterm{} indicating the existence of a duplicate identifier.
  \end{itemize}
\end{itemize}

\FormallyParagraph
\begin{mathpar}
\inferrule[okay]{
  \cardinality{\{\id_{1..k}\}} = k
}{
  \checknoduplicates(\id_{1..k}) \typearrow \True
}
\end{mathpar}

\begin{mathpar}
\inferrule[error]{
  i,j \in 1..k\\
  i \neq j\\
  \id_i = \id_j
}{
  \checknoduplicates(\id_{1..k}) \typearrow \TypeErrorVal{\IdentifierAlreadyDeclared}
}
\end{mathpar}

\TypingRuleDef{Sort}
\hypertarget{def-sort}{}
The parametric function
\[
\sort(\overname{T^*}{\vlone}, \overname{(T\times T)\rightarrow \{-1,0,1\}}{\compare}) \typearrow \overname{T^*}{\vltwo}
\]
sorts a list of elements of type $T$ --- $\vlone$ --- using the comparison function $\compare$,
resulting in the sorted list $\vltwo$.
$\compare(a, b)$ returns $1$ to mean that $a$ should be ordered before $b$,
$0$ to mean that $a$ and $b$ can be ordered in any way,
and $-1$ to mean that $b$ should be ordered before $a$.

\ExampleDef{Sorting Identifiers}
The following is an example of sorting lists of identifiers using lexicographic order:\\
$\sort([\vy, \vx], \compareidentifier) \typearrow [\vx, \vy]$.

The following is an example of sorting monomial bindings''
(see \TypingRuleRef{PolynomialToExpr}):\\
$\sort([(x, 1), (y, 1), (x, -1)], \comparemonomialbindings) \typearrow [(x, -1), (x, 1), (y, 1)]$.

\ProseParagraph
\OneApplies
\begin{itemize}
  \item \AllApplyCase{empty\_or\_single}
  \begin{itemize}
    \item $\vlone$ is either empty or contains a single element;
    \item $\vltwo$ is $\vlone$.
  \end{itemize}

  \item \AllApplyCase{two\_or\_more}
  \begin{itemize}
    \item $\vlone$ contains at least two elements;
    \item $f$ is a permutation of $1..n$;
    \item $\vltwo$ is the application of the permutation $f$ to $\vlone$;
    \item applying $\compare$ to every pair of consecutive elements in $\vltwo$ yields either $0$ or $1$.
  \end{itemize}
\end{itemize}

\FormallyParagraph
\begin{mathpar}
\inferrule[empty\_or\_single]{
  \listlen{\vlone} = n\\
  n < 2
}{
  \sort(\vlone, \compare) \typearrow \overname{\vlone}{\vltwo}
}
\and
\inferrule[two\_or\_more]{
  \listlen{\vlone} = n\\
  f : 1..n \rightarrow 1..n \text{ is a bijection}\\
  \vltwo \eqdef [\ i=1..n: \vlone[f(i)]\ ]\\
  i=1..n-1: \compare(\vltwo[i], \vltwo[i+1]) \geq 0
}{
  \sort(\vlone, \compare) \typearrow \vltwo
}
\end{mathpar}

\TypingRuleDef{FindBitfieldOpt}
\hypertarget{def-findbitfieldopt}{}
The function
\[
  \findbitfieldopt(\overname{\identifier}{\name} \aslsep \overname{\bitfield^*}{\bitfields})
  \aslto \overname{\langle\bitfield\rangle}{\vr}
\]
returns the bitfield associated with the name $\name$ in the list of bitfields $\bitfields$,
if there is one. Otherwise, the result is $\None$.

\ExampleDef{Finding Bitfields}
In \listingref{eget-bitfield}, annotating the expressions
\verb|p.flag|, \verb|p.data|, \verb|p.detailed_data.info|
require finding the corresponding bitfields, which all exist in the \verb|Packet| \bitvectortypeterm.

In contrast, in \listingref{eget-bad-bitfield}, annotating the expression \verb|p.undeclared_identifier|
requires finding the bitfield \verb|undeclared_identifier| in the \verb|Packet| \bitvectortypeterm,
which does not exist.

\ProseParagraph
\OneApplies
\begin{itemize}
  \item \AllApplyCase{match}
  \begin{itemize}
    \item $\bitfields$ starts with a bitfield $\vbf$;
    \item obtaining the name associated with $\vbf$ yields $\name$;
    \item the result is $\vbf$.
  \end{itemize}

  \item \AllApplyCase{tail}
  \begin{itemize}
    \item $\bitfields$ starts with a bitfield $\vbf$ and continues with the tail list $\bitfieldsp$;
    \item obtaining the name associated with $\vbf$ yields $\namep$, which is different than $\name$;
    \item finding the bitfield associated with $\name$ in $\bitfieldsp$ yields the result $\vr$.
  \end{itemize}

  \item \AllApplyCase{empty}
  \begin{itemize}
    \item $\bitfields$ is an empty list;
    \item the result is $\None$.
  \end{itemize}
\end{itemize}

\FormallyParagraph
\begin{mathpar}
\inferrule[match]{
  \bitfieldgetname(\vbf) \typearrow \name
}{
  \findbitfieldopt(\name, \overname{\vbf \concat \bitfieldsp}{\bitfields}) \typearrow \overname{\langle\vbf\rangle}{\vr}
}
\and
\inferrule[tail]{
  \bitfieldgetname(\vbf) \typearrow \namep\\
  \name \neq \namep\\
  \findbitfieldopt(\name, \bitfieldsp) \typearrow \vr
}{
  \findbitfieldopt(\name, \overname{\vbf \concat \bitfieldsp}{\bitfields}) \typearrow \vr
}
\and
\inferrule[empty]{}{
  \findbitfieldopt(\name, \overname{\emptylist}{\bitfields}) \typearrow \None
}
\end{mathpar}

\TypingRuleDef{TypeOfArrayLength}
\hypertarget{def-typeofarraylength}{}
The function
\[
  \typeofarraylength(\overname{\arrayindex}{\size}) \aslto
  \overname{\ty}{\vt}
\]
returns the type for the array length $\size$ in $\vt$.

\ExampleDef{Retrieving the Type of an Array from an Array Index}

In \listingref{typing-tarray}, annotating the expression \verb|int_arr[[3]]|,
yields \\
$\ArrayLengthExpr(\ELInt{3})$, and
\[
\begin{array}{r}
\typeofarraylength(\ArrayLengthExpr(\ELInt{3})) \typearrow \\
\unconstrainedinteger
\end{array}
\]

Annotating the expression \verb|big_little_arr[[LITTLE]]| yields\\
$\ArrayLengthEnum(\BitsArray, [\vBIG, \vLITTLE])$, and
\[
\begin{array}{r}
\typeofarraylength(\ArrayLengthEnum(\BitsArray, [\vBIG, \vLITTLE])) \typearrow \\
\TNamed(\BitsArray) \enspace.
\end{array}
\]

\ProseParagraph
\OneApplies
\begin{itemize}
  \item \AllApplyCase{enum}
  \begin{itemize}
    \item $\size$ is an enumeration index over the enumeration $\vs$, that is, \\ $\ArrayLengthEnum(\vs, \Ignore)$;
    \item $\vt$ is the named type for $\vs$, that is, $\TNamed(\vs)$.
  \end{itemize}

  \item \AllApplyCase{expr}
  \begin{itemize}
    \item $\size$ is an expression for integer-sized arrays, that is, $\ArrayLengthExpr(\Ignore)$;
    \item $\vt$ is the \unconstrainedintegertype.
  \end{itemize}
\end{itemize}

\FormallyParagraph
\begin{mathpar}
\inferrule[enum]{}
{
  \typeofarraylength(\ArrayLengthEnum(\vs, \Ignore)) \typearrow \TNamed(\vs)
}
\and
\inferrule[expr]{}{
  \typeofarraylength(\ArrayLengthExpr(\Ignore)) \typearrow \TInt(\unconstrained)
}
\end{mathpar}
\CodeSubsection{\TypeOfArrayLengthBegin}{\TypeOfArrayLengthEnd}{../types.ml}

\TypingRuleDef{AssocOpt}
\hypertarget{def-assocopt}{}
The function
\[
  \assocopt(\overname{(\identifier\times T)^*}{\vli} \aslsep \overname{\identifier}{\id}) \typearrow \langle \overname{T}{\vv} \rangle
\]
returns the value $\vv$ associated with the identifier $\id$ in the list of pairs $\vli$ or $\None$, if no such association exists.

\ExampleDef{Finding a Value Associated with an Identifier}
In \listingref{eget-record-field}, annotating the expressions \verb|my_record.i|
and \verb|my_record.b| require finding the types associated with the fields \verb|i| and \verb|b|
of the type \\
\verb|record{i: integer, b: boolean}|, respectively:
\[
\begin{array}{rcl}
\assocopt([(\vi, \unconstrainedinteger), (\vb, \TBool)], \vi) &\typearrow& \Some{\unconstrainedinteger}\\
\assocopt([(\vi, \unconstrainedinteger), (\vb, \TBool)], \vb) &\typearrow& \Some{\TBool}\\
\end{array}
\]

In \listingref{eget-bad-record-field}, however, annotating the expression \\
\verb|my_record.undeclared_identifier| fails:
\[
\assocopt([(\vi, \unconstrainedinteger), (\vb, \TBool)], \undeclaredidentifier) \typearrow \None\\
\]

\ProseParagraph
\OneApplies
\begin{itemize}
  \item \AllApplyCase{member}
  \begin{itemize}
    \item a pair $(\id,\vv)$ exists in the list $\vli$;
    \item the result is $\langle\vv\rangle$.
  \end{itemize}

  \item \AllApplyCase{not\_member}
  \begin{itemize}
    \item every pair $(\vx,\Ignore)$ in the list $\vli$ has $\vx\neq\id$;
    \item the result is $\None$.
  \end{itemize}
\end{itemize}

\FormallyParagraph
\begin{mathpar}
\inferrule[not\_member]{
  (\vx, \vv) \in \vli: \vx \neq \id
}{
  \assocopt(\vli, \id) \typearrow \None
}
\and
\inferrule[member]{
  (\id, \vv) \in \vli
}{
  \assocopt(\vli, \id) \typearrow \langle \vv \rangle
}
\end{mathpar}

\section{Static Environment Utilities}

\TypingRuleDef{WithEmptyLocal}
\hypertarget{def-withemptylocal}{}
The function
\[
  \withemptylocal(\overname{\globalstaticenvs}{\genv})
  \aslto \overname{\staticenvs}{\tenv}
\]
constructs a static environment from the global static environment $\genv$
and the empty local static environment.

\ExampleDef{Constructing a Static Environment with an Empty Local Environment}
In \listingref{CheckVarNotInGEnv}, the typechecker constructs a static environment
with an empty local environment to annotate the \verb|Color| type declaration.

\ProseParagraph
The result is a static environment where the global component is $\genv$ and the local component
is the local static environment of $\emptytenv$.
\FormallyParagraph
\begin{mathpar}
\inferrule{}{
  \withemptylocal(\genv) \typearrow (\genv, L^{\emptytenv})
}
\end{mathpar}
\CodeSubsection{\WithEmptyLocalBegin}{\WithEmptyLocalEnd}{../types.ml}

\TypingRuleDef{CheckVarNotInEnv}
\hypertarget{def-checkvarnotinenv}{}
The function
\[
  \checkvarnotinenv{\overname{\staticenvs}{\tenv} \aslsep \overname{\Strings}{\id}}
  \aslto \{\True\} \cup \TTypeError
\]
checks whether $\id$ is already declared in $\tenv$. If it is, the result is a \typingerrorterm{},
and otherwise the result is $\True$.

\ExampleDef{Checking Whether Variables are Bound in the Static Environment}
See \ExampleRef{Checking Whether an Identifier is Associated with a Global Storage Element} and
\ExampleRef{Checking Whether an Identifier is Associated with a Local Storage Element}.

The specification in \listingref{CheckVarNotInEnv-bad1} is ill-typed since \verb|A|
is declared as both a parameter and an argument.
\ASLListing{A repeating parameter}{CheckVarNotInEnv-bad1}{\typingtests/TypingRule.CheckVarNotInEnv.bad1.asl}

The specification in \listingref{CheckVarNotInEnv-bad2} is ill-typed since the variable \verb|y|
is repeated.
\ASLListing{A repeating local variable}{CheckVarNotInEnv-bad2}{\typingtests/TypingRule.CheckVarNotInEnv.bad2.asl}

\ProseParagraph
\AllApply
\begin{itemize}
  \item applying $\isundefined$ to $\vx$ in $\genv$ yields $\vb$;
  \item checking whether $\vb$ is $\True$ yields $\True$\ProseTerminateAs{\IdentifierAlreadyDeclared}.
\end{itemize}
\FormallyParagraph
\begin{mathpar}
\inferrule{
  \isundefined(\tenv, \id) \typearrow \vb\\
  \checktrans{\vb}{\IdentifierAlreadyDeclared} \checktransarrow \True \OrTypeError
}{
  \checkvarnotinenv{\tenv, \id} \typearrow \True
}
\end{mathpar}
\CodeSubsection{\CheckVarNotInEnvBegin}{\CheckVarNotInEnvEnd}{../types.ml}

\TypingRuleDef{CheckVarNotInGEnv}
\hypertarget{def-checkvarnotingenv}{}
The function
\[
  \checkvarnotingenv{\overname{\globalstaticenvs}{\genv} \aslsep \overname{\Strings}{\vx}}
  \aslto \{\True\} \cup \overname{\TTypeError}{\TypeErrorConfig}
\]
checks whether $\id$ is already declared in the global static environment $\genv$.
If it is, the result is a \typingerrorterm{}, and otherwise the result is $\True$.

\ExampleDef{Checking Whether an Identifier is Bound in the Global Static Environment}
In \listingref{CheckVarNotInGEnv}, the typechecker ensures that the following
identifiers are not bound in the global static environment upon annotating
the respective constructs:
\begin{itemize}
  \item \verb|Color| is checked for the type declaration;
  \item \verb|RED|, \verb|GREEN|, \verb|BLUE| for the enumeration labels;
  \item \verb|x| for the global variable.
\end{itemize}

\ASLListing{Checking whether an identifier is bound in the global static environment}{CheckVarNotInGEnv}{\typingtests/TypingRule.CheckVarNotInGEnv.asl}

The specification in \listingref{CheckVarNotInGEnv-bad} is ill-typed, since \verb|RED|
is declared both as a global variable and as an enumeration label.
\ASLListing{A name clash}{CheckVarNotInGEnv-bad}{\typingtests/TypingRule.CheckVarNotInGEnv.bad.asl}

\ProseParagraph
\AllApply
\begin{itemize}
  \item applying $\isglobalundefined$ to $\vx$ in $\genv$ yields $\vb$;
  \item checking whether $\vb$ is $\True$ yields $\True$\ProseTerminateAs{\IdentifierAlreadyDeclared}.
\end{itemize}

\FormallyParagraph
\begin{mathpar}
\inferrule{
  \isglobalundefined(\genv, \id) \typearrow \vb\\
  \checktrans{\vb}{\IdentifierAlreadyDeclared} \checktransarrow \True \OrTypeError
}{
  \checkvarnotingenv{\genv, \id} \typearrow \True
}
\end{mathpar}
\CodeSubsection{\CheckVarNotInGEnvBegin}{\CheckVarNotInGEnvEnd}{../Interpreter.ml}

\TypingRuleDef{AddLocal}
\hypertarget{def-addlocal}{}
The function
\[
  \addlocal(
    \overname{\staticenvs}{\tenv} \aslsep
    \overname{\identifier}{\id} \aslsep
    \overname{\ty}{\tty} \aslsep
    \overname{\localdeclkeyword}{\ldk})
  \aslto
  \overname{\staticenvs}{\newtenv}
\]
adds the identifier $\id$ as a local storage element with type $\tty$ and local declaration keyword $\ldk$
to the local environment of $\tenv$, resulting in the static environment $\newtenv$.

\ExampleDef{Adding a Local Storage Element}
In \listingref{AddLocal},
the following local storage elements are added:
\begin{itemize}
  \item \verb|N| as a parameter of \verb|foo|;
  \item \verb|bv| as an argument of \verb|foo|;
  \item \verb|x| as a local variable of \verb|main|;
  \item \verb|i| as the index of a \verb|for| loop in \verb|main|;
  \item \verb|exn| as a caught exception in \verb|main|.
\end{itemize}
\ASLListing{Adding a local storage element}{AddLocal}{\typingtests/TypingRule.AddLocal.asl}

\ProseParagraph
\AllApply
\begin{itemize}
  \item the map $\newlocalstoragetypes$ is defined by updating the map \\
        $\localstoragetypes$ of $\tenv$
        with the binding $\id$ to the type $\tty$ and local declaration keyword $\ldk$, that is, $(\tty,\ldk)$;
  \item $\newtenv$ is defined by updating the local environment with the binding of \\
        $\localstoragetypes$ to $\newlocalstoragetypes$.
\end{itemize}

\FormallyParagraph
\begin{mathpar}
\inferrule{
  \newlocalstoragetypes \eqdef L^\tenv.\localstoragetypes[\id \mapsto (\tty, \ldk)]\\
  \newtenv \eqdef (G^\tenv, L^\tenv[\localstoragetypes \mapsto \newlocalstoragetypes])
}
{
  \addlocal(\tenv, \id, \tty, \ldk) \typearrow \newtenv
}
\end{mathpar}

\TypingRuleDef{IsUndefined}
\hypertarget{def-isundefined}{}
The function
\[
\isundefined(\overname{\staticenvs}{\tenv} \aslsep \overname{\identifier}{\vx})
\aslto \overname{\Bool}{\vb}
\]
checks whether the identifier $\vx$ is defined as a storage element in the static environment $\tenv$.

See \ExampleRef{Checking Whether an Identifier is Associated with a Global Storage Element} and
\ExampleRef{Checking Whether an Identifier is Associated with a Local Storage Element}.

\ProseParagraph
$\vb$ is $\True$ if and only if $\vx$ is both undefined in the global static environment of $\tenv$
(see $\isglobalundefined$) and undefined in the local static environment of $\tenv$ (see $\islocalundefined$).

\FormallyParagraph
\begin{mathpar}
\inferrule{
  \isglobalundefined(G^\tenv, \vx) \typearrow \vbone\\
  \islocalundefined(L^\tenv, \vx) \typearrow \vbtwo\\
}{
  \isundefined(\tenv, \vx) \typearrow \overname{\vbone \land \vbtwo}{\vb}
}
\end{mathpar}
\CodeSubsection{\IsUndefinedBegin}{\IsUndefinedEnd}{../StaticEnv.ml}

\TypingRuleDef{IsGlobalUndefined}
\hypertarget{def-isglobalundefined}{}
The function
\[
\isglobalundefined(\overname{\globalstaticenvs}{\genv} \aslsep \overname{\identifier}{\vx}) \aslto \overname{\Bool}{\vb}
\]
checks whether the identifier $\vx$ is defined in the global static environment $\genv$ when subprogram definitions are ignored (see \RequirementRef{GlobalNamespace}), yielding the result in $\vb$.

\ExampleDef{Checking Whether an Identifier is Associated with a Global Storage Element}
\listingref{isglobalundefined} shows a specification which defines a variable \verb|X| and a subprogram \verb|Y|.
In the typing environment obtained by typechecking this specification, $\isglobalundefined$ will yield $\False$ for \verb|X| and $\True$ for \verb|Y|.
\ASLListing{Checking whether an identifier is defined in the global static environment}{isglobalundefined}{\typingtests/TypingRule.IsGlobalUndefined.asl}

\ProseParagraph
Define $\vb$ as $\True$ if and only if $\vx$ is not bound in any of the following maps of $\genv$:
$\globalstoragetypes$, and $\declaredtypes$.
\FormallyParagraph
\begin{mathpar}
\inferrule{
  {
  \begin{array}{llcl}
    \vb \eqdef & \genv.\globalstoragetypes(\vx) &=& \bot\ \land\\
              & \genv.\declaredtypes(\vx) &=& \bot
  \end{array}
  }
}{
  \isglobalundefined(\genv, \vx) \typearrow \vb
}
\end{mathpar}
\CodeSubsection{\IsGlobalUndefinedBegin}{\IsGlobalUndefinedEnd}{../StaticEnv.ml}

\TypingRuleDef{IsLocalUndefined}
\hypertarget{def-islocalundefined}{}
The function
\[
\islocalundefined(\overname{\localstaticenvs}{\lenv} \aslsep \overname{\identifier}{\vx}) \aslto \overname{\Bool}{\vb}
\]
checks whether $\vx$ is declared as a local storage element in the static local environment $\lenv$, yielding the result in $\vb$.

\ExampleDef{Checking Whether an Identifier is Associated with a Local Storage Element}
In \listingref{expressions-evar},
the following identifiers are defined at the point of annotating the statement \verb|return 0;|
of the \verb|main| function:
\verb|LOCAL_CONSTANT|,
\verb|var_x|,
\verb|y|,
\verb|local_non_constant|,
\verb|z|.

In \listingref{expressions-evar-undefined}, the identifier \verb|t| is undefined in the \verb|main| function.

\ProseParagraph
Define $\vb$ as $\True$ if and only if $\vx$ is not bound in the $\localstoragetypes$ of the static local environment $\lenv$.
\FormallyParagraph
\begin{mathpar}
\inferrule{}{
  \islocalundefined(\lenv, \vx) \typearrow \overname{L^\tenv.\localstoragetypes(\vx) = \bot}{\vb}
}
\end{mathpar}
\CodeSubsection{\IsLocalUndefinedBegin}{\IsLocalUndefinedEnd}{../StaticEnv.ml}

\TypingRuleDef{LookupConstant}
\hypertarget{def-lookupconstant}{}
The function
\[
  \lookupconstant(\overname{\staticenvs}{\tenv} \aslsep \overname{\identifier}{\vs})
  \;\aslto\; \overname{\literal}{\vv}\ \cup\ \{\bot\}
\]
looks up the environment $\tenv$ for a constant $\vv$ associated with an identifier
$\vs$. The result is $\bot$ if $\vs$ is not associated with any constant.

\ExampleDef{Looking Up Constants}
The specification in \listingref{LookupConstant}
shows examples where global constants and enumeration labels,
which are also bound to identifiers as constants, are looked up
in annotating statements.
\ASLListing{Looking up constants}{LookupConstant}{\typingtests/TypingRule.LookupConstant.asl}

\ProseParagraph
\ProseEqdef{$\vv$}{the value to which $\vs$ is bound in the $\constantvalues$ map
of the global static environment and $\bot$ if it is unbound.}

\FormallyParagraph
\begin{mathpar}
\inferrule{}{
  \lookupconstant(\tenv, \vs) \typearrow \overname{G^\tenv.\constantvalues(\vs)}{\vv}
}
\end{mathpar}
\CodeSubsection{\LookupConstantBegin}{\LookupConstantEnd}{../StaticEnv.ml}

\TypingRuleDef{AddGlobalConstant}
\hypertarget{def-addglobalconstant}{}
The function
\[
\addglobalconstant(
  \overname{\globalstaticenvs}{\genv} \aslsep
  \overname{\identifier}{\name} \aslsep
  \overname{\literal}{\vv}) \typearrow
  \overname{\globalstaticenvs}{\newgenv}
\]
binds the identifier $\name$ to the literal $\vv$ in the global static environment $\genv$,
yielding the updated global static environment $\newgenv$.

\ExampleDef{Binding Global Storage Elements to Constants}
The specification in \listingref{AddGlobalConstant} is well-typed.
Specifically, annotating the statement \verb|constant FOUR = 4;|
binds \verb|FOUR| to \verb|4|, which allows the type system
to infer that the statement \verb|var bv: bits(2^FOUR) = Zeros{FOUR*FOUR};| is well-typed.
\ASLListing{Binding global storage elements to constants}{AddGlobalConstant}{\typingtests/TypingRule.AddGlobalConstant.asl}

\ProseParagraph
Define $\newgenv$ as $\genv$ with the $\constantvalues$ map updated to bind $\name$ to $\vv$.
\FormallyParagraph
\begin{mathpar}
\inferrule{}{
  \addglobalconstant(\genv, \name, \vv) \typearrow \overname{\genv.\constantvalues[\name\mapsto\vv]}{\newgenv}
}
\end{mathpar}
\CodeSubsection{\AddGlobalConstantBegin}{\AddGlobalConstantEnd}{../StaticEnv.ml}

\TypingRuleDef{LookupImmutableExpr}
\hypertarget{def-lookupimmutableexpr}{}
The function
\[
\lookupimmutableexpr(\overname{\staticenvs}{\tenv} \aslsep \overname{\identifier}{\vx}) \aslto \overname{\expr}{\ve} \cup \{\bot\}
\]
looks up the static environment $\tenv$ for an immutable expression associated with the identifier $\vx$,
returning $\bot$ if there is none.

See \ExampleRef{Remembering Global Immutable Expressions}
and \ExampleRef{Updating Static Environments for Immutable Expressions}.

\ProseParagraph
\OneApplies
\begin{itemize}
  \item \AllApplyCase{local}
  \begin{itemize}
    \item applying $\exprequiv$ to $\vx$ in the local component of $\tenv$, yields $\ve$.
  \end{itemize}

  \item \AllApplyCase{global}
  \begin{itemize}
    \item applying $\exprequiv$ to $\vx$ in the local component of $\tenv$, yields $\bot$;
    \item applying $\exprequiv$ to $\vx$ in the global component of $\tenv$, yields $\ve$.
  \end{itemize}

  \item \AllApplyCase{none}
  \begin{itemize}
    \item applying $\exprequiv$ to $\vx$ in the local component of $\tenv$, yields $\bot$;
    \item applying $\exprequiv$ to $\vx$ in the global component of $\tenv$, yields $\bot$;
    \item $\ve$ is $\bot$.
  \end{itemize}
\end{itemize}
\FormallyParagraph
\begin{mathpar}
\inferrule[local]{
  L^\tenv.\exprequiv(\vx) = \ve
}{
  \lookupimmutableexpr(\tenv, \vx) \typearrow \ve
}
\end{mathpar}

\begin{mathpar}
\inferrule[global]{
  L^\tenv.\exprequiv(\vx) = \bot\\
  G^\tenv.\exprequiv(\vx) = \ve\\
}{
  \lookupimmutableexpr(\tenv, \vx) \typearrow \ve
}
\end{mathpar}

\begin{mathpar}
\inferrule[none]{
  L^\tenv.\exprequiv(\vx) = \bot\\
  G^\tenv.\exprequiv(\vx) = \bot\\
}{
  \lookupimmutableexpr(\tenv, \vx) \typearrow \bot
}
\end{mathpar}
\CodeSubsection{\AddGlobalStorageBegin}{\AddGlobalStorageEnd}{../StaticEnv.ml}

\TypingRuleDef{AddGlobalImmutableExpr}
\hypertarget{def-addglobalimmutableexpr}{}
The function
\[
\addglobalimmutableexpr(
  \overname{\staticenvs}{\tenv} \aslsep
  \overname{\identifier}{\vx}
  \aslsep \overname{\expr}{\ve}) \aslto \overname{\staticenvs}{\newtenv}
\]
binds the identifier $\vx$, which is assumed to name a global storage element,
to the expression $\ve$, which is assumed to be \symbolicallyevaluable,
in the static environment $\tenv$,
resulting in the updated environment $\newtenv$.

\ExampleDef{Remembering Global Immutable Expressions}
The specification in \listingref{AddGlobalImmutableExpr}
is well-typed, since it binds \verb|w| to \verb|2|
before annotating the global storage declaration for \verb|x|,
which is why it is able to prove that \verb|'11'| \typesatisfies{}
\verb|bits(2)|.
\ASLListing{Remembering global immutable expressions}{AddGlobalImmutableExpr}{\typingtests/TypingRule.AddGlobalImmutableExpr.asl}

\ProseParagraph
\ProseEqdef{$\newtenv$}{
  the static environment with the same local environment as $\tenv$ and a global environment
  where $\exprequiv$ binds $\vx$ to $\ve$.}

\FormallyParagraph
\begin{mathpar}
\inferrule{}{
  \addglobalimmutableexpr(\tenv, \vx, \ve) \typearrow \overname{(G^\tenv.\exprequiv[\vx \mapsto \ve], L^\tenv)}{\newtenv}
}
\end{mathpar}
\CodeSubsection{\AddGlobalImmutableExprBegin}{\AddGlobalImmutableExprEnd}{../StaticEnv.ml}

\TypingRuleDef{AddLocalImmutableExpr}
\hypertarget{def-addlocalimmutableexpr}{}
The function
\[
\addlocalimmutableexpr(
  \overname{\staticenvs}{\tenv} \aslsep
  \overname{\identifier}{\vx}
  \aslsep \overname{\expr}{\ve}) \aslto \overname{\staticenvs}{\newtenv}
\]
binds the identifier $\vx$, which is assumed to name a local storage element,
to the expression $\ve$, which is assumed to be \symbolicallyevaluable,
in the static environment $\tenv$,
resulting in the updated environment $\newtenv$.

See \ExampleRef{Updating Static Environments for Immutable Expressions}.

\ProseParagraph
\AllApply
\begin{itemize}
  \item define $\newtenv$ as the static environment with the same global environment as $\tenv$ and a local environment
        where $\exprequiv$ binds $\vx$ to $\ve$.
\end{itemize}
\FormallyParagraph
\begin{mathpar}
\inferrule{}{
  \addlocalimmutableexpr(\tenv, \vx, \ve) \typearrow \overname{(G^\tenv, L^\tenv.\exprequiv[\vx \mapsto \ve])}{\newtenv}
}
\end{mathpar}
\CodeSubsection{\AddLocalImmutableExprBegin}{\AddLocalImmutableExprEnd}{../StaticEnv.ml}

\TypingRuleDef{ShouldRememberImmutableExpression}
\hypertarget{def-shouldrememberimmutableexpression}{}
The helper function
\[
  \shouldrememberimmutableexpression(\overname{\TSideEffectSet}{\vses}) \typearrow \overname{\Bool}{\vb}
\]
tests whether the \sideeffectsetterm\ $\vses$
allows an expression with those \sideeffectdescriptorsterm\ to be
recorded as an immutable expression in the appropriate $\exprequiv$ map component
of the static environment, so that it can later be used to
reason about type satisfaction, yielding the result in $\vb$.

\ExampleDef{Remembering vs. not Remembering Immutable Expressions}
See \ExampleRef{Updating Static Environments for Immutable Expressions}
for examples of well-typed specifications where immutable expressions are remembered.

The specification in \listingref{ShouldRememberImmutableExpression}
is ill-typed, since \verb|l| is a mutable storage element, which is therefore
not bound to \verb|x| after its declaration, which is why the type satisfaction
test for \verb|t| fails.
\ASLListing{Remembering vs. not remembering immutable expressions}{ShouldRememberImmutableExpression}{\typingtests/TypingRule.ShouldRememberImmutableExpression.bad.asl}

\ProseParagraph
Define $\vb$ as $\True$ if and only if
applying $\issymbolicallyevaluable$ to $\vses$ with \PerformsAssertionsTerm\ removed from it,
yields $\True$.

\FormallyParagraph
\begin{mathpar}
\inferrule{
  \issymbolicallyevaluable(\vses \setminus \{\PerformsAssertions\}) \typearrow \vb
}{
  \shouldrememberimmutableexpression(\vses) \typearrow \vb
}
\end{mathpar}

\TypingRuleDef{AddImmutableExpr}
\hypertarget{def-addimmutableexpression}{}
The function
\[
\begin{array}{r}
\addimmutableexpression(
  \overname{\staticenvs}{\tenv} \aslsep
  \overname{\localdeclkeyword}{\ldk} \aslsep
  \overname{\langle\overname{\expr}{\ve}\times\overname{\TSideEffectSet}{\vsese}\rangle}{\veopt} \aslsep
  \overname{\identifier}{\vx}) \\
  \aslto \overname{\staticenvs}{\newtenv} \cup\ \TTypeError
\end{array}
\]
conditionally updates the static environment $\tenv$ for a \localdeclarationitem{} $\ldk$,
an optional pair $\veopt$ consisting
of an expression and its associated \sideeffectdescriptorsterm{},
and an identifier $\vx$,
yielding the updated static environment $\newtenv$.
More precisely, $\addimmutableexpression(\tenv, \ldk, \veopt, \vx)$
associates an expression with the identifier $\vx$
in the static environment $\tenv$, if one exists in $\veopt$ and it is \symbolicallyevaluable\ with
respect to the \sideeffectsetterm\ $\vsese$,
along with the local declaration keyword $\ldk$.
\ProseOtherwiseTypeError

\ExampleDef{Updating Static Environments for Immutable Expressions}
In \listingref{AddImmutableExpr}, determining that the declarations for \verb|t|
is well-typed succeeds since $\addimmutableexpression$ is used to bind \verb|x| to \verb|1|.
Similarly, determining that the declaration for \verb|y|
is well-typed succeeds since $\addimmutableexpression$ is used to bind \verb|k| to \verb|64|
and \verb|sub_k| to \verb|64|.
\ASLListing{Updating static environments for immutable expressions}{AddImmutableExpr}{\typingtests/TypingRule.AddImmutableExpr.asl}

See also \ExampleRef{Remembering vs. not Remembering Immutable Expressions} for an ill-typed
specification.

\ProseParagraph
\OneApplies
\begin{itemize}
  \item \AllApplyCase{ok}
  \begin{itemize}
    \item $\vep$ contains the expression $\ve$ and \sideeffectsetterm\ $\vsese$;
    \item $\ldk$ is $\LDKLet$;
    \item applying $\shouldrememberimmutableexpression$ to $\vsese$ yields $\True$;
    \item applying $\normalize$ to $\ve$ in $\tenv$ yields $\vep$\ProseOrTypeError;
    \item applying $\addlocalimmutableexpr$ to $\vx$ and $\ve$ yields $\newtenv$.
  \end{itemize}

  \item \AllApplyCase{fail}
  \begin{itemize}
    \item \OneApplies
    \begin{itemize}
      \item $\vep$ is $\None$;
      \item $\ldk$ is not $\LDKLet$;
      \item $\vep$ contains the expression $\ve$ and \sideeffectsetterm\ $\vsese$ and
            applying $\shouldrememberimmutableexpression$ to $\vsese$ yields $\True$;
    \end{itemize}
    \item define $\newtenv$ as $\tenv$.
  \end{itemize}
\end{itemize}

\FormallyParagraph
\begin{mathpar}
\inferrule[ok]{
  \shouldrememberimmutableexpression(\vsese) \typearrow \True\\
  \normalize(\tenv, \ve) \typearrow \vep \OrTypeError\\\\
  \addlocalimmutableexpr(\vx, \vep) \typearrow \newtenv
}{
  \addimmutableexpression(\tenv, \overname{\LDKLet}{\ldk}, \overname{\langle\ve, \vsese\rangle}{\veopt}, \vx) \typearrow \newtenv
}
\end{mathpar}

\begin{mathpar}
\inferrule[fail]{
  {
    \begin{array}{ll}
      \ldk \neq \LDKLet & \lor\\
      \veopt = \None & \lor\\
      \veopt = \langle\ve, \vsese\rangle \land \shouldrememberimmutableexpression(\vsese) \typearrow \False
    \end{array}
  }
}{
  \addimmutableexpression(\tenv, \ldk, \veopt, \vx) \typearrow \overname{\tenv}{\newtenv}
}
\end{mathpar}

\TypingRuleDef{AddSubprogram}
\hypertarget{def-addsubprogram}{}
The function
\[
  \addsubprogram(
    \overname{\staticenvs}{\tenv} \aslsep
    \overname{\Strings}{\name} \aslsep
    \overname{\func}{\funcdef} \aslsep
    \overname{\TSideEffectSet}{\vs})
   \aslto
  \overname{\staticenvs}{\newtenv}
\]
updates the global environment of $\tenv$ by mapping the (unique) subprogram identifier $\name$
to the function definition $\funcdef$ and \sideeffectdescriptorsterm\ $\vs$ in $\tenv$,
resulting in a new static environment $\newtenv$.

\ExampleDef{Updating a Static Environment for a Subprogram Declaration}
Consider \listingref{AddSubprogramDecls}
and assume that the \integertypeterm{} version of \increment{} is annotated
before the \realtypeterm{} version.
Then, annotating the \integertypeterm{} version of \increment{} applies
$\addsubprogram$ to the empty environment, \increment{} as a name,
the subprogram definition node
\[
\left\{
\begin{array}{rcl}
\funcname &:& \increment, \\
\funcparameters &:& \emptylist,\\
\funcargs &:& [(\vx, \unconstrainedinteger)],\\
\funcbody &:& \SReturn(\AbbrevEBinop{+}{\AbbrevEVar{\vx}}{\ELInt{1}}),\\
\funcreturntype &:& \Some{\unconstrainedinteger},\\
\funcsubprogramtype &:& \STFunction\\
\funcrecurselimit    &:& \None\\
\funcbuiltin &:& \False\\
\funcoverride &:& \None\\
\end{array}
\right\}
\]
and the empty set of \sideeffectdescriptorsterm.

Updating the \realtypeterm{} version of \increment{}
applies $\addsubprogram$ to the environment where the \integertypeterm{} version
of \increment{} is already bound, \incrementone{} as a name,
the subprogram definition node
\[
\left\{
\begin{array}{rcl}
\funcname &:& \incrementone, \\
\funcparameters &:& \emptylist,\\
\funcargs &:& [(\vx, \TReal)],\\
\funcbody &:& \SReturn(\AbbrevEBinop{+}{\AbbrevEVar{\vx}}{\ELiteral(\lreal(1/1))}),\\
\funcreturntype &:& \Some{\TReal},\\
\funcsubprogramtype &:& \STFunction\\
\funcrecurselimit    &:& \None\\
\funcbuiltin &:& \False\\
\funcoverride &:& \None\\
\end{array}
\right\}
\]
and the empty set of \sideeffectdescriptorsterm.

\ProseParagraph
Define $\newtenv$ as $\tenv$ with the $\subprograms$ map in the global component is updated by
binding $\name$ to $\funcdef$.

\FormallyParagraph
\begin{mathpar}
\inferrule{
  \newtenv \eqdef (G^\tenv.\subprograms[\name\mapsto(\funcdef, \vs)], L^\tenv)
}{
  \addsubprogram(\tenv, \name, \funcdef, \vs) \typearrow \newtenv
}
\end{mathpar}
\CodeSubsection{\AddSubprogramBegin}{\AddSubprogramEnd}{../StaticEnv.ml}

\TypingRuleDef{AddType}
\hypertarget{def-addtype}{}
The function
\[
  \addtype(
    \overname{\staticenvs}{\tenv} \aslsep
    \overname{\identifier}{\name} \aslsep
    \overname{\ty}{\tty} \aslsep
    \overname{\TTimeFrame}{\vf})
   \aslto
  \overname{\staticenvs}{\newtenv}
\]
binds the type $\tty$ and \timeframeterm\ $\vf$ to the identifier $\name$ in the static environment $\tenv$,
yielding the modified static environment $\newtenv$.

\ExampleDef{Adding Types to Environments}
In \listingref{DeclareType}, the declaration of the enumeration type \verb|Color|
is achieved via \TypingRuleRef{DeclareType}, where
$\addtype$ is applied to
the empty environment, \verb|Color| as the type name,
$\TEnum(\texttt{RED, GREEN, BLUE})$ as the type, and $\timeframeconstant$ as the \timeframeterm,
binding \verb|Color| to the pair $(\TEnum(\texttt{RED, GREEN, BLUE}), \timeframeconstant)$.

For the declaration of the type \verb|Record|,
$\addtype$ is applied to the environment where \verb|num_bits| has been added to the environment,
\verb|Record| as the type name,\\
$\TRecord([(\texttt{data},\TBits(\texttt{num\_bits}, \emptylist))])$ as the type,
and $\timeframeexecution$ as the \timeframeterm{} (since \verb|num_bits| is not a constant).

\ProseParagraph
Define $\newtenv$ as $\tenv$ where the $\declaredtypes$ map of the global component is updated
by binding $\name$ to $\tty$ and $\vf$.

\FormallyParagraph
\begin{mathpar}
\inferrule{}{
  \addtype(\tenv, \name, \tty, \vf) \typearrow \overname{(G^\tenv.\declaredtypes[\name\mapsto(\tty, \vf)] , L^\tenv)}{\newtenv}
}
\end{mathpar}
\CodeSubsection{\AddTypeBegin}{\AddTypeEnd}{../StaticEnv.ml}
