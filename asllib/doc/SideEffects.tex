\chapter{Side Effects\label{chap:SideEffects}}

This chapter defines a static \emph{side effect analysis}.
The analysis aims to answer which expressions are \emph{pure} or \symbolicallyevaluable{}.
For purity, the analysis is \emph{sound}: it proves that a \underline{sufficient condition} for purity holds.

Intuitively, a pure expression is one whose evaluation does not affect the evaluation of any further expressions.
In other words, a pure expression can be evaluated multiple times (or not at all) with no effect on the overall specification.
A \symbolicallyevaluable{} expression is compatible with symbolic reduction and equivalence testing (\chapref{SymbolicEquivalenceTesting}).

%%%%%%%%%%%%%%%%%%%%%%%%%%%%%%%%%%%%%%%%%%%%%%%%%%%%%%%%%%%%%%%%%%%%%%%%%%%%%%%%
\section{Time Frames\label{sec:TimeFrames}}
%%%%%%%%%%%%%%%%%%%%%%%%%%%%%%%%%%%%%%%%%%%%%%%%%%%%%%%%%%%%%%%%%%%%%%%%%%%%%%%%
\hypertarget{def-timeframe}{}
We divide side effects by \emph{\timeframesterm}, which indicate the phase where a side effect occurs:
\begin{description}
    \item[Constant] Contains effects that take place during static evaluation (see \chapref{StaticEvaluation}). That is, during typechecking.
    \item[Execution] Contains effects that take place during semantic evaluation.
\end{description}

Formally, \timeframesterm\ are totally ordered via $\timeframeless$ as follows:
\hypertarget{def-timeframetype}{}
\hypertarget{def-timeframeless}{}
\hypertarget{def-timeframeconstant}{}
\hypertarget{def-timeframeexecution}{}
\[
\TTimeFrame \triangleq \{ \timeframeconstant \timeframeless \timeframeexecution \}
\]
Additionally, we define the less-than-or-equal ordering as follows:
\hypertarget{def-timeframeleq}{}
\[
f \timeframeleq f' \triangleq f \timeframeless f' \lor f = f' \enspace.
\]

We now define some helper functions for constructing time frames.

\hypertarget{def-timeframemax}{}
We define the maximum of a set of time frames $\timeframemax : \pow{\TTimeFrame} \rightarrow \TTimeFrame$
as follows:
\[
    \timeframemax(T) \triangleq t \text{ such that } t\in T \land \forall t'\in T.\ t' \timeframeleq t \enspace.
\]

\TypingRuleDef{TimeFrameLDK}
\hypertarget{def-timeframeofldk}{}
The function
\[
    \timeframeofldk(\overname{\localdeclkeyword}{\ldk}) \aslto \overname{\TTimeFrame}{\vt}
\]
constructs a \timeframeterm\ $\vt$ from a local declaration keyword $\ldk$.

\ProseParagraph
\ProseEqdef{$\vt$}{$\timeframeconstant$ if $\ldk$ is $\LDKConstant$, and
    $\timeframeexecution$ otherwise}.

\FormallyParagraph
\begin{mathpar}
\inferrule{}{
    {
        \timeframeofldk(\ldk) \typearrow
    \begin{cases}
        \timeframeconstant & \text{if }\ldk = \LDKConstant\\
        \timeframeexecution & \text{if }\ldk = \LDKLet\\
        \timeframeexecution & \text{if }\ldk = \LDKVar\\
    \end{cases}
    }
}
\end{mathpar}

\TypingRuleDef{TimeFrameGDK}
\hypertarget{def-timeframeofgdk}{}
The function
\[
    \timeframeofgdk(\overname{\globaldeclkeyword}{\gdk}) \aslto \overname{\TTimeFrame}{\vt}
\]
constructs a \timeframeterm\ $\vt$ from a global declaration keyword $\gdk$.

\ProseParagraph
\ProseEqdef{$\vt$}{$\timeframeconstant$ if $\gdk$ is $\GDKConstant$,
    $\timeframeexecution$ if $\gdk$ is $\GDKConfig$,
    $\timeframeexecution$ if $\gdk$ is $\GDKLet$, and
    $\timeframeexecution$ if $\gdk$ is $\GDKVar$.
}

\FormallyParagraph
\begin{mathpar}
\inferrule{}{
    {
        \timeframeofgdk(\gdk) \typearrow
    \begin{cases}
        \timeframeconstant  & \text{if }\gdk = \GDKConstant\\
        \timeframeexecution & \text{if }\gdk = \GDKConfig\\
        \timeframeexecution & \text{if }\gdk = \GDKLet\\
        \timeframeexecution & \text{if }\gdk = \GDKVar\\
    \end{cases}
    }
}
\end{mathpar}

%%%%%%%%%%%%%%%%%%%%%%%%%%%%%%%%%%%%%%%%%%%%%%%%%%%%%%%%%%%%%%%%%%%%%%%%%%%%%%%%
\section{Side Effect Descriptors\label{sec:SideEffectDescriptors}}
%%%%%%%%%%%%%%%%%%%%%%%%%%%%%%%%%%%%%%%%%%%%%%%%%%%%%%%%%%%%%%%%%%%%%%%%%%%%%%%%

\hypertarget{def-sideeffectdescriptorterm}{}
We now define \sideeffectdescriptorsterm,
which are configurations used to describe side effects, as explained below:
\hypertarget{def-tsideeffect}{}
\hypertarget{def-readlocal}{}
\[
\TSideEffect \triangleq \left\lbrace
\begin{array}{ll}
    \ReadLocal(\overname{\identifier}{\vx}, \overname{\TTimeFrame}{\vt}, \overname{\Bool}{\vimmutable})     & \cup
    \hypertarget{def-writelocal}{}\\
    \WriteLocal(\overname{\identifier}{\vx})                                            & \cup
    \hypertarget{def-readglobal}{}\\
    \ReadGlobal(\overname{\identifier}{\vx}, \overname{\TTimeFrame}{\vt}, \overname{\Bool}{\vimmutable})    & \cup
    \hypertarget{def-writeglobal}{}\\
    \WriteGlobal(\overname{\identifier}{\vx})                                           & \cup
    \hypertarget{def-throwexception}{}\\
    \ThrowException(\overname{\identifier}{\vx})                                        & \cup
    \hypertarget{def-recursivecall}{}\\
    \RecursiveCall(\overname{\identifier}{\vf})                                         & \cup
    \hypertarget{def-performsassertions}{}\\
    \PerformsAssertions                                                 & \cup
    \hypertarget{def-nondeterministic}{}\\
    \NonDeterministic                                                   &
\end{array} \right.
\]
\hypertarget{def-readlocalterm}{}
\begin{description}
    \item[$\ReadLocal$] a \ReadLocalTerm\ describes an evaluation of a construct that leads to reading the value of the local storage element
        $\vx$ at the \timeframeterm\ $\vt$ where $\vimmutable$ is $\True$ if and only if $\vx$
        was declared as an immutable local storage element (that is, \texttt{constant} or \texttt{let});
    \hypertarget{def-writelocalterm}{}
    \item[$\WriteLocal$] a \WriteLocalTerm\ describes an evaluation of a construct that leads to modifying the value of the local storage element
        $\vx$;
    \hypertarget{def-readglobalterm}{}
    \item[$\ReadGlobal$] a \ReadGlobalTerm\ describes an evaluation of a construct that leads to reading the value of the global storage element
        $\vx$ at the \timeframeterm\ $\vt$ where $\vimmutable$ is $\True$ if and only if $\vx$
        was declared as an immutable local storage element (that is, \texttt{constant}, \texttt{config}, or \texttt{let});
    \hypertarget{def-writeglobalterm}{}
    \item[$\WriteGlobal$] a \WriteGlobalTerm\ describes an evaluation of a construct that leads to modifying the value of the global storage element
        $\vx$;
    \hypertarget{def-throwexceptionterm}{}
    \item[$\ThrowException$] an \ThrowExceptionTerm\ describes an evaluation of a construct that leads to raising an exception whose type
        is named $\vx$;
    \hypertarget{def-recursivecallterm}{}
    \item[$\RecursiveCall$] a \RecursiveCallTerm\ describes an evaluation of a construct that leads to calling the recursive function $\vf$;
    \hypertarget{def-performsassertionsterm}{}
    \item[$\PerformsAssertions$] a \PerformsAssertionsTerm\ describes an evaluation of a construct that leads to evaluating an \texttt{assert} statement;
    \hypertarget{def-nondeterministicterm}{}
    \item[$\NonDeterministic$] a \NonDeterministicTerm\ describes an evaluation of a construct that leads to evaluating a non-deterministic
        expression (either \ARBITRARY\ or a library function call known to be non-deterministic).
\end{description}

We now define a few helper functions over \timeframesterm.

\TypingRuleDef{TimeFrame}
\hypertarget{def-sideeffecttimeframe}{}
The function
\[
    \timeframe(\overname{\TSideEffect}{\vs}) \aslto \overname{\TTimeFrame}{\vt}
\]
retrieves the \timeframeterm\ $\vt$ from a \sideeffectdescriptorterm\ $\vs$.

\ProseParagraph
\OneApplies
\begin{itemize}
    \item \AllApplyCase{read\_local}
    \begin{itemize}
        \item $\vs$ is a \ReadLocalTerm{} for the \timeframeterm{} $\vt$.
    \end{itemize}

    \item \AllApplyCase{read\_global}
    \begin{itemize}
        \item $\vs$ is a \ReadGlobalTerm{} for the \timeframeterm{} $\vt$.
    \end{itemize}

    \item \AllApplyCase{performs\_assertions}
    \begin{itemize}
        \item $\vs$ is a \PerformsAssertionsTerm{};
        \item \Proseeqdef{$\vt$}{$\timeframeconstant$}.
    \end{itemize}

    \item \AllApplyCase{other}
    \begin{itemize}
        \item $\vs$ is either a \WriteLocalTerm{}, a \WriteGlobalTerm{},
                a \NonDeterministicTerm{}, a \RecursiveCallTerm{}, or a
                \ThrowExceptionTerm{};
        \item \Proseeqdef{$\vt$}{$\timeframeexecution$}.
    \end{itemize}
\end{itemize}

\FormallyParagraph
\begin{mathpar}
\inferrule[read\_local]{}{
    \timeframe(\overname{\ReadLocal(\Ignore, \vt, \Ignore)}{\vs}) \typearrow \vt
}
\and
\inferrule[read\_global]{}{
    \timeframe(\overname{\ReadGlobal(\Ignore, \vt, \Ignore)}{\vs}) \typearrow \vt
}
\end{mathpar}

\begin{mathpar}
\inferrule[performs\_assertions]{}{
    \timeframe(\overname{\PerformsAssertions}{\vs}) \typearrow \overname{\timeframeconstant}{\vt}
}
\end{mathpar}

\begin{mathpar}
\inferrule[other]{
    {
    \configdomain{\vs} \in \left\{
        \begin{array}{c}
            \WriteLocal, \\
            \WriteGlobal, \\
            \NonDeterministic, \\
            \RecursiveCall, \\
            \ThrowException
        \end{array}
    \right\}
    }
}{
    \timeframe(\vs) \typearrow \overname{\timeframeexecution}{\vt}
}
\end{mathpar}

\TypingRuleDef{SideEffectIsPure}
\hypertarget{def-sideeffectispure}{}
\[
    \sideeffectispure(\overname{\TSideEffect}{\vs}) \aslto \overname{\Bool}{\vb}
\]
defines whether a \sideeffectdescriptorsterm\ $\vs$ is considered \emph{pure},
yielding the result in $\vb$.
Intuitively, a \emph{pure} \sideeffectdescriptorterm\ helps to establish that
an expression evaluates without modifying values of storage elements.

\ProseParagraph
Define $\vb$ as $\True$ if and only if $\vt$ is either a \ReadLocalTerm, a \ReadGlobalTerm,
a \NonDeterministicTerm, or a \PerformsAssertionsTerm.

\FormallyParagraph
\begin{mathpar}
\inferrule{
    \vb \eqdef \configdomain{\vs} \in \{\ReadLocal, \ReadGlobal, \NonDeterministic, \PerformsAssertions\}
}{
    \sideeffectispure(\vt) \typearrow \vb
}
\end{mathpar}

\TypingRuleDef{SideEffectIsSymbolicallyEvaluable}
\hypertarget{def-sideeffectissymbolicallyevaluable}{}
\[
    \sideeffectissymbolicallyevaluable(\overname{\TSideEffect}{\vs}) \aslto \overname{\Bool}{\vb}
\]
defines whether a \sideeffectdescriptorsterm\ $\vs$ is considered \emph{\symbolicallyevaluable},
yielding the result in $\vb$.
Intuitively, a \emph{symbolically evaluable} \sideeffectdescriptorterm\ helps establish that
an expression evaluates without failing assertions, without modifying any storage element,
and always yielding the same result, that is, deterministically.

\ProseParagraph
Define $\vb$ as $\True$ if and only if $\vs$ is either
a \ReadLocalTerm\ associated with an immutable storage element, or
a \ReadGlobalTerm\ associated with an immutable storage element.

\FormallyParagraph
\begin{mathpar}
\inferrule{
    \vb \eqdef \vs = \ReadLocal(\Ignore, \Ignore, \True) \lor \vs = \ReadGlobal(\Ignore, \Ignore, \True)
}{
    \sideeffectissymbolicallyevaluable(\vs) \typearrow \vb
}
\end{mathpar}

\TypingRuleDef{LDKIsImmutable}
\hypertarget{def-ldkisimmutable}{}
The function
\[
\ldkisimmutable(\overname{\localdeclkeyword}{\ldk}) \typearrow \overname{\True}{\vb}
\]
tests whether the local declaration keyword $\ldk$ relates to an immutable storage element,
yielding the result in $\vb$.

\ProseParagraph
Define $\vb$ as $\True$ if and only if $\ldk$ corresponds to either the keyword \texttt{constant} or
the keyword \texttt{let}.

\FormallyParagraph
\begin{mathpar}
\inferrule{}{
  \ldkisimmutable(\ldk) \typearrow \overname{\ldk \in \{\LDKConstant, \LDKLet\}}{\vb}
}
\end{mathpar}

\TypingRuleDef{GDKIsImmutable}
\hypertarget{def-gdkisimmutable}{}
The function
\[
\gdkisimmutable(\overname{\globaldeclkeyword}{\gdk}) \typearrow \overname{\True}{\vb}
\]
tests whether the global declaration keyword $\gdk$ relates to an immutable storage element,
yielding the result in $\vb$.

\ProseParagraph
Define $\vb$ as $\True$ if and only if $\gdk$ corresponds to either the keyword \texttt{constant},
the keyword \texttt{config}, or the keyword \texttt{let}.

\FormallyParagraph
\begin{mathpar}
\inferrule{}{
  \gdkisimmutable(\gdk) \typearrow \overname{\gdk \in \{\GDKConstant, \GDKConfig, \GDKLet\}}{\vb}
}
\end{mathpar}

%%%%%%%%%%%%%%%%%%%%%%%%%%%%%%%%%%%%%%%%%%%%%%%%%%%%%%%%%%%%%%%%%%%%%%%%%%%%%%%%
\section{Side Effect Sets\label{sec:SideEffectSets}}
%%%%%%%%%%%%%%%%%%%%%%%%%%%%%%%%%%%%%%%%%%%%%%%%%%%%%%%%%%%%%%%%%%%%%%%%%%%%%%%%

\TypingRuleDef{MaxTimeFrame}
\hypertarget{def-maxtimeframe}{}
The function
\[
    \maxtimeframe(\overname{\TSideEffectSet}{\vses}) \aslto \overname{\TTimeFrame}{\vf}
\]
defines the maximal \timeframeterm\ for a \sideeffectsetterm\ $\vses$, which is returned
in $\vf$.
(If $\vses$ is empty, the result is \timeframeconstant.)

\ProseParagraph
\OneApplies
\begin{itemize}
    \item \AllApplyCase{execution}
    \begin{itemize}
        \item there exists a \sideeffectdescriptorterm\ in $\vses$ that is either
            a \WriteLocalTerm, a \WriteGlobalTerm, an \ThrowExceptionTerm, a \RecursiveCallTerm, or
            a \NonDeterministicTerm;
        \item define $\vf$ as \timeframeexecution.
    \end{itemize}

    \item \AllApplyCase{reads}
    \begin{itemize}
        \item define $\vreads$ as the subset of $\vses$ that contains only
            \sideeffectdescriptorsterm\ that are either \ReadLocalTerm\ or \ReadGlobalTerm;
        \item $\vreads$ is equal to $\vses$;
        \item define $\vtimeframes$ as the \timeframesterm\ appearing in the \sideeffectdescriptorsterm\
            in $\vreads$;
        \item define $\vf$ as the greatest time frame in the union of $\vtimeframes$ and the singleton set for \timeframeconstant,
              where $\timeframeleq$ is used to compare any two \timeframesterm.
    \end{itemize}
\end{itemize}

\FormallyParagraph
\begin{mathpar}
\inferrule[execution]{
    {
    \exists \vs \in \vses.\ \configdomain{\vs} \in \left\{
        \begin{array}{c}
            \WriteLocal, \\
            \WriteGlobal, \\
            \ThrowException, \\
            \RecursiveCall, \\
            \NonDeterministic
        \end{array}
        \right\}
    }
}{
    \maxtimeframe(\vses) \typearrow \overname{\timeframeexecution}{\vf}
}
\end{mathpar}

\begin{mathpar}
\inferrule[reads]{
    \vreads \eqdef \{ \vs \in \vses \;|\; \configdomain{\vs} \in \{\ReadLocal, \ReadGlobal\} \}\\
    \vses = \vreads\\
    \vtimeframes \eqdef \{ \timeframe(\vfp) \;|\; \vfp\in\vreads \} \\
    \vf \eqdef \timeframemax(\vtimeframes \cup \{\timeframeconstant\})
}{
    \maxtimeframe(\vses) \typearrow \vf
}
\end{mathpar}

\TypingRuleDef{SESIsSymbolicallyEvaluable}
\hypertarget{def-issymbolicallyevaluable}{}
\hypertarget{def-symbolicallyevaluable}{}
The function
\[
  \issymbolicallyevaluable(\overname{\TSideEffectSet}{\vses}) \aslto \overname{\Bool}{\bv}
\]
tests whether a set of \sideeffectdescriptorsterm\ $\vses$ are all \symbolicallyevaluable,
yielding the result in $\vb$.

\ProseParagraph
Define $\vb$ as $\True$ if and only if every \sideeffectdescriptorterm\ $\vs$ in $\vses$
is \symbolicallyevaluable.

\FormallyParagraph
\begin{mathpar}
\inferrule{
  \vb \eqdef \bigwedge_{\vs\in\vses} \sideeffectissymbolicallyevaluable(\vs)
}{
  \issymbolicallyevaluable(\vses) \typearrow \vb
}
\end{mathpar}

\hypertarget{def-checksymbolicallyevaluable}{}
\TypingRuleDef{CheckSymbolicallyEvaluable}
The function
\[
  \checksymbolicallyevaluable(\overname{\TSideEffectSet}{\vses}) \aslto
  \{\True\} \cup \TTypeError
\]
returns $\True$ if the set of \sideeffectdescriptorsterm\ $\vses$ is \symbolicallyevaluable.
\ProseOtherwiseTypeError

\ProseParagraph
\AllApply
\begin{itemize}
  \item applying $\issymbolicallyevaluable$ to $\ve$ in $\tenv$ yields $\vb$;
  \item the result is $\True$ if $\vb$ is $\True$, otherwise it is a \typingerrorterm{} indicating that the expression
  is not \symbolicallyevaluable.
\end{itemize}

\FormallyParagraph
\begin{mathpar}
\inferrule{
  \issymbolicallyevaluable(\vses) \typearrow \vb\\
  \checktrans{\vb}{NotSymbolicallyEvaluable} \checktransarrow \True \OrTypeError
}{
  \checksymbolicallyevaluable(\vses) \typearrow \True
}
\end{mathpar}
\CodeSubsection{\CheckSymbolicallyEvaluableBegin}{\CheckSymbolicallyEvaluableEnd}{../Typing.ml}

\TypingRuleDef{SESIsPure}
\hypertarget{def-sesispure}{}
The function
\[
    \sesispure(\overname{\TSideEffectSet}{\vses}) \aslto \overname{\Bool}{\vb}
\]
tests whether all side effects in the set $\vses$ are pure, yielding the result in $\vb$.

\ProseParagraph
Define $\vb$ as $\True$ if and only if $\sideeffectispure$ holds for
every \sideeffectdescriptorterm\ $\vs$ in $\vses$.

\FormallyParagraph
\begin{mathpar}
\inferrule{
    \bigwedge_{\vs\in\vses} \sideeffectispure(\vs)
}{
    \sesispure(\vses) \typearrow \True
}
\end{mathpar}

\TypingRuleDef{SESIsDeterministic}
\hypertarget{def-sesisdeterministic}{}
The function
\[
  \sesisdeterministic(\overname{\TSideEffectSet}{\vses}) \aslto \overname{\Bool}{\bv}
\]
tests whether the \NonDeterministic\ \sideeffectdescriptorterm\ is not included in $\vses$,
yielding the result in $\vb$.

\ProseParagraph
Define $\vb$ as $\True$ if and only if \NonDeterministic\ is not included in $\vses$.

\FormallyParagraph
\begin{mathpar}
\inferrule{}{
  \sesisdeterministic(\vses) \typearrow \overname{\NonDeterministic \not\in \vses}{\vb}
}
\end{mathpar}

\TypingRuleDef{SESIsBefore}
\hypertarget{def-sesisbefore}{}
The function
\[
  \sesisbefore(\overname{\TSideEffectSet}{\vses} \aslsep \overname{\TTimeFrame}{\vt}) \aslto \overname{\Bool}{\vb}
\]
tests whether the \timeframesterm\ of \sideeffectdescriptorsterm\ in $\vses$ are all less than or equal to the \timeframeterm\
$\vt$, yielding the result in $\vb$.
% Transliteration note: this abstracts leq_constant_time

\ProseParagraph
Define $\vb$ as $\True$ if and only if the maximal \timeframeterm\ of all \sideeffectdescriptorsterm\ in $\vses$
is less than or equal to $\vt$ with respect to $\timeframeleq$.

\FormallyParagraph
\begin{mathpar}
\inferrule{}{
    \sesisbefore(\vses, \vt) \typearrow \overname{\maxtimeframe(\vses) \timeframeleq \vt}{\vb}
}
\end{mathpar}
