\chapter{Errors\label{chap:Errors}}

This chapter describes the errors defined for ASL.
%
An error denotes the presence of a bug in an ASL specification,
and the ASL Reference mandates the different types of errors
(corresponding to different types of bugs) that implementations must detect,
including when these errors must be detected and reported to users.

\ChapterOutline
\begin{itemize}
  \item \secref{How Implementations Should Handle Errors} describes how implementations of ASL
        should handle errors;
  \item \secref{Error Kinds} defines a classification of ASL errors;
  \item \secref{ErrorCodesSummary} summarises the error codes used throughout this reference;
  \item \secref{BuildErrors} details \builderrorsterm;
  \item \secref{TypingErrors} details \typingerrorsterm;
  \item \secref{DynamicErrors} details \dynamicerrorsterm.
\end{itemize}

\section{How Implementations Should Handle Errors\label{sec:How Implementations Should Handle Errors}}

\RequirementDef{StaticErrorCheck} Implementations should detect and report \staticerrorsterm. \\
Specifically, \staticerrorsterm{} must never cause a \dynamicerrorterm{} or
cause an exception to be raised.
%
\secref{Example Interpreter} shows how an interpreter detects \builderrorsterm{}
and \typingerrorsterm{}.
%
\listingref{typing-error-reporting} shows a specification containing a \typingerrorterm,
followed by an example of a report from \aslref{} in a \linuxbashshell.

\ASLListing{A specification resulting in a type error}{typing-error-reporting}{\definitiontests/TypingErrorReporting.asl}
% CONSOLE_BEGIN CONSOLE_STDERR aslref \definitiontests/TypingErrorReporting.asl
\begin{Verbatim}[fontsize=\footnotesize, frame=single]
File ../tests/ASLDefinition.t/TypingErrorReporting.asl, line 3,
  characters 11 to 22:
    return 5 + "hello";
           ^^^^^^^^^^^
ASL Type error: Illegal application of operator + on types integer {5}
  and string.
\end{Verbatim}
% CONSOLE_END

\RequirementDef{DynamicErrorBehavior} The behaviour of an implementation when a \dynamicerrorterm{}
occurs is implementation-defined, including, but not limited to,
(incorrect) behavior not following the dynamic semantics, termination of execution or raising an exception.
%
Implementations should detect and report \dynamicerrorsterm{} on a ``best effort'' basis
(that is, they are not required to detect and/or report \dynamicerrorsterm{}).
%
If an implementation raises an exception in response to a \dynamicerrorterm{},
the exception must have the \supertypeterm{}
\verb|runtime_exception|.
%
\listingref{dynamic-error-reporting} shows a specification containing a \dynamicerrorterm{},
followed by an example of a report from \aslref{} in a \linuxbashshell.

\ASLListing{A specification resulting in a dynamic error}{dynamic-error-reporting}{\definitiontests/DynamicErrorReporting.asl}

% CONSOLE_BEGIN CONSOLE_STDERR CONSOLE_CMD aslref \definitiontests/DynamicErrorReporting.asl
\begin{Verbatim}[fontsize=\footnotesize, frame=single]
> aslref ../tests/ASLDefinition.t/DynamicErrorReporting.asl
ASL Dynamic error: Illegal application of operator DIV for values 128 and 7.
\end{Verbatim}
% CONSOLE_END

\RequirementDef{DynamicErrorHost}
If an implementation terminates execution in response to a \dynamicerrorterm{},
it should signal an error to the hosting environment.
%
Applying \aslref{} in a \linuxbashshell{} on the specification in \listingref{dynamic-error-reporting}
yields the error status \texttt{1}.

\RequirementDef{DynamicErrorAssert}
An assertion failure arising from an \assertionstatementterm\ is a \dynamicerrorterm{}
(see \SemanticsRuleRef{SAssert} and \errorcodeterm{} $\DynamicAssertionFailure$).
%
\listingref{AssertionStatement} shows an example of a specification failing
an \assertionstatementterm{}.
The report from \aslref{} in a \linuxbashshell{} is shown next:
% CONSOLE_BEGIN CONSOLE_STDERR aslref \definitiontests/AssertionStatement.asl
\begin{Verbatim}[fontsize=\footnotesize, frame=single]
File ../tests/ASLDefinition.t/AssertionStatement.asl, line 5,
  characters 11 to 22:
    assert a + b < 256;
           ^^^^^^^^^^^
ASL Dynamic error: Assertion failed: ((a + b) < 256).
\end{Verbatim}
% CONSOLE_END

\RequirementDef{DynamicErrorUnreachable}
Evaluation of the \unreachablestatementterm{} is a \dynamicerrorterm{}
(see \SemanticsRuleRef{SUnreachable} and \errorcodeterm{} $\UnreachableError$).
%
\listingref{UnreachableStatement} shows an example specification that
fails with a \dynamicerrorsterm{} due to evaluation of an \unreachablestatementterm{}.
The report from \aslref{} in a \linuxbashshell{} is shown next:

% CONSOLE_BEGIN CONSOLE_STDERR aslref \definitiontests/UnreachableStatement.asl
\begin{Verbatim}[fontsize=\footnotesize, frame=single]
diagnostic assertion failed: example message
File ../tests/ASLDefinition.t/UnreachableStatement.asl, line 5,
  characters 8 to 20:
        unreachable;
        ^^^^^^^^^^^^
ASL Dynamic error: unreachable reached.
\end{Verbatim}
% CONSOLE_END

\section{Error Kinds\label{sec:Error Kinds}}
\hypertarget{def-errorcodeterm}{}
Each type of error has an \emph{\errorcodeterm}, which uniquely identifies it
(see \secref{ErrorCodesSummary} for the full list of \errorcodesterm),
a description of what the error means,
and a \emph{kind}, which dictates in which phase it is detected and reported.
Error codes must be consistent across implementations, but each implementation
can define its own appropriate error messages.

ASL includes the following error kinds:
\begin{description}
  \item[Static Errors] \hypertarget{def-staticerrorterm}{}
  A \emph{\staticerrorterm} is detected by inspecting a specification without evaluating it.
  That is, across all possible executions. \staticerrorsterm{} can be further classified:
  \begin{description}
    \item[Build Errors] \hypertarget{def-builderrorterm}{}
      Detected and reported during lexical analysis, parsing, and building of AST.
      \Builderrorsterm{} are always detected and reported.
      Their error codes, listed in \secref{BuildErrors}, are prefixed with \BuildErrorPrefix.

    \item[Type Errors] \hypertarget{def-typingerrorterm}{}
      Detected and reported during typechecking.
      \Typingerrorsterm{} are always detected and reported, even if the part of the specification that causes them is never executed.
      Their error codes, listed in \secref{TypingErrors}, are prefixed with \TypeErrorPrefix.
  \end{description}

  \item[Dynamic Errors] \hypertarget{def-dynamicerrorterm}{}
    Detected and reported during execution, if and only if the part of the specification that causes them is executed.
    Their error codes, listed in \secref{DynamicErrors}, are prefixed with \DynamicErrorPrefix.
\end{description}

\Builderrorsterm{} and \typingerrorsterm{} are known collectively as \staticerrorsterm.

\section{Error Codes Summary\label{sec:ErrorCodesSummary}}
The following table summarises all error codes.

\begin{center}
\begin{tabular}{clc}
\hline
\textbf{Code} & \textbf{Name} & \textbf{Kind} \\
\hline
  $\LexicalError$                & Lexical error                  & Build         \\
  $\ParseError$                  & Parse error                    & "             \\
  $\ReservedIdentifier$          & Reserved identifier            & "             \\
  $\BinopPrecedence$             & Binary operation precedence    & "             \\
  $\BuildBadDeclaration$         & Bad declaration                & "             \\
  \hline
  $\UndefinedIdentifier$         & Undefined identifier           & Typing        \\
  $\IdentifierAlreadyDeclared$   & Identifier already declared    & "             \\
  $\AssignmentToImmutable$       & Assign to immutable            & "             \\
  $\TypeSatisfactionFailure$     & Type satisfaction failure      & "             \\
  $\NoLCA$                       & Lowest common ancestor         & "             \\
  $\NoBaseValue$                 & No base value                  & "             \\
  $\TypeAssertionFailure$        & Type assertion failure         & "             \\
  $\StaticEvaluationFailure$     & Static evaluation failure      & "             \\
  $\BadOperands$                 & Bad operands                   & "             \\
  $\UnexpectedType$              & Unexpected type                & "             \\
  $\BadTupleIndex$               & Bad tuple index                & "             \\
  $\BadSlices$                   & Bad slices                     & "             \\
  $\BadField$                    & Bad field                      & "             \\
  $\BadSubprogramDeclaration$    & Bad subprogram declaration     & "             \\
  $\BadDeclaration$              & Bad declaration                & "             \\
  $\BadCall$                     & Bad call                       & "             \\
  $\SideEffectViolation$         & Side effect violation          & "             \\
  $\OverridingError$             & Overriding error               & "             \\
  $\PrecisionLostDefining$       & Declaration with an imprecise type & "         \\
  \hline
  $\UnreachableError$            & Unreachable error              & Dynamic       \\
  $\DynamicTypeAssertionFailure$ & Dynamic type assertion failure & "             \\
  $\ArbitraryEmptyType$          & \ARBITRARY{} empty type        & "             \\
  $\DynamicBadOperands$          & Bad operands                   & "             \\
  $\LimitExceeded$               & Limit exceeded                 & "             \\
  $\UncaughtException$           & Uncaught exception             & "             \\
  $\BadIndex$                    & Bad index                      & "             \\
  $\OverlappingSliceAssignment$  & Overlapping slice assignment   & "             \\
  $\NegativeArrayLength$         & Negative array length          & "             \\
\end{tabular}
\end{center}

\section{Build Errors\label{sec:BuildErrors}}
\begin{description}
  \item[$\LexicalError$]
    \textit{Lexical error.}
    An error was encountered during lexical analysis.
    See \nameref{sec:TopLevelRule.CheckAndInterpret} for an example.

  \item[$\ParseError$]
    \textit{Parse error.}
    An error was encountered during parsing. \\
    See \nameref{sec:TopLevelRule.CheckAndInterpret} for an example.

  \hypertarget{def-reservedidentifier}{}
  \item[$\ReservedIdentifier$]
    \textit{Reserved identifier.}
    A reserved identifier was used. \\
    See \LexicalRuleRef{ReservedIdentifiers}.

  \hypertarget{def-binopprecedence}{}
  \item[$\BinopPrecedence$]
    \textit{Binary operation precedence.}
    A compound binary expression consisting of two associative binary operators of the same precedence was used without sufficient parentheses.
    See \ASTRuleRef{CheckNotSamePrec}.

  \hypertarget{def-buildbaddeclaration}{}
  \item[$\BuildBadDeclaration$]
    \textit{Bad declaration.}
    A top-level declaration is invalid.
    For example, the standard library defines a non-subprogram (\ASTRuleRef{SetBuiltin}).
\end{description}

\section{Type Errors\label{sec:TypingErrors}}
\begin{description}

\hypertarget{def-undefinedidentifier}{}
\item[$\UndefinedIdentifier$]
  \textit{Undefined identifier.}
  An identifier is missing a definition of the appropriate kind.
  See \TypingRuleRef{SubprogramForSignature} for an example.

\hypertarget{def-identifieralreadydeclared}{}
\item[$\IdentifierAlreadyDeclared$]
  \textit{Identifier already declared.}
  An attempt to declare an identifier which has already been defined.
  For example:
  \begin{itemize}
    \item Re-defining a local variable (see the use of \TypingRuleRef{CheckVarNotInEnv} in \TypingRuleRef{LDVar}).
    \item Re-defining a global variable (see the use of \TypingRuleRef{CheckVarNotInGEnv} in \TypingRuleRef{DeclareGlobalStorage}).
  \end{itemize}

\hypertarget{def-aim}{}
\item[$\AssignmentToImmutable$]
  \textit{Assign to immutable.}
  An assignment has an immutable storage element on its left-hand side.
  See \TypingRuleRef{LEVar}.

\hypertarget{def-typesatisfactionfailure}{}
\item[$\TypeSatisfactionFailure$]
  \textit{Type satisfaction failure.}
  See \TypingRuleRef{TypeSatisfaction}.

\hypertarget{def-staticevaluationfailure}{}
\item[$\StaticEvaluationFailure$]
  \textit{Static evaluation failure.}
  Static evaluation did not produce a literal.
  See \TypingRuleRef{StaticEval}.

\hypertarget{def-nolca}{}
\item[$\NoLCA$]
  \textit{Lowest common ancestor.}
  The two branches of a conditional expression have types with no common ancestor.
  See \TypingRuleRef{LowestCommonAncestor}.

\hypertarget{def-nobasevalue}{}
\item[$\NoBaseValue$]
  \textit{No base value.}
  A \basevalueterm{} for a given type cannot be constructed, either because one cannot be statically inferred from the type or because the static domain of the type is empty
  See \TypingRuleRef{BaseValue}.

\hypertarget{def-typeassertionfailure}{}
\item[$\TypeAssertionFailure$]
  \textit{Type assertion failure.}
  An asserting type conversion must always fail dynamically.
  See \TypingRuleRef{CheckATC}.

\hypertarget{def-badoperands}{}
\item[$\BadOperands$]
  \textit{Bad operands.}
  A primitive operator was provided with invalid operands during typechecking.
  For example:
  \begin{itemize}
    \item The operands had the wrong types \\
      (\TypingRuleRef{ApplyUnopType}, \TypingRuleRef{ApplyBinopTypes}).
    \item Static evaluation of a primitive operator encountered an error \\
      (\TypingRuleRef{UnopLiterals}, \TypingRuleRef{BinopLiterals}).
    \item The operator must always fail dynamically, because the type of one of its operands is empty (\TypingRuleRef{BinopFilterRhs}).
  \end{itemize}

\hypertarget{def-unexpectedtype}{}
\item[$\UnexpectedType$]
  \textit{Unexpected type.}
  In a context where a particular type was required, another one was found instead.
  For example:
  \begin{itemize}
    \item Expected a constrained integer, found an unconstrained one \\
      (\TypingRuleRef{CheckConstrainedInteger}).
    \item Expected integer types in \texttt{for}-loop bounds (\TypingRuleRef{SForConstraints}).
    \item Expected a bitvector (\TypingRuleRef{ApplyBinopTypes}).
    \item Expected a structured type (\TypingRuleRef{ERecord}).
    \item Expected a \tupletypeterm{} of a specific length (\TypingRuleRef{LEDestructuring}).
    \item Expected a printable type (\TypingRuleRef{SPrint}).
    \item Encountered a forbidden \pendingconstrainedintegertype{} (\TypingRuleRef{TInt}).
    \item An anonymous enumeration or \structuredtype{} was used as a type annotation outside of a type declaration (\TypingRuleRef{TNonDecl}).
    \item A collection type was used as a type annotation outside of a global variable declaration (\TypingRuleRef{CheckIsNotCollection}).
  \end{itemize}

\hypertarget{def-badtupleindex}{}
\item[$\BadTupleIndex$]
  \textit{Bad tuple index.}
  A tuple index is out of bounds.
  See \TypingRuleRef{EGetTupleItem}.

\hypertarget{def-badslices}{}
\item[$\BadSlices$]
  \textit{Bad slices.}
  One or more bitvector slices are invalid.
  For example:
  \begin{itemize}
    \item Bitfields overlap on the left-hand side of an assignment (\TypingRuleRef{LESlice}).
    \item Bit slices that are \symbolicallyevaluable{} overlap on the left-hand side of an assignment (\TypingRuleRef{DisjointSlicesToPositions}).
      Note that if the overlapping slices are not \symbolicallyevaluable{}, then this is a \dynamicerrorterm{} ($\OverlappingSliceAssignment$).
    \item A slice expression has an empty list of slices (\TypingRuleRef{ESlice}).
    \item Bitfield slices overlap in a bitvector type declaration \\
      (\TypingRuleRef{DisjointSlicesToPositions}).
    \item A bitfield slice in a bitvector type declaration is defined with its upper index less than its lower index (\TypingRuleRef{BitfieldSliceToPositions}).
    \item A bitfield slice is (partially) out-of-bounds for its enclosing bitvector type declaration (\TypingRuleRef{CheckPositionsInWidth}).
    \item Bitfield slices in a bitvector type declaration share name and scope, but define different slices (\TypingRuleRef{CheckCommonBitfieldsAlign}).
  \end{itemize}

\hypertarget{def-badfield}{}
\item[$\BadField$]
  \textit{Bad field.}
  Invalid usage of a field of a \structuredtype{} or bitfield of a bitvector.
  For example:
  \begin{itemize}
    \item An initialization expression for a \structuredtype{} is missing a field (\TypingRuleRef{ERecord}).
    \item An access (read or write) is made to a non-existent field for a \structuredtype{} or bitfield of a bitvector type (\TypingRuleRef{LESetStructuredField}).
  \end{itemize}

\hypertarget{def-badsubprogramdeclaration}{}
\item[$\BadSubprogramDeclaration$]
  \textit{Bad subprogram declaration.}
  A subprogram declaration is invalid.
  For example:
  \begin{itemize}
    \item Incorrect declaration of parameters (\TypingRuleRef{CheckParamDecls}).
    \item Clashes with another subprogram (\TypingRuleRef{AddNewFunc})
    \item A procedure or setter returns a value (\TypingRuleRef{SReturn}).
    \item A function contains a control-flow path that does not terminate with either:
          return of a value, throwing of an exception, or an \unreachablestatementterm{} \\
      (\TypingRuleRef{CheckControlFlow}).
  \end{itemize}

\hypertarget{def-baddeclaration}{}
\item[$\BadDeclaration$]
  \textit{Bad declaration.}
  A top-level non-subprogram declaration is invalid.
  For example, there is a circular definition: a non-subprogram declaration appears in a mutually recursive set of declarations (\TypingRuleRef{TypeCheckMutuallyRec}).

\hypertarget{def-badcall}{}
\item[$\BadCall$]
  \textit{Bad call.}
  A function or procedure call is invalid.
  For example:
  \begin{itemize}
    \item The call does not match any defined subprograms \\
          (\TypingRuleRef{SubprogramForSignature}).
    \item An incorrect number of arguments or parameters was passed \\
      (\TypingRuleRef{AnnotateCallActualsTyped}).
    \item The call site expects a function or getter, but instead finds a procedure or setter, or \textit{vice versa} (\TypingRuleRef{AnnotateCallActualsTyped}).
  \end{itemize}

\hypertarget{def-sideeffectviolation}{}
\item[$\SideEffectViolation$]
  \textit{Side effect violation.}
  An error was detected by side effect analysis (\chapref{SideEffects}).
  For example:
  \begin{itemize}
    \item An impure expression was provided where a pure one was required \\
      (\TypingRuleRef{SAssert}).
    \item A non-constant-time initialization expression was provided for a constant declaration (\TypingRuleRef{SDecl}.\textsc{constant}).
  \end{itemize}

\hypertarget{def-overridingerror}{}
\item[$\OverridingError$]
  \textit{Overriding error.}
  An error was encountered during overriding.
  For example:
  \begin{itemize}
    \item Two \texttt{implementation} subprograms had clashing signatures \\ (\TypingRuleRef{CheckImplementationsUnique}).
    \item An \texttt{implementation} subprogram did not have exactly one corresponding \\
          \texttt{impdef} subprogram (\TypingRuleRef{ProcessOverrides}).
  \end{itemize}

\item[$\PrecisionLostDefining$]%
  \hypertarget{def-precisionlostdefining}{}
  \textit{Declaration with an imprecise type.}
  An attempt to declare a storage element with an implicit and imprecise type.
  See \TypingRuleRef{LDVar}.
\end{description}

\section{Dynamic Errors\label{sec:DynamicErrors}}
\begin{description}

\hypertarget{def-unreachableerror}{}
\item[$\UnreachableError$]
  \textit{Unreachable.}
  An \unreachablestatementterm\ statement was evaluated\\
    (see \SemanticsRuleRef{SUnreachable}).

\hypertarget{def-dynamicassertionfailure}{}
\item[$\DynamicAssertionFailure$]
  \textit{Dynamic assertion failure.}
  An \assertionstatementterm\ evaluated to $\False$
    (see \SemanticsRuleRef{SAssert}).

\hypertarget{def-dynamictypeassertionfailure}{}
\item[$\DynamicTypeAssertionFailure$]
  \textit{Dynamic type assertion failure.}
  A type assertion (\texttt{$e$ as $t$}) failed
    (see \SemanticsRuleRef{ATC}).

\hypertarget{def-arbitraryemptytype}{}
\item[$\ArbitraryEmptyType$]
  \textit{\ARBITRARY{} empty type.}
  An expression \texttt{ARBITRARY : $t$} is evaluated and $t$ is an empty type (see \SemanticsRuleRef{EArbitrary}).

\hypertarget{def-dynamicbadoperands}{}
\item[$\DynamicBadOperands$]
  \textit{Bad operands.}
  A primitive operator was provided invalid operands during evaluation (see \SemanticsRuleRef{UnopValues} and \SemanticsRuleRef{BinopValues}).
  For example, a division by zero or modulo by zero.

\hypertarget{def-limitexceeded}{}
\item[$\LimitExceeded$]
  \textit{Limit exceeded.}
  A loop or recursion limit was exceeded \\
  (see \SemanticsRuleRef{TickLoopLimit} and \SemanticsRuleRef{CheckRecurseLimit}).

\hypertarget{def-uncaughtexception}{}
\item[$\UncaughtException$]
  \textit{Uncaught exception.}
  An exception thrown in the specification was not caught (see \SemanticsRuleRef{EvalSpec}.\textsc{throwing}).

\hypertarget{def-badindex}{}
\item[$\BadIndex$]
  \textit{Bad index.}
  An invalid index was encountered.
  For example:
  \begin{itemize}
    \item A bitslice index was out of bounds (\SemanticsRuleRef{ReadFromBitvector}).
    \item An array index was out of bounds (\SemanticsRuleRef{EGetArray}).
  \end{itemize}

\hypertarget{def-overlappingsliceassignment}{}
\item[$\OverlappingSliceAssignment$]
  \textit{Overlapping slice assignment.}
  Bitvector slices that are not \symbolicallyevaluable{} overlap on the left-hand side of an assignment \\
  (\SemanticsRuleRef{CheckNonOverlappingSlices}).
  Note that overlapping \emph{bitfields} are \typingerrorsterm{},
  and that if the overlapping slices are \symbolicallyevaluable{} this too is a \typingerrorterm{} ($\BadSlices$).

\hypertarget{def-negativearraylength}{}
\item[$\NegativeArrayLength$]
  \textit{Negative array length.}
  The expression used to determine the length of the array evaluates to a negative integer value
  (see \SemanticsRuleRef{EArray}).
\end{description}
