\chapter{Introduction\label{chap:Introduction}}

This reference defines Arm’s Architecture Specification Language (ASL), which is the language
used in Arm’s architecture reference manuals to describe the Arm architecture.

ASL is designed and used to specify architectures. As a specification language, it is designed to be accessible,
understandable, and unambiguous to programmers, hardware
engineers, and hardware verification engineers, who collectively have quite a small intersection of languages they
all understand. It can intentionally under-specify behaviors in the architecture being described.

ASL is:
\begin{itemize}
    \item a first-order language with strong static typechecking,
    \item whitespace-insensitive, and
    \item imperative.
\end{itemize}

ASL has support for the following features (non-exhaustive list):
\begin{itemize}
    \item bitvectors:
    \begin{itemize}
        \item as a type.
        \item as a literal constant.
        \item bitvector concatenation.
        \item bitvector constants with wildcards.
        \item bitslices.
        \item dependent types to support function overloading using bitvector lengths.
        \item dependent types to reason about lengths of bitvectors.
    \end{itemize}
    \item unbounded arithmetic types “integer” and “real”;
    \item explicit non-determinism;
    \item exceptions;
    \item enumerations;
    \item arrays;
    \item records;
    \item call-by-value;
    \item type inference.
\end{itemize}

ASL does not have support for:
\begin{itemize}
    \item references or pointers;
    \item macros;
    \item templates;
    \item virtual functions.
\end{itemize}

A \emph{\specificationterm} consists of a self-contained collection of ASL code.
\identd{GVBK}
More specifically, a \specificationterm\ is the set of \globaldeclarationsterm\
written in ASL code which describe an architecture.

\section{Example Specifications}

\subsection{Example Specification 1}
\listingref{spec1} shows a small example of a specification written in ASL. It consists of the following declarations:
\begin{itemize}
    \item Global bitvectors \texttt{R0}, \texttt{R1}, and \texttt{R2} representing the state of the system.
    \item A function \texttt{MyOR} demonstrating a simple bitwise OR function of 2 bitvectors.
    \item Initialization of \texttt{R0} and \texttt{R1} bitvectors.
    \item Assignment of bitvector \texttt{R2} with the result of a function call.
\end{itemize}

\begin{center}
\lstinputlisting[caption=Example specification 1\label{listing:spec1}]{\definitiontests/spec1.asl}
\end{center}

\subsection{Example Specification 2}
\listingref{spec2} shows a small example of a specification written in ASL. It consists of the following declarations:
\begin{itemize}
\item A global variable \texttt{COUNT} representing the state of the system.
\item A procedure \texttt{ColdReset} to initialize the state of the system when power is applied and the system is reset.
    This interpretation of the function is a convention used in this particular specification. It is up to each
    specification to decide the role of each function.
\item A procedure \texttt{Step} to advance the state of the system. That is, it defines the \emph{transition relation} of the system.
    Again, this interpretation is a convention used in this particular specification, not part of the ASL language
    itself.
\end{itemize}

\begin{center}
\lstinputlisting[caption=Example specification 2\label{listing:spec2}]{\definitiontests/spec2.asl}
\end{center}

\subsection{Example Specification 3}
\listingref{spec3} shows a small example of a specification in ASL. It consists of the following declarations:
\begin{itemize}
    \item A function \texttt{Dot8} which operates on 2 bitvectors a byte at a time.
    \item A global variable \texttt{COUNT} to indicate the number of calls to the \texttt{Fib} function.
    \item A function \texttt{Fib} demonstrating recursion with a bound of 1000 on its depth.
    \item Assignment of a global bitvector \texttt{X} with a call to the \texttt{Dot8} function.
    \item Assignment of a variable from the result of a call to the recursive function \texttt{Fib}.
    \item A function \texttt{main}.
\end{itemize}

\ASLListing{Example specification 3}{spec3}{\definitiontests/spec3.asl}

\section{Structure of this Reference}
This reference defines the various constructs of ASL.
Each construct is defined via a subset of the following:
\begin{description}
    \item[Preamble] A high-level explanation of what the construct is intended for using
        prose and code examples;
    \item[Guidelines] Rules that provide informal high-level guidance.
        That is, they under-specify ASL.
        Guideline rules follow the naming convention \RequirementDefExample{Name}.
        Guideline rules that introduce syntactic sugar have the form \SyntacticSugarExample{Name};
    \item[Syntax] Rules that define how the construct is expressed in syntax;
    \item[AST] Rules that define how the construct is expressed in an abstract syntax tree (AST);
    \item[AST build rules] Rules for building an AST from parse trees.
        These rules follow the naming convention \ASTRuleDefExample{Name}
        and defined in terms of \inferencerules;
    \item[Typing rules] Rules expressing the type system.
        These rules follow the naming convention \TypingRuleDefExample{Name}.
        Typing rules are defined by a paragraph titled \emph{Prose},
        which defines the rule in prose,
        a paragraph titled \emph{Formally}, which defines the rule in terms of \inferencerules,
        and are accompanied by examples;
    \item[Dynamic semantics rules] Rules expressing the dynamic semantics.
        These rules follow the naming convention \SemanticsRuleDefExample{Name}.
        Semantics rules are defined by a paragraph titled \emph{Prose},
        which defines the rule in prose,
        a paragraph titled \emph{Formally}, which defines the rule in terms of \inferencerules,
        and accompanied by examples.
\end{description}

\subsection{Outline of the Rest of this Reference}
The rest of this document introduces elements of the ASL language
and formalizes them:
\begin{itemize}
    \item \chapref{FormalSystem} contains the mathematical definitions used
        throughout this document;
    \item \chapref{LexicalStructure} introduces the ASL lexical structure;
    \item \chapref{Syntax} introduces the ASL syntax;
    \item \chapref{AbstractSyntax} introduces the abstract syntax trees (AST)
        that represent ASL constructs.
        Familiarity with the AST is \underline{essential} for understanding the
        type system and the dynamic semantics;
    \item \chapref{TypeChecking} and \chapref{Semantics} introduce basic
        definitions needed to formalize the type system and dynamic semantics;
    \item \chapref{Literals}--\chapref{Specifications}
        define the various constructs in ASL, roughly following the structure of the AST
        in a bottom-up fashion;
    \item \chapref{TopLevel} is where all of the formalisms are used together to
        demonstrate how they can be utilized to form an interpreter for an ASL specification;
    \item \chapref{SideEffects}--\chapref{SemanticsUtilityRules} are additional
        technical chapters for aspects of the type system and dynamic semantics.
    \item \chapref{RuntimeEnvironment} describes how ASL specifications may be used
        within a runtime environment.
    \item \chapref{Errors} classifies the types of errors in ASL specifications.
\end{itemize}
