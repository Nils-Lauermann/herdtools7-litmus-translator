\section{Base Values\label{sec:BaseValues}}
\hypertarget{def-basevalueterm}{}
Each type, with the exceptions stated below, have a \basevalueterm,
which is used to initialize storage elements (either local of global),
if an initializer is not supplied.

\RequirementDef{NoBaseValue}
The following types do not have a \basevalueterm{}:
\begin{itemize}
    \item \parameterizedintegertypes{};
    \item a \wellconstrainedintegertype{} whose list of constraints
        represents the empty set;
    \item a \bitvectortypeterm{} whose length is negative.
\end{itemize}

\identi{WVQZ}
Subprogram parameters can be parameterized integers, and since they will be initialized by their
invocation, there is no need to have a \basevalueterm{} for them.

\ExampleDef{Base Values}
\listingref{base-values} shows a specification with examples of well-typed \basevalueterm{}
for various types, followed by the output to the console.
\ASLListing{Well-typed Base Values}{base-values}{\typingtests/TypingRule.BaseValue.asl}
% CONSOLE_BEGIN aslref \typingtests/TypingRule.BaseValue.asl
\begin{Verbatim}[fontsize=\footnotesize, frame=single]
global_base = 0, unconstrained_integer_base = 0, constrained_integer_base = -3
bool_base = FALSE, real_base = 0, string_base = , enumeration_base = RED
bits_base = 0x00
tuple_base = (0, -3, RED)
record_base      = {data=0x00, time=0, flag=FALSE}
record_base_init = {data=0x00, time=0, flag=FALSE}
exception_base = {msg=}
integer_array_base = [[0, 0, 0, 0]]
enumeration_array_base = [[RED=0, GREEN=0, BLUE=0]]
\end{Verbatim}
% CONSOLE_END

\listingref{base-values-parameterised} shows a specification that relies on the base value of a \bitvectortypeterm{} whose width is a \parameterizedintegertype{}.
\ASLListing{Base Value for Parameterized Bitvector Width}{base-values-parameterised}{\typingtests/TypingRule.BaseValue.parameterized.asl}

\ExampleDef{Types Without Base Value}
\listingref{base-values-bad-negative-width} shows an ill-typed specification
where the width of a bitvector is negative.
\ASLListing{No Base Value for Bitvectors of Negative Width}{base-values-bad-negative-width}{\typingtests/TypingRule.BaseValue.bad_negative_width.asl}

\listingref{base-values-bad-empty-type} shows an ill-typed specification
where the constraint \verb|5..0| represents an empty set.
Therefore, the domain of values for the type \verb|integer{5..0}| is empty,
which negates the possibility of having a \basevalueterm.
\ASLListing{No Base Value for an Empty Integer Type}{base-values-bad-empty-type}{\typingtests/TypingRule.BaseValue.bad_empty.asl}

\hypertarget{def-basevalue}{}
The function
\[
\basevalue(\overname{\staticenvs}{\tenv} \aslsep \overname{\ty}{\vt}) \aslto
\overname{\expr}{\veinit} \cup \overname{\TTypeError}{\TypeErrorConfig}
\]
returns the expression $\veinit$ which can be used to initialize a storage element
of type $\vt$ in the \staticenvironmentterm{} $\tenv$.
\ProseOtherwiseTypeError

\TypingRuleDef{BaseValue}
See \ExampleRef{Base Values} and \ExampleRef{Types Without Base Value}.

\ProseParagraph
\OneApplies
\begin{itemize}
    \item \AllApplyCase{t\_bool} \identr{CPCK}
    \begin{itemize}
        \item $\vt$ is the Boolean type;
        \item $\veinit$ is the literal expression for $\False$.
    \end{itemize}

    \item \AllApplyCase{t\_bits\_static} \identr{ZVPT}
    \begin{itemize}
        \item $\vt$ is the bitvector type with width expression $\ve$;
        \item applying $\reducetozopt$ to $\ve$ in $\tenv$ yields $\vzopt$;
        \item $\vzopt$ is not $\None$;
        \item view $\vzopt$ as the singleton integer $\length$;
        \item checking that $\length$ is greater or equal to $0$ yields $\True$\ProseTerminateAs{\NoBaseValue};
        \item $\veinit$ is the literal expression for a bitvector made of a sequence of $\length$ values of $0$.
    \end{itemize}

    \item \AllApplyCase{t\_bits\_non\_static}
    \begin{itemize}
        \item $\vt$ is the bitvector type with width expression $\ve$;
        \item applying $\reducetozopt$ to $\ve$ in $\tenv$ yields $\vzopt$;
        \item $\vzopt$ is $\None$;
        \item $\veinit$ is the literal expression for a slice of the integer literal for $0$, with start position $0$ and length $\ve$.
    \end{itemize}

    \item \AllApplyCase{t\_enum} \identr{LCCN}
    \begin{itemize}
        \item $\vt$ is the \enumerationtypeterm{} with a list of labels where $\name$ as its \head;
        \item $\name$ is bound to the literal $\vl$ by the $\constantvalues$ in the \globalstaticenvironmentterm{} of $\tenv$;
        \item $\veinit$ is the literal expression for $\vl$, that is, $\eliteral{\vl}$.
    \end{itemize}

    \item \AllApplyCase{t\_int\_unconstrained} \identr{NJDZ}
    \begin{itemize}
        \item $\vt$ is the \unconstrainedintegertype;
        \item $\veinit$ is the literal expression for $0$, that is, $\ELiteral(\LInt(0))$.
    \end{itemize}

    \item \AllApplyCase{t\_int\_parameterized} \identr{QGGH}
    \begin{itemize}
        \item $\vt$ is the \parameterizedintegertype;
        \item the result is a \typingerrorterm{} indicating the lack of a statically known base value.
    \end{itemize}

    \item \AllApplyCase{t\_int\_wellconstrained} \identr{CFTD}
    \begin{itemize}
        \item $\vt$ is the \wellconstrainedintegertype\ with a list of constraints $\cs$;
        \item define $\vzminlist$ as the concatenation of lists obtained for each
              constraint $\cs[\vi]$ in $\tenv$, for each $\vi\in\listrange(\cs)$, via $\constraintabsmin$;
        \item checking whether $\vzminlist$ is empty yields $\True$\ProseOrTypeError{\NoBaseValue};
        \item determining the minimal absolute integer in $\vzminlist$ via $\listminabs$ yields $\vzmin$;
        \item $\veinit$ is the integer literal expression for $\vzmin$.
    \end{itemize}

    \item \AllApplyCase{t\_named}
    \begin{itemize}
        \item $\vt$ is the \namedtype\ for $\id$;
        \item obtaining the \underlyingtype\ for $\id$ in $\tenv$ yields $\vtp$\ProseOrTypeError;
        \item applying $\basevalue$ to $\vtp$ in $\tenv$ yields $\veinit$\ProseOrTypeError.
    \end{itemize}

    \item \AllApplyCase{t\_real} \identr{GYCG}
    \begin{itemize}
        \item $\vt$ is the \realtypeterm{};
        \item $\veinit$ is the real literal expression for $0$.
    \end{itemize}

    \item \AllApplyCase{structured} \identr{MBRM} \identr{SVJB}
    \begin{itemize}
        \item $\vt$ is a \structuredtype\ with list of fields $\fields$;
        \item applying $\basevalue$ to $\vtefield$ in $\tenv$ for each $(\name, \vtefield)$ in $\fields$
              yields $\ve_\name$\ProseOrTypeError;
        \item $\veinit$ is the record construction expression assigning each field $\name$
              where $(\name, \vtefield)$ is an element of $\fields$ to $\vtefield$, that is, \\
              $\ERecord((\name, \vtefield) \in \fields: (\name, \ve_\name))$.
    \end{itemize}

    \item \AllApplyCase{t\_string} \identr{WKCY}
    \begin{itemize}
        \item $\vt$ is the \stringtypeterm{};
        \item $\veinit$ is the string literal expression for the empty list of characters.
    \end{itemize}

    \item \AllApplyCase{t\_tuple} \identr{QWSQ}
    \begin{itemize}
        \item $\vt$ is the \tupletypeterm{} over the list of types $\vt_{1..k}$, that is, $\TTuple(\vt_{1..k})$;
        \item applying $\basevalue$ to each type $\vt_\vi$ in $\tenv$ for $\vi=1..k$; yields the list of expressions $\ve_{1..k}$;
        \item $\veinit$ is the tuple expression $\ETuple(\ve_{1..k})$.
    \end{itemize}

    \item \AllApplyCase{t\_array\_enum}
    \begin{itemize}
        \item $\vt$ is the enumerated array type over for the enumeration $\venum$ and labels $\vlabels$ and element type $\tty$,
              that is, $\TArray(\ArrayLengthEnum(\venum, \vlabels), \tty)$ ;
        \item applying $\basevalue$ to $\tty$ in $\tenv$ yields the expression $\vvalue$\ProseOrTypeError;
        \item $\veinit$ is the array construction expression for an enumerated array with labels $\vlabels$ and initial value $\vvalue$,
              that is, $\EEnumArray\{\EArrayLabels: \vlabels, \EArrayValue: \vvalue\}$.
    \end{itemize}

    \item \AllApplyCase{t\_array\_expr} \identr{WGVR}
    \begin{itemize}
        \item $\vt$ is the array type over an integer index expression $\vlength$ and element type $\tty$, that is,
              $\TArray(\ArrayLengthExpr(\vlength), \tty)$ ;
        \item applying $\basevalue$ to $\tty$ in $\tenv$ yields the expression $\vvalue$\ProseOrTypeError;
        \item $\veinit$ is the array construction expression with length expression $\vlength$ and value expression $\vvalue$,
              that is, $\EArray\{\EArrayLength: \length, \EArrayValue: \vvalue\}$.
    \end{itemize}
\end{itemize}

\FormallyParagraph
\begin{mathpar}
\inferrule[t\_bool]{}{
    \basevalue(\tenv, \overname{\TBool}{\vt}) \typearrow \overname{\ELiteral(\LBool(\False))}{\veinit}
}
\end{mathpar}

\begin{mathpar}
\inferrule[t\_bits\_static]{
    \reducetozopt(\tenv, \ve) \typearrow \vzopt\\
    \vzopt \neq \None\\\\
    \vzopt \eqname \langle\length\rangle\\
    \checktrans{\length \geq 0}{\NoBaseValue} \checktransarrow \True\OrTypeError
}{
    \basevalue(\tenv, \overname{\TBits(\ve, \Ignore)}{\vt}) \typearrow \overname{\ELiteral(\LBitvector(i=1..\length: 0))}{\veinit}
}
\end{mathpar}

\begin{mathpar}
\inferrule[t\_bits\_non\_static]{
    \reducetozopt(\tenv, \ve) \typearrow \vzopt\\
    \vzopt = \None \\\\
    \veinit \eqdef \ESlice (\ELInt{0}, [\SliceLength(\ELInt{0}, \ve)])
}{
    \basevalue(\tenv, \overname{\TBits(\ve, \Ignore)}{\vt}) \typearrow \veinit
}
\end{mathpar}

\begin{mathpar}
\inferrule[t\_enum]{%
    \lookupconstant(\tenv, \name) \typearrow \vl
}{%
    \basevalue(\tenv, \overname{\TEnum(\name \concat \Ignore)}{\vt}) \typearrow \overname{\ELiteral(\vl)}{\veinit}
}
\end{mathpar}

\begin{mathpar}
\inferrule[t\_int\_unconstrained]{}{
    \basevalue(\tenv, \overname{\unconstrainedinteger}{\vt}) \typearrow \overname{\ELiteral(\LInt(0))}{\veinit}
}
\end{mathpar}

\begin{mathpar}
\inferrule[t\_int\_parameterized]{}{
    \basevalue(\tenv, \overname{\TInt(\parameterized(\id))}{\vt}) \typearrow \TypeErrorVal{\NoBaseValue}
}
\end{mathpar}

\begin{mathpar}
\inferrule[t\_int\_wellconstrained]{
    \cs \eqname \vc_{1..k}\\
    \vzminlist \eqdef \constraintabsmin(\tenv, \vc_1) \concat \ldots \concat \constraintabsmin(\tenv, \vc_k)\\
    \checktrans{\vzminlist \neq \emptyset}{\NoBaseValue} \typearrow \True \OrTypeError\\\\
    \listminabs(\vzminlist) \typearrow \vzmin
}{
    \basevalue(\tenv, \overname{\TInt(\wellconstrained(\cs))}{\vt}) \typearrow \overname{\ELiteral(\LInt(\vzmin))}{\veinit}
}
\end{mathpar}

\begin{mathpar}
\inferrule[t\_named]{
    \makeanonymous(\tenv, \TNamed(\id)) \typearrow \vtp \OrTypeError\\\\
    \basevalue(\tenv, \vtp) \typearrow \veinit \OrTypeError
}{
    \basevalue(\tenv, \overname{\TNamed(\id)}{\vt}) \typearrow \veinit
}
\end{mathpar}

\begin{mathpar}
\inferrule[t\_real]{}{
    \basevalue(\tenv, \overname{\TReal}{\vt}) \typearrow \overname{\ELiteral(\LReal(0))}{\veinit}
}
\end{mathpar}

\begin{mathpar}
\inferrule[structured]{
    \isstructured(\vt) \typearrow \True\\
    \vt \eqname L(\fields)\\
    (\name, \vtefield) \in \fields: \basevalue(\tenv, \vtefield) \typearrow \ve_\name \OrTypeError
}{
    \basevalue(\tenv, \vt) \typearrow \overname{\ERecord((\name, \vtefield) \in \fields: (\name, \ve_\name))}{\veinit}
}
\end{mathpar}

\begin{mathpar}
\inferrule[t\_string]{}{
    \basevalue(\tenv, \overname{\TString}{\vt}) \typearrow \overname{\ELiteral(\LString(\emptylist))}{\veinit}
}
\end{mathpar}

\begin{mathpar}
\inferrule[t\_tuple]{
    \vi=1..k: \basevalue(\tenv, \vt_\vi) \typearrow \ve_\vi \OrTypeError
}{
    \basevalue(\tenv, \overname{\TTuple}{\vt_{1..k}}) \typearrow \overname{\ETuple(\ve_{1..k})}{\veinit}
}
\end{mathpar}

\begin{mathpar}
\inferrule[t\_array\_enum]{
    \basevalue(\tenv, \tty) \typearrow \vvalue \OrTypeError
}{
    {
        \begin{array}{r}
            \basevalue(\tenv, \overname{\TArray(\ArrayLengthEnum(\venum, \vlabels), \tty)}{\vt}) \typearrow \\
            \overname{\EEnumArray\{\EArrayLabels: \vlabels, \EArrayValue: \vvalue\}}{\veinit}
        \end{array}
    }
}
\end{mathpar}

\begin{mathpar}
\inferrule[t\_array\_expr]{
    \basevalue(\tenv, \tty) \typearrow \vvalue \OrTypeError
}{
    {
        \begin{array}{r}
            \basevalue(\tenv, \overname{\TArray(\ArrayLengthExpr(\length), \tty)}{\vt}) \typearrow\\
            \overname{\EArray\{\EArrayLength: \length, \EArrayValue: \vvalue\}}{\veinit}
        \end{array}
    }
}
\end{mathpar}

\TypingRuleDef{ConstraintAbsMin}
\hypertarget{def-constraintabsmin}{}
The function
\[
    \constraintabsmin(\overname{\staticenvs}{\tenv} \aslsep \overname{\intconstraint}{\vc}) \aslto
    \overname{\Z^*}{\vzs}
    \cup \overname{\TTypeError}{\TypeErrorVal{\NoBaseValue}}
\]
returns a single element list containing the integer closest to $0$ that satisfies the constraint $\vc$ in $\tenv$, if one exists,
and an empty list if the constraint represents an empty set.
Otherwise, the result is $\TypeErrorVal{\NoBaseValue}$.

\ExampleDef{Minimal Absolute Value in a Constraint List}
The minimal absolute value of \verb|{7, -2}| is \verb|-2|.\\
%
The minimal absolute value of \verb|{2, -2}| is \verb|2|.\\
%
The minimal absolute value of \verb|{-2..2, 5}| is \verb|0|.

\ProseParagraph
\OneApplies
\begin{itemize}
    \item \AllApplyCase{exact}
    \begin{itemize}
        \item $\vc$ is the constraint given by the expression $\ve$, that is, $\ConstraintExact(\ve)$;
        \item applying $\reducetozopt$ to $\ve$ in $\tenv$ yields the optional integer $\vzopt$;
        \item checking that $\vzopt$ is not $\None$ yields $\True$\ProseTerminateAs{\NoBaseValue};
        \item view $\vzopt$ as the singleton set for the integer $\vz$;
        \item define $\vzs$ as the single element list containing $\vz$.
    \end{itemize}

    \item \AllApplyCase{range}
    \begin{itemize}
        \item $\vc$ is the constraint given by the expression $\veone$ and $\vetwo$, that is, \\
                $\ConstraintRange(\veone, \vetwo)$;
        \item applying $\reducetozopt$ to $\veone$ in $\tenv$ yields the optional integer $\vzoptone$;
        \item checking that $\vzoptone$ is not $\None$ yields $\True$\ProseTerminateAs{\NoBaseValue};
        \item view $\vzoptone$ as the singleton set for $\vvone$;
        \item applying $\reducetozopt$ to $\vetwo$ in $\tenv$ yields the optional integer $\vzopttwo$;
        \item checking that $\vzopttwo$ is not $\None$ yields $\True$\ProseTerminateAs{\NoBaseValue};
        \item view $\vzopttwo$ as the singleton set for $\vvtwo$;
        \item define $\vzs$ as based on the following cases for $\vvone$ and $\vvtwo$:
        \begin{itemize}
            \item the empty list, if $\vvone$ is greater than $\vvtwo$ (since there are no integers satisfying the constraint);
            \item the single element list for $\vvtwo$, if $\vvone$ is less than $\vvtwo$ and both are negative;
            \item the single element list for $0$, if $\vvone$ is negative and $\vvtwo$ is non-negative;
            \item the single element list for $\vvone$, if $\vvone$ is non-negative and $\vvtwo$ is greater or equal to $\vvone$.
        \end{itemize}
    \end{itemize}
\end{itemize}

\FormallyParagraph
\begin{mathpar}
\inferrule[exact]{
    \reducetozopt(\tenv, \ve) \typearrow \vzopt\\
    \checktrans{\vzopt \neq \None}{\NoBaseValue} \checktransarrow \True \OrTypeError\\\\
    \vzopt \eqname \langle\vz\rangle
}{
    \constraintabsmin(\overname{\ConstraintExact(\ve)}{\vc}) \typearrow \overname{[\vz]}{\vzs}
}
\end{mathpar}

\begin{mathpar}
\inferrule[range]{
    \reducetozopt(\tenv, \veone) \typearrow \vzoptone\\
    \checktrans{\vzoptone \neq \None}{\NoBaseValue} \checktransarrow \True \OrTypeError\\\\
    \vzoptone \eqname \langle\vvone\rangle\\
    \reducetozopt(\tenv, \vetwo) \typearrow \vzopttwo\\
    \checktrans{\vzopttwo \neq \None}{\NoBaseValue} \checktransarrow \True \OrTypeError\\\\
    \vzopttwo \eqname \langle\vvtwo\rangle\\
    \vzs \eqdef {
        \begin{cases}
           \emptylist & \vvone > \vvtwo\\
           [\vvtwo] & \vvone \leq \vvtwo < 0\\
           [0] & \vvone < 0 \leq \vvtwo\\
           [\vvone] & 0 \leq \vvone \leq \vvtwo\\
        \end{cases}
    }
}{
    \constraintabsmin(\overname{\ConstraintRange(\veone, \vetwo)}{\vc}) \typearrow \vzs
}
\end{mathpar}

\TypingRuleDef{ListMinAbs}
\hypertarget{def-listminabs}{}
The function
\[
\listminabs(\overname{\Z^*}{\vl}) \aslto \overname{\Z}{\vz}
\]
returns $\vz$ --- the integer closest to $0$ among the list
of integers in the list $\vl$. The result is biased towards positive integers. That is,
if two integers $x$ and $y$ have the same absolute value and $x$ is positive and $y$ is negative
then $x$ is considered closer to $0$.

\ExampleDef{Minimal Absolute Value}
The minimal absolute value of $[9, -3]$ is $-3$,
and the minimal absolute value of $[2, -2]$ is $2$.

\ProseParagraph
\OneApplies
\begin{itemize}
    \item \AllApplyCase{one}
    \begin{itemize}
        \item $\vl$ is the single element list for $\vz$.
    \end{itemize}

    \item \AllApplyCase{more\_than\_one}
    \begin{itemize}
        \item $\vl$ is the list where $\vzone$ is its \head\ and $\vltwo$ is its \tail;
        \item $\vltwo$ is not the empty list;
        \item applying $\listminabs$ to $\vltwo$ yields $\vztwo$;
        \item define $\vz$ based on $\vzone$ and $\vztwo$ by the following cases:
        \begin{itemize}
            \item $\vzone$ if the absolute value of $\vzone$ is less than the absolute value of $\vztwo$;
            \item $\vztwo$ if the absolute value of $\vzone$ is greater than the absolute value of $\vztwo$;
            \item $\vzone$ if $\vzone$ is equal to $\vztwo$;
            \item the absolute value of $\vzone$ if the absolute value of $\vzone$ is equal to the absolute value of $\vztwo$
                    and $\vzone$ is not equal to $\vztwo$;
        \end{itemize}
    \end{itemize}
\end{itemize}

\FormallyParagraph
\begin{mathpar}
\inferrule[one]{}{
    \listminabs(\overname{[\vz]}{\vl}) \typearrow \vz
}
\end{mathpar}

\begin{mathpar}
\inferrule[more\_than\_one]{
    \vlone \neq \emptylist\\
    \listminabs(\vltwo) = \vztwo\\
    {
        \vz \eqdef \begin{cases}
            \vzone & |\vzone| < |\vztwo|\\
            \vztwo & |\vzone| > |\vztwo|\\
            \vzone & \vzone = \vztwo\\
            |\vzone| & |\vzone| = |\vztwo| \land \vzone \neq \vztwo
        \end{cases}
    }
}{
    \listminabs(\overname{[\vzone] \concat \vltwo}{\vl}) \typearrow \vz
}
\end{mathpar}
