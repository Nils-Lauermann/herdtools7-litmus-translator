\section{ALP3}

The following changes have been made.

\subsection{ASL-625: \texttt{config}s}
\begin{itemize}
  \item Require type annotations for \texttt{config}-declared global storage elements.
  \item Permit only singular types for \texttt{config}s.
  \item Enforce constant time-frame for \texttt{config} types and initializing expressions.
  \item Remove the configuration time-frame in side-effect analysis.
  \item Describe the language intent of \texttt{config}s (see \secref{ConfigurableGlobalStorageDeclarations}).
\end{itemize}

\subsection{ASL-705: subprogram overriding}
Introduce \texttt{impdef} keyword for a subprogram that can be overridden by a corresponding \texttt{implementation} keyword (see \secref{Overriding Subprograms}).

\subsection{ASL-736: paired syntax for getters/setters}
Require getters/setters to be declared together using a dedicated syntax, as follows
(see \chapref{SubprogramDeclarations}):
\begin{lstlisting}
accessor Name{params}(args) <=> return_type
begin
  getter begin
    ... // getter implementation
  end;

  setter = value_in begin
    ... // setter implementation
  end;
end;
\end{lstlisting}

\subsection{ASL-740: enumeration labels are now literals}
Instead of treating enumeration labels as integer constants, consider them to be first-class literals (see for example \secref{DynDomain}).

\subsection{ASL-757: alternative mask syntax}
Support for an alternative mask syntax as follows:
\begin{lstlisting}
x == '1(0)(0)1'                // equivalent to x == '1xx1'
x == '1(00)1'                  // equivalent to x == '1xx1'
x == '1(01)1'                  // equivalent to x == '1xx1'
y != '0(1)1(1)'                // equivalent to y != '0x1x'
x IN {'1(0)(0)1', '0(1)1(1)'}  // equivalent to x IN {'1xx1', '0x1x'}
\end{lstlisting}

\subsection{ASL-762: remove ``primitive subprograms''}
\begin{itemize}
  \item Permit multiplication of reals and integers (or integers and reals), returning a real.
  \item Implement the following functions in ASL itself: \texttt{UInt}, \texttt{SInt}, \texttt{Real}, \texttt{RoundDown}, \texttt{RoundUp}, \texttt{RoundTowardsZero}.
  \item Remove the concept of ``primitive subprograms'' from the ASL language, leaving them as implementation-specific details.
\end{itemize}

\subsection{ASL-772, ASL-773: constraint set limits}
\begin{itemize}
  \item Introduce a new limit on exploding constraint sets \\ (see \TypingRuleRef{AnnotateConstraintBinop}).
  \item Produce errors when constraints that have lost precision are ``silently'' propagated (see uses of \TypingRuleRef{CheckNoPrecisionLoss}). For example:
  \begin{lstlisting}
    // ERROR - lost precision when assigning to implicitly constrained integer
    let w = UInt(Zeros{64}) * UInt(Zeros{64})

    // ERROR - lost precision when assigning to pending constrained integer
    let x : integer{-} = UInt(Zeros{64}) * UInt(Zeros{64})

    // OK - lost precision, but still within annotated constraints
    let y : integer{0..2^128} = UInt(Zeros{64}) * UInt(Zeros{64})

    // OK - lost precision when assigning to unconstrained integer
    let z : integer = UInt(Zeros{64}) * UInt(Zeros{64})
  \end{lstlisting}
\end{itemize}

\subsection{ASL-780: \texttt{collection}s}
Introduce a variant of \texttt{record}s using the keyword \texttt{collection}.
These behave as records for sequential semantics, but are viewed as a disjoint collection of storage elements for concurrent semantics.
See \secref{CollectionTypes}, \TypingRuleRef{EGetCollectionField}, and \SemanticsRuleRef{EGetCollectionFields} for example.

\subsection{ASL-781: name clashing}
Permit name clashes between subprograms and other identifiers, allowing reuse of subprogram names as storage element names.

\subsection{ASL-782: evaluation order}
Where there is a choice of evaluation order, ASL now specifies an ordering.
See \secref{EvaluationOrder}.

\subsection{ASL-785: pending constrained global storage elements}
Extend the \texttt{integer\{-\}} syntax for pending constrained integers to global storage elements.

\subsection{Updated standard library}
Implemented the following functions:
\begin{itemize}
  \item ASL-763: \texttt{AlignDownSize}, \texttt{AlignUpSize}, \texttt{AlignDownP2}, and \texttt{AlignUpP2} (for both integers and bitvectors)
  \item ASL-778: \texttt{LowestSetBitNZ} and \texttt{HighestSetBitNZ}
  \item ASL-786: \texttt{ROL} and \texttt{ROL\_C}
  \item ASL-787: \texttt{FloorLog2} and \texttt{CeilLog2}
\end{itemize}

\noindent
Modified the following signatures (ASL-788):
\begin{itemize}
  \item \texttt{UInt}:
    \begin{itemize}
      \item previously \begin{small}\verb|UInt{N} (x: bits(N)) => integer{0..2^N-1}|\end{small}
      \item now \begin{small}\verb|UInt{N: integer{1..128}}|\end{small}
    \end{itemize}
  \item \texttt{SInt}:
    \begin{itemize}
      \item previously \begin{small}\verb|SInt{N} (x: bits(N)) => integer{-(2^(N-1)) .. 2^(N-1)-1}|\end{small}
      \item now \begin{small}\verb|SInt{N: integer{1..128}}|\end{small}
    \end{itemize}
    \item \texttt{AlignDownSize}:
    \begin{itemize}
      \item previously
      \begin{small}\verb|AlignDownSize{N}(x: bits(N), size: integer {1..2^N}) => bits(N)|\end{small}
      \item now \begin{small}\verb|AlignDownSize{N: integer{1..128}}(x: bits(N), size: integer {1..2^N}) => bits(N)|\end{small}
    \end{itemize}
    \item \texttt{AlignUpSize}:
    \begin{itemize}
      \item previously \begin{small}\verb|AlignUpSize{N}(x: bits(N), size: integer {1..2^N}) => bits(N)|\end{small}
      \item now \begin{small}\verb|AlignUpSize{N: integer{1..128}}(x: bits(N), size: integer {1..2^N}) => bits(N) ...|\end{small}
    \end{itemize}
\end{itemize}

\noindent
Removed the following functions:
\begin{itemize}
  \item ASL-787: \texttt{Log2}
\end{itemize}
